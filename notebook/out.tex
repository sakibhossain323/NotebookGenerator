\newcommand\TEAM{IUT Pounopunik}
\newcommand\UNI{Islamic University of Technology}
\newcommand\COLS{3}
\newcommand\ORT{landscape}
\newcommand\FSZ{12}
\documentclass[FSZ,a4paper,onesided]{article}
\usepackage[utf8]{inputenc}
\usepackage{amsmath}
\usepackage{listings}
\usepackage{graphicx}
\usepackage{multicol}
\usepackage[utf8]{inputenc}
\usepackage[english]{babel}
\usepackage[usenames,dvipsnames]{color}
\usepackage{verbatim}
\usepackage{hyperref}
\usepackage{geometry}
\usepackage{fancyhdr}
\usepackage{titlesec}
\usepackage{pdflscape}
\geometry{verbose,\ORT,a4paper,tmargin=1.0cm,bmargin=.5cm,lmargin=.5cm,rmargin=.5cm, headsep=.5cm}

\definecolor{dkgreen}{rgb}{0,0.6,0}
\definecolor{gray}{rgb}{0.5,0.5,0.5}
\definecolor{mauve}{rgb}{0.58,0,0.82}

\lstset{frame=tb,
  language=C++,
  aboveskip=1mm,
  belowskip=1mm,
  showstringspaces=false,
  columns=flexible,
  basicstyle={\small\ttfamily},
  numbers=none,
  numberstyle=\footnotesize\color{gray},
  keywordstyle=\color{blue},
  commentstyle=\color{dkgreen},
  stringstyle=\color{mauve},
  breaklines=true,
  breakatwhitespace=false,
  tabsize=1
}

\fancyhf{}
\renewcommand{\headrulewidth}{1pt}
\pagestyle{fancy}
\lhead{\large{\textbf{\TEAM}, \textbf{\UNI}}}
\rhead{\thepage}

\titleformat*{\section}{\large\bfseries}
\titleformat*{\subsection}{\normalsize\bfseries}
\titleformat*{\subsubsection}{\normalsize}
\titlespacing*{\section}
{0pt}{0ex}{0ex}
\titlespacing*{\subsection}
{0pt}{0ex}{0ex}
\titlespacing*{\subsubsection}
{0pt}{0ex}{0ex}
\setlength{\columnsep}{0.05in}
\setlength{\columnseprule}{1px}


\begin{document}

\begin{multicols*}{\COLS}
\pagenumbering{gobble}
\tableofcontents
\newpage
\pagenumbering{arabic}
\lstloadlanguages{C++,Java}
\subsection*{Sublime Build}
\begin{lstlisting}[language= Pascal, commentstyle=\color{black}, numberstyle=\tiny\color{black}, keywordstyle=\color{black}, stringstyle=\color{black},
]
{
    "shell_cmd": "g++ -std=c++17 -o ${file_path}/${file_base_name} ${file} && ${file_path}/${file_base_name} < input.txt > output.txt",
    "working_dir": "${file_path}",
    "selector": "source.cpp"
}
\end{lstlisting}
\subsection*{Sublime Build Ubuntu}

\begin{lstlisting}[language= Pascal, commentstyle=\color{black}, numberstyle=\tiny\color{black}, keywordstyle=\color{black}, stringstyle=\color{black},
]
{
"cmd" : ["g++ -std=c++20 -DLOCAL $file_name -o $file_base_name 
&&timeout 4s ./$file_base_name<inputf.in>outputf.in"],
"selector" : "source.cpp",
"file_regex": "^(..[^:]*):([0-9]+):?([0-9]+)?:? (.*)$",
"shell": true,
"working_dir" : "$file_path"
}
\end{lstlisting}

\subsection*{Stress-tester}
\begin{lstlisting}[language= Pascal, commentstyle=\color{black}, numberstyle=\tiny\color{black}, keywordstyle=\color{black}, stringstyle=\color{black},
]
#!/bin/bash
# Call as stresstester ITERATIONS [--count]

g++ gen.cpp -o gen
g++ sol.cpp -o sol
g++ brute.cpp -o brute

for i in $(seq 1 "$1") ; do
    echo "Attempt $i/$1"
    ./gen > in.txt
    ./sol < in.txt > out1.txt
    ./brute < in.txt > out2.txt
    diff -y out1.txt out2.txt > diff.txt
    if [ $? -ne 0 ] ; then
        echo "Differing Testcase Found:"; cat in.txt
        echo -e "\nOutputs:"; cat diff.txt
        break
    fi
done
\end{lstlisting}


\section{All Macros}
\begin{lstlisting}
/*--- DEBUG TEMPLATE STARTS HERE ---*/
void show(int x) {cerr << x;}
void show(long long x) {cerr << x;}
void show(double x) {cerr << x;}
void show(char x) {cerr << '\'' << x << '\'';}
void show(const string &x) {cerr << '\"' << x << '\"';}
void show(bool x) {cerr << (x ? "true" : "false");}

template<typename T, typename V>
void show(pair<T, V> x) { cerr << '{'; show(x.first); cerr << ", "; show(x.second); cerr << '}'; }
template<typename T>
void show(T x) {int f = 0; cerr << "{"; for (auto &i: x) cerr << (f++ ? ", " : ""), show(i); cerr << "}";}

void debug_out(string s) {
  s.clear();
  cerr << s << '\n';
}
template <typename T, typename... V>
void debug_out(string s, T t, V... v) {
  s.erase(remove(s.begin(), s.end(), ' '), s.end());
  cerr << "        "; // 8 spaces
  cerr << s.substr(0, s.find(','));
  s = s.substr(s.find(',') + 1);
  cerr << " = ";
  show(t);
  cerr << endl;
  if(sizeof...(v)) debug_out(s, v...);
}
#define dbg(x...) cerr << "LINE: " << __LINE__ << endl; debug_out(#x, x); cerr << endl; 
/*--- DEBUG TEMPLATE ENDS HERE ---*/
//#pragma GCC optimize("Ofast")
//#pragma GCC optimization ("O3")
//#pragma comment(linker, "/stack:200000000")
//#pragma GCC optimize("unroll-loops")
//#pragma GCC target("sse,sse2,sse3,ssse3,sse4,popcnt,abm,mmx,avx,tune=native")

#include <bits/stdc++.h>
#include <ext/pb_ds/assoc_container.hpp>
#include <ext/pb_ds/tree_policy.hpp>
using namespace std;
using namespace __gnu_pbds;
  //find_by_order(k) --> returns iterator to the kth largest element counting from 0
  //order_of_key(val) --> returns the number of items in a set that are strictly smaller than our item
template <typename DT> 
using ordered_set = tree <DT, null_type, less<DT>, rb_tree_tag,tree_order_statistics_node_update>;
mt19937 rnd(chrono::steady_clock::now().time_since_epoch().count());

#ifdef LOCAL
#include "dbg.h"
#else
#define dbg(x...)
#endif

int main() {
  cin.tie(0) -> sync_with_stdio(0);
}
\end{lstlisting}
\section{Data Structure}
\subsection{Sparse Table}
\begin{lstlisting}
ll spars[MAX][18];

void build(vector<ll>& a) //1-indexed
{
  int n = a.size();
  for(int i = 1; i <= n; i++) spars[i][0] = a[i-1];

  for(int p = 1; p <= 18; p++)
  {
    for(int i = 1; i+(1<<p) - 1 <= n; i++)
    {
      spars[i][p] = min(spars[i][p-1], spars[i+(1<<(p-1))][p-1]);
    } 
  }
}

ll query(int l, int r)
{
  int p = 31 - __builtin_clz(r-l+1);
  return min(spars[l][p], spars[r-(1<<p)+1][p]);
}\end{lstlisting}
\subsection{BIT}
\begin{lstlisting}
template <typename T> class BIT 
{
  public:
    int n; vector<T> tree;

    BIT(int size) // 1-indexed
    {
      n = size; tree.assign(n+1, 0);
    }

    BIT(const vector<T> &a) : BIT(a.size())
    {
      for(int i = 1; i <= n; i++) update(i, a[i-1]);
    }

    T query(int i)
    {
      T ans = 0;
      for( ; i >= 1; i-= (i & -i)) ans+= tree[i];
      return ans;
    }

    T query(int l, int r)
    {
      return query(r) - query(l-1);
    }

    void update(int i, T delta)
    {
      for( ; i <= n; i+= (i & -i)) tree[i]+= delta;
    }
};\end{lstlisting}
\subsection{Lazy SegmentTree}
\begin{lstlisting}
ll tree[4*MAX], lazy[4*MAX]; // 1-indexed

void build(vector<ll>&a, int b = 0, int e = -1, int v=1)
{
  if(e == -1) e = a.size()-1;

  if(b == e)
  {
    tree[v] = a[b];
    return;
  }
  
  int mid = (b+e)/2;
  build(a, b, mid, 2*v);
  build(a, mid+1, e,2*v+1);
  tree[v] = tree[2*v] + tree[2*v+1];
}

ll query(int l, int r, int b, int e, int v=1, ll carry = 0)
{
  if(b > r || e < l) return 0;
  if(b >= l && e <= r) return tree[v]+carry*(e-b+1);

  int mid = (b+e)/2;
  ll lseg = query(l, r, b, mid, 2*v, carry+lazy[v]);
  ll rseg = query(l, r, mid+1, e, 2*v+1, carry+lazy[v]);
  return lseg + rseg;
}

void update(int l, int r, ll val, int b, int e, int v = 1)
{
  if(b > r || e < l) return;
  if(b >= l && e <= r)
  {
    tree[v]+= (e-b+1)*val;
    lazy[v]+= val;
    return;
  }

  int mid = (b+e)/2;
  update(l, r, val, b, mid, 2*v);
  update(l, r, val, mid+1, e, 2*v+1);
  tree[v] = tree[2*v] + tree[2*v+1]+ (e-b+1)*lazy[v];
}\end{lstlisting}
\subsection{Generic SegmentTree}
\begin{lstlisting}
template<typename ST, typename LZ> 
class SegmentTree {
private:
  int n;
  ST *tree, identity;
  ST (*merge) (ST, ST);

  LZ *lazy, unmark;
  void (*mergeLazy)(int, int, LZ&, LZ);
  void (*applyLazy)(int, int, ST&, LZ);

  void  build(vector<ST> &arr, int lo, int hi, int cur=1)
  {
    if(lo == hi)
    {
      tree[cur] = arr[lo-1];
      return; 
    }
    int mid = (hi+lo)/2, left = 2*cur, right = 2*cur+1;
    build(arr, lo, mid, left);
    build(arr, mid+1, hi, right);
    tree[cur] = merge(tree[left], tree[right]);
  }

  void propagate(int lo, int hi, int cur)
  {
    applyLazy(lo, hi, tree[cur], lazy[cur]);
    if(lo < hi)
    {
      int mid = (lo+hi)/2, left = 2*cur, right = 2*cur+1;
      mergeLazy(lo, mid, lazy[left], lazy[cur]);
      mergeLazy(mid+1, hi, lazy[right], lazy[cur]);
    }
    lazy[cur] = unmark;
  }

  void update(int from, int upto, LZ delta, int lo, int hi, int cur=1)
  {
    if(lo>hi) return;
    
    propagate(lo, hi, cur);
    if(from > hi or upto < lo) return;
    if(from<= lo and upto >= hi)
    {
      mergeLazy(lo, hi, lazy[cur], delta);
      propagate(lo, hi, cur);
      return;
    }
    int mid = (lo+hi)/2, left = 2*cur, right = 2*cur+1;
    update(from, upto, delta, lo, mid, left);
    update(from, upto, delta, mid+1, hi, right);
    tree[cur] = merge(tree[left], tree[right]);
  }

  ST query(int from, int upto, int lo, int hi, int cur=1)
  {
    if(lo>hi) return identity;

    propagate(lo, hi, cur);
    if(from > hi or upto < lo) return identity;
    if(from<= lo and upto >= hi) return tree[cur];
    int mid = (lo+hi)/2, left = 2*cur, right = 2*cur+1;
    ST lseg = query(from, upto, lo, mid, left);
    ST rseg = query(from, upto, mid+1, hi, right);
    return merge(lseg, rseg);
  }

public:
  SegmentTree(
    vector<ST> arr, ST (*merge) (ST, ST), ST identity,
    void (*mergeLazy)(int, int, LZ&, LZ),
    void (*applyLazy)(int, int, ST&, LZ), LZ unmark
  ):
    n(arr.size()), merge(merge), identity(identity),
    mergeLazy(mergeLazy), applyLazy(applyLazy), unmark(unmark)
  {
    tree = new ST[n*4];
    lazy = new LZ[n*4];
    build(arr, 1, n);
    fill(lazy, lazy+n*4, unmark);
  }

  void update(int from, int upto, LZ delta)
  {
    update(from, upto, delta, 1, n);
  }

  ST query(int from, int upto)
  {
    return query(from, upto, 1, n);
  }

  ~SegmentTree()
  {
    delete[] tree;
    delete[] lazy;
  }
};


ll add(ll l, ll r) { return l+r;}
void mergeAdd(int lo, int hi, ll &cur, ll pending) { cur+= pending;}
void applyAdd(int lo, int hi, ll &cur, ll pending) { cur+= pending*(hi-lo+1);}

void solve(int tcase)
{
  vector<ll> a(n);   
  SegmentTree<ll, ll> st(a, add, 0, mergeAdd, applyAdd, 0);
}\end{lstlisting}
\subsection{MO}
\begin{lstlisting}
struct node {
  LL l, r, idx;
};
bool cmp(const node &x, const node &y) {
  return x.r < y.r;
}
void add(LL x) {
  if(mp[x] % 2) curr++;
  mp[x]++;
}
void diminish(LL x) {
  if(mp[x] % 2 == 0) curr--;
  mp[x]--;
}
void solve()
{
  BLOCK_SIZE = sqrt(n) + 1;
  rep(i, 0, q-1) {
    LL x, y; cin >> x >> y;
    x--; y--;
    query[x / BLOCK_SIZE].pb({x, y, i});
    m = max(m, x / BLOCK_SIZE);
  }
  rep(i, 0, m) sort(all(query[i]), cmp);
  LL mo_left = 0, mo_right = -1;
  rep(i, 0, m) {
    for(auto [left, right, id] : query[i]) {
      while(mo_right < right) add(v[++mo_right]);
      while(mo_right > right) diminish(v[mo_right--]);
      while(mo_left < left) diminish(v[mo_left++]);
      while(mo_left > left) add(v[--mo_left]);
      answer[id] = curr;
    }
  }
  rep(i, 0, q-1) cout << answer[i] << endl;
}\end{lstlisting}
\subsection{MergeSort Tree}
\begin{lstlisting}
vector<LL> tree[5*MAXN];
LL A[N];
void build_tree(LL now , LL curLeft, LL curRight) {
          if(curLeft == curRight) {
                tree[now].push_back(A[curLeft]);
                return;
          }
          LL mid = (curLeft + curRight) / 2;
          build_tree(2 * now, curLeft, mid);
          build_tree(2 * now + 1, mid + 1 , curRight);
          tree[now] = merge(tree[2 * now] , tree[2 * now + 1]);
}
LL query(LL now, LL curLeft, LL curRight, LL l, LL r, LL k) {
        if(curRight < l || curLeft > r) return 0;
        if(curLeft >= l && curRight <= r)
                Return lower_bound(tree[now].begin(), tree[now].end(), k) - tree[now].begin();
        LL mid = (curLeft + curRight) / 2;
        return query(2 * now, curLeft, mid, l, r, k) + query(2 * now + 1, mid + 1, curRight, l, r, k);
}\end{lstlisting}
\subsection{BIT2d}
\begin{lstlisting}
const int N = 1008;
int bit[N][N], n, m;
int a[N][N], q;
void update(int x, int y, int val) {
  for (; x < N; x += -x & x)
    for (int j = y; j < N; j += -j & j) bit[x][j] += val;
}
int get(int x, int y) {
  int ans = 0;
  for (; x; x -= x & -x)
    for (int j = y; j; j -= j & -j) ans += bit[x][j];
  return ans;
}
int get(int x1, int y1, int x2, int y2) {
  return get(x2, y2) - get(x1 - 1, y2) - get(x2, y1 - 1) + get(x1 - 1, y1 - 1);
}
\end{lstlisting}
\subsection{SparseTable2d}
\begin{lstlisting}
#include <bits/stdc++.h>
using namespace std;

const int MAXN = 505;
const int LOGN = 9;

// O(n^2 (logn)^2
// Supports Rectangular Query
int A[MAXN][MAXN];
int M[MAXN][MAXN][LOGN][LOGN];

void Build2DSparse(int N) {
    for (int i = 1; i <= N; i++) {
        for (int j = 1; j <= N; j++) {
            M[i][j][0][0] = A[i][j];
        }
        for (int q = 1; (1 << q) <= N; q++) {
            int add = 1 << (q - 1);
            for (int j = 1; j + add <= N; j++) {
                M[i][j][0][q] = max(M[i][j][0][q - 1], M[i][j + add][0][q - 1]);
            }
        }
    }

    for (int p = 1; (1 << p) <= N; p++) {
        int add = 1 << (p - 1);
        for (int i = 1; i + add <= N; i++) {
            for (int q = 0; (1 << q) <= N; q++) {
                for (int j = 1; j <= N; j++) {
                    M[i][j][p][q] = max(M[i][j][p - 1][q], M[i + add][j][p - 1][q]);
                }
            }
        }
    }
}

// returns max of all A[i][j], where x1<=i<=x2 and y1<=j<=y2
int Query(int x1, int y1, int x2, int y2) {
    int kX = log2(x2 - x1 + 1);
    int kY = log2(y2 - y1 + 1);
    int addX = 1 << kX;
    int addY = 1 << kY;

    int ret1 = max(M[x1][y1][kX][kY], M[x1][y2 - addY + 1][kX][kY]);
    int ret2 = max(M[x2 - addX + 1][y1][kX][kY],
                                  M[x2 - addX + 1][y2 - addY + 1][kX][kY]);
    return max(ret1, ret2);
}
\end{lstlisting}
\subsection{SegmentTree}
\begin{lstlisting}

#include <bits/stdc++.h>

using namespace std;
using ll = long long;
using pii = pair<ll, ll>;

const ll N=5e5+5, mod=998244353;

using treenode = ll;
using lazynode = pii;
#define fir(a) for(int i=0; i<a; i++)

treenode treeidn = 0;
lazynode lazyidn = {1, 0};

vector<ll> v(N);
vector<treenode> tree(4*N, treeidn);
vector<lazynode> lazy(4*N, lazyidn);

treenode merge(treenode &a, treenode &b){
  return ;//add merge function here
}
void lazyapply(treenode &to, ll l, ll r, lazynode &fr){
  //apply lazy to treenode here
}
void lazymerge(lazynode &to, lazynode &fr){
  //combine to lazy updates here
}

//-------dont touch: start
void build(ll id, ll l, ll r){
  if(l==r){
    tree[id]=v[l];
    return;
  }
  ll m=(l+r)/2;
  build(id*2+1, l, m);
  build(id*2+2, m+1, r);
  tree[id] = merge(tree[id*2+1], tree[id*2+2]);
}
void push(ll id, ll l, ll r){
  if(l-r){
    ll m=(l+r)/2;
    lazyapply(tree[2*id+1], l, m, lazy[id]);
    lazymerge(lazy[2*id+1], lazy[id]);

    lazyapply(tree[2*id+2], m+1, r, lazy[id]);
    lazymerge(lazy[2*id+2], lazy[id]);

    lazy[id]=lazyidn;
  }
}
treenode query(ll id, ll l, ll r, ll ql, ll qr){
  push(id, l, r);
  if(ql<=l && r<=qr) return tree[id];
  if(ql>r || qr<l) return treeidn;
  
  ll m=(l+r)/2;
  treenode tl=query(id*2+1, l, m, ql, qr);
  treenode tr=query(id*2+2, m+1, r, ql, qr);
  return merge(tl, tr);
}
void update(ll id, ll l, ll r, ll ul, ll ur, lazynode uv){
  push(id, l, r);
  if(ul<=l && r<=ur){
    lazyapply(tree[id], l, r, uv);
    lazymerge(lazy[id], uv);
    return;
  }
  if(ul>r || ur<l) return;
  
  ll m=(l+r)/2;
  update(id*2+1, l, m, ul, ur, uv);
  update(id*2+2, m+1, r, ul, ur, uv);
  tree[id]=merge(tree[id*2+1], tree[id*2+2]);
  return;
}
//--------dont touch: end

void solve(){
  ll n, q; cin>>n>>q;
  fir(n) cin>>v[i];

  build(0, 0, n-1);

  while(q--){
    ll t; cin>>t;
    if(t){
      ll l, r; cin>>l>>r;
      cout<<query(0, 0, n-1, l, r-1)<<"\n";
    }else{
      ll l, r, a, b; cin>>l>>r>>a>>b;
      update(0, 0, n-1, l, r-1, {a, b});
    }
  }
  return;
}
\end{lstlisting}
\subsection{SQRT Decomp}
\begin{lstlisting}
// input data
int n;
vector<int> a (n);
// preprocessing
int len = (int) sqrt (n + .0) + 1; // size of the block and the number of blocks
vector<int> b (len);
for (int i=0; i<n; ++i)
        b[i / len] += a[i];
// answering the queries
for (;;) {
        int l, r;
    // read input data for the next query
        int sum = 0;
        for (int i=l; i<=r; )
                if (i % len == 0 && i + len - 1 <= r) {
                        // if the whole block starting at i belongs to [l, r]
                        sum += b[i / len];
                        i += len;
                }
                else {
                        sum += a[i];
                        ++i;
                }
}

int sum = 0;
int c_l = l / len,   c_r = r / len;
if (c_l == c_r)
        for (int i=l; i<=r; ++i)
                sum += a[i];
else {
        for (int i=l, end=(c_l+1)*len-1; i<=end; ++i)
                sum += a[i];
        for (int i=c_l+1; i<=c_r-1; ++i)
                sum += b[i];
        for (int i=c_r*len; i<=r; ++i)
                sum += a[i];
}


void remove(idx);  // TODO: remove value at idx from data structure
void add(idx);     // TODO: add value at idx from data structure
int get_answer();  // TODO: extract the current answer of the data structure
int block_size;
struct Query {
        int l, r, idx;
        bool operator<(Query other) const
        {
                return make_pair(l / block_size, r) <
                              make_pair(other.l / block_size, other.r);
        }
};
vector<int> mo_s_algorithm(vector<Query> queries) {
        vector<int> answers(queries.size());
        sort(queries.begin(), queries.end());

        // TODO: initialize data structure

        int cur_l = 0;
        int cur_r = -1;
        // invariant: data structure will always reflect the range [cur_l, cur_r]
        for (Query q : queries) {
                while (cur_l > q.l) {
                        cur_l--;
                        add(cur_l);
                }
                while (cur_r < q.r) {
                        cur_r++;
                        add(cur_r);
                }
                while (cur_l < q.l) {
                        remove(cur_l);
                        cur_l++;
                }
                while (cur_r > q.r) {
                        remove(cur_r);
                        cur_r--;
                }
                answers[q.idx] = get_answer();
        }
        return answers;
}\end{lstlisting}
\section{Graph}
\subsection{DSU BySize}
\begin{lstlisting}
vector<int> parent, setSize;
void make_set(int v) {
  parent[v] = v;
  setSize[v] = 1;
}
int find_set(int v) {
  if (v == parent[v]) return v;
  return parent[v] = find_set(parent[v]);
}
void union_sets(int a, int b) {
  a = find_set(a);
  b = find_set(b);
if (a != b) {
    if (setSize[a] < setSize[b]) swap(a, b);
    parent[b] = a;
    setSize[a] += setSize[b];
  }
}
int main() {
  int n;
  cin >> n;
  parent.resize(n);
  setSize.resize(n);
  for (int i = 0; i < n; i++)  make_set(i);
}\end{lstlisting}
\subsection{MST Kruskal}
\begin{lstlisting}
const ll sz = 1e5 + 7;
vector<ll> pr(sz);

ll find(ll x) {
  if (pr[x] == x) return x;
  return pr[x] = find(pr[x]);
}

void _union(ll x, ll y) {
  pr[find(y)] = find(x);
}

signed main() {
  ll n, m, i;
  cin >> n >> m;
  vector<tuple<ll, ll, ll>> edg(m);
  iota(pr.begin(), pr.begin() + n + 1, 0);
  for (auto &[w, u, v] : edg) cin >> u >> v >> w;
  sort(edg.begin(), edg.end());

  ll cost = 0;
  for (auto [w, u, v] : edg) {
    if (find(u) != find(v)) {
      _union(u, v);
      cost += w;
    }
  }
  for (i = 1; i < n; i++) {
    if (find(i) != find(i + 1)) {
      cout << "IMPOSSIBLE\n";
      return 0;
    }
  }
  cout << cost << "\n";
}
\end{lstlisting}
\subsection{Dijkstra}
\begin{lstlisting}
using pll = pair<ll, ll>;
vector<pll> adj[MAX];
vector<ll> dist(MAX, INF);
vector<ll> par(MAX, -1);

void dijkstra(int src)
{
  dist[src] = 0;
  priority_queue<pll, vector<pll>, greater<pll>> pq;
  pq.push({0, src});

  while(!pq.empty())
  {
    auto [d, u] = pq.top();
    pq.pop();

    if(d > dist[u]) continue;

    for(auto &[v, w]: adj[u])
    {
      if(dist[u]+w < dist[v])
      {
        dist[v] = dist[u]+w;
        par[v] = u; 
        pq.push({dist[v], v});
      }
    }
  }
}
\end{lstlisting}
\subsection{Bellman Ford}
\begin{lstlisting}
#define sz 100007
ll INF = 1e18;
vector<tuple<ll, ll, ll>> edg;
vector<ll> dis(sz, INF);

void bellman_ford(ll n) {
  ll i, brk;
  dis[1] = 0ll;
  for (i = 1; i <= n; i++) {
    brk = 0;
    for (auto [u, v, w] : edg) {
      if (dis[v] > dis[u] + w)
        dis[v] = dis[u] + w; // for directional graph
      else
        brk++;
    }
    if (brk == n)
      break; // optimization
  }
}

bellman_ford(n);
\end{lstlisting}
\subsection{Floyd Warshall}
\begin{lstlisting}
vector<vector<ll>> w(sz, vector<ll>(sz, inf));

void floyd_warshall(ll n) {
  ll i, j, k;
  for (i = 1; i <= n; i++)
    w[i][i] = 0;
  for (k = 1; k <= n; k++) {
    for (i = 1; i <= n; i++) {
      for (j = 1; j <= n; j++) {
        w[j][i] = w[i][j] = min(w[i][j], w[i][k] + w[k][j]); // for bidirectional graph
      }
    }
  }
}

w[b][a] = w[a][b] = min(c, w[a][b]); // for bidirectional graph
floyd_warshall(n);
\end{lstlisting}
\subsection{SCC}
\begin{lstlisting}
#include <bits/stdc++.h>
using namespace std;

#define ll long long

const ll sz = 1e5 + 7;
vector<ll> adj[sz];
vector<ll> Radj[sz];
vector<bool> vis(sz);
vector<ll> ord;

void dfs1(ll cur) {
  vis[cur] = 1;
  for (auto nxt : adj[cur]) {
    if (!vis[nxt])
      dfs1(nxt);
  }
  ord.push_back(cur);
}

void dfs2(ll cur) {
  vis[cur] = 1;
  for (auto nxt : Radj[cur]) {
    if (!vis[nxt])
      dfs2(nxt);
  }
}

signed main() {
  // strongly connected component kosaraju's algorithm
  ll n, m, a, b, i;
  cin >> n >> m;
  for (i = 0; i < m; i++) {
    cin >> a >> b;
    adj[a].push_back(b);
    Radj[b].push_back(a);
  }
  for (i = 1; i <= n; i++) {
    if (!vis[i])
      dfs1(i);
  }
  reverse(ord.begin(), ord.end());
  for (i = 1; i <= n; i++)
    vis[i] = 0;
  vector<ll> scc;
  for (auto e : ord) {
    if (!vis[e]) {
      scc.push_back(e);
      dfs2(e);
    }
  }
  if (scc.size() == 1) {
    cout << "YES\n";
    return 0;
  }
  cout << "NO\n";

  for (i = 1; i <= n; i++)
    vis[i] = 0;
  dfs2(scc[0]);
  if (vis[scc[1]])
    cout << scc[0] << " " << scc[1] << "\n";
  else
    cout << scc[1] << " " << scc[0] << "\n";
  return 0;
}
\end{lstlisting}
\subsection{LCA}
\begin{lstlisting}
LL n, l, timer;
vector<vector<LL>> adj;
vector<LL> tin, tout;
vector<vector<LL>> up;
void dfs(LL v, LL p) {
  tin[v] = ++timer;
  up[v][0] = p;
  for (LL i = 1; i <= l; ++i)
    up[v][i] = up[up[v][i-1]][i-1];
  for (LL u : adj[v]) {
    if (u != p) dfs(u, v);
  }
  tout[v] = ++timer;
}
bool is_ancestor(LL u, LL v)  {
  return tin[u] <= tin[v] && tout[u] >= tout[v];
}
LL lca(LL u, LL v) {
  if (is_ancestor(u, v)) return u;
  if (is_ancestor(v, u)) return v;
  for (LL i = l; i >= 0; --i) {
    if (!is_ancestor(up[u][i], v)) u = up[u][i];
  }
  return up[u][0];
}
void preprocess(LL root) {
  tin.resize(n);
  tout.resize(n);
  timer = 0;
  l = ceil(log2(n));
  up.assign(n, vector<LL>(l + 1));
  dfs(root, root);
}\end{lstlisting}
\subsection{EulerTourTree}
\begin{lstlisting}

using ll = long long;
using vi = vector<ll>;
using grid = vector<vi>;

void et(grid &edg, ll at, ll pt, grid &tr, ll &id){
  tr[0][id]=at; //val[at];
  tr[1][at]=id++;

  for(ll to: edg[at]) if(to-pt){
    et(edg, to, at, tr, id);
  }
  tr[0][id]=at; //val[at];
  tr[2][at]=id++;
  return;
}

grid etour(grid &edg, ll rt){
  ll cn=edg.size(), id=1;
  grid tour={vi(2*cn, 0), vi(cn), vi(cn)};
  et(edg, rt, 0, tour, id);
  return tour;
}
\end{lstlisting}
\subsection{BFS}
\begin{lstlisting}

ll bfs(grid &edg, ll sn){
  ll cn=edg.size(), lv=-1, cl=0, nl=1, at, ls;
  vi vst(cn+1, 0), prt(cn+1, -1);
  queue<ll> call;
  call.push(sn); vst[sn]++;
  while(!call.empty()){
    if(!cl){
      lv++; cl=nl; nl=0;
    }

    at=call.front();
    //if(at==en) return lv;
    call.pop(); cl--; ls=at;
    for(ll to:edg[at]){
      if(!vst[to]){

        prt[to]=at;
        call.push(to);
        vst[to]++;
        nl++;
      }
    }
  }
  return 0;
  //return ls; //for deepest.
}
\end{lstlisting}
\section{String}
\subsection{Hashing}
\begin{lstlisting}
class HashedString {
private:
  static const long long M = 1e9 + 7;
  static const long long B = 256;
  static vector<long long> pow;
  vector<long long> p_hash;

public:
  HashedString(const string& s) : p_hash(s.size() + 1) {
    while (pow.size() < s.size()) {
      pow.push_back((pow.back() * B) % M);
    }
    p_hash[0] = 0;
    for (int i = 0; i < s.size(); i++) {
      p_hash[i + 1] = ((p_hash[i] * B) % M + s[i]) % M;
    }
  }

  long long getHash(int start, int end) {
    long long raw_val = (
      p_hash[end + 1] - (p_hash[start] * pow[end - start + 1])
    );
    return (raw_val % M + M) % M;
  }
};

vector<long long> HashedString::pow = {1};
\end{lstlisting}
\subsection{Double hash}
\begin{lstlisting}
// define +, -, * for (PLL, LL) and (PLL, PLL), % for (PLL, PLL);
PLL base(1949313259, 1997293877);
PLL mod(2091573227, 2117566807);

PLL power(PLL a, LL p) {
  PLL ans = PLL(1, 1);
  for(; p; p >>= 1, a = a * a % mod) {
      if(p & 1) ans = ans * a % mod;
  }
  return ans;
}

PLL inverse(PLL a) { return power(a, (mod.ff - 1) * (mod.ss - 1) - 1); }
PLL inv_base = inverse(base);
PLL val;
vector<PLL> P;

void hash_init(int n) {
  P.resize(n + 1);
  P[0] = PLL(1, 1);
  for (int i = 1; i <= n; i++) P[i] = (P[i - 1] * base) % mod;
}
PLL append(PLL cur, char c) { return (cur * base + c) % mod; }
/// prepends c to string with size k
PLL prepend(PLL cur, int k, char c) { return (P[k] * c + cur) % mod; }
/// replaces the i-th (0-indexed) character from right from a to b;
PLL replace(PLL cur, int i, char a, char b) {
  cur = (cur + P[i] * (b - a)) % mod;
  return (cur + mod) % mod;
}
/// Erases c from the back of the string
PLL pop_back(PLL hash, char c) {
  return (((hash - c) * inv_base) % mod + mod) % mod;
}
/// Erases c from front of the string with size len
PLL pop_front(PLL hash, int len, char c) {
  return ((hash - P[len - 1] * c) % mod + mod) % mod;
}
/// concatenates two strings where length of the right is k
PLL concat(PLL left, PLL right, int k) { return (left * P[k] + right) % mod; }
/// Calculates hash of string with size len repeated cnt times
/// This is O(log n). For O(1), pre-calculate inverses
PLL repeat(PLL hash, int len, LL cnt) {
  PLL mul = (P[len * cnt] - 1) * inverse(P[len] - 1);
  mul = (mul % mod + mod) % mod;
  PLL ret = (hash * mul) % mod;
  if (P[len].ff == 1) ret.ff = hash.ff * cnt;
  if (P[len].ss == 1) ret.ss = hash.ss * cnt;
  return ret;
}
LL get(PLL hash) { return ((hash.ff << 32) ^ hash.ss); }
struct hashlist {
  int len;
  vector<PLL> H, R;
  hashlist() {}
  hashlist(string& s) {
    len = (int)s.size();
    hash_init(len);
    H.resize(len + 1, PLL(0, 0)), R.resize(len + 2, PLL(0, 0));
    for (int i = 1; i <= len; i++) H[i] = append(H[i - 1], s[i - 1]);
    for (int i = len; i >= 1; i--) R[i] = append(R[i + 1], s[i - 1]);
  }
  /// 1-indexed
  PLL range_hash(int l, int r) {
    return ((H[r] - H[l - 1] * P[r - l + 1]) % mod + mod) % mod;
  }
  PLL reverse_hash(int l, int r) {
    return ((R[l] - R[r + 1] * P[r - l + 1]) % mod + mod) % mod;
  }
  PLL concat_range_hash(int l1, int r1, int l2, int r2) {
    return concat(range_hash(l1, r1), range_hash(l2, r2), r2 - l2 + 1);
  }
  PLL concat_reverse_hash(int l1, int r1, int l2, int r2) {
    return concat(reverse_hash(l2, r2), reverse_hash(l1, r1), r1 - l1 + 1);
  }
};
\end{lstlisting}
\subsection{Aho Corasick}
\begin{lstlisting}
struct AC {
int N, P;
const int A = 26;
vector<vector<int>> next;
vector<int> link, out_link;
vector<vector<int>> out;
AC() : N(0), P(0) { node(); }
int node() {
  next.emplace_back(A, 0);
  link.emplace_back(0);
  out_link.emplace_back(0);
  out.emplace_back(0);
  return N++;
}
inline int get(char c) { return c - 'a'; }
int add_pattern(const string T) {
  int u = 0;
  for (auto c : T) {
    if (!next[u][get(c)]) next[u][get(c)] = node();
    u = next[u][get(c)];
  }
  out[u].push_back(P);
  return P++;
}
void compute() {
  queue<int> q;
  for (q.push(0); !q.empty();) {
    int u = q.front(); q.pop();
    for (int c = 0; c < A; ++c) {
      int v = next[u][c];
      if (!v) next[u][c] = next[link[u]][c];
      else {
        link[v] = u ? next[link[u]][c] : 0;
        out_link[v] = out[link[v]].empty() ? out_link[link[v]] : link[v];
        q.push(v);
      }
    }
  }
}
int advance(int u, char c) {
  while (u && !next[u][get(c)]) u = link[u];
  u = next[u][get(c)];
  return u;
}
void match(const string S) {
  int u = 0;
  for (auto c : S) {
    u = advance(u, c);
    for (int v = u; v; v = out_link[v]) {
      for (auto p : out[v]) cout << "match " << p << endl;
    }
  }
}
};
int main() {
  AC aho; int n; cin >> n;
  while (n--) {
    string s; cin >> s;
    aho.add_pattern(s);
  }
  aho.compute(); string text;
  cin >> text; aho.match(text);
  return 0;
}
\end{lstlisting}
\subsection{KMP}
\begin{lstlisting}
vector<int> prefix_function(string s) {
  int n = (int)s.length();
  vector<int> pi(n);
  
  for (int i = 1; i < n; i++) {
    int j = pi[i - 1];
    while (j > 0 && s[i] != s[j])
      j = pi[j - 1];
    if (s[i] == s[j])
      j++;
    pi[i] = j;
  }
  
  return pi;
}
\end{lstlisting}
\subsection{Manacher's}
\begin{lstlisting}
vector<int> d1(n);
// d[i] = number of palindromes taking s[i] as center
for (int i = 0, l = 0, r = -1; i < n; i++) {
  int k = (i > r) ? 1 : min(d1[l + r - i], r - i + 1);
  while (0 <= i - k && i + k < n && s[i - k] == s[i + k]) k++;
  d1[i] = k--;
  if (i + k > r) l = i - k, r = i + k;
}
vector<int> d2(n);
// d[i] = number of palindromes taking s[i-1] and s[i] as center
for (int i = 0, l = 0, r = -1; i < n; i++) {
  int k = (i > r) ? 0 : min(d2[l + r - i + 1], r - i + 1);
  while (0 <= i - k - 1 && i + k < n && s[i - k - 1] == s[i + k]) k++;
  d2[i] = k--;
  if (i + k > r) l = i - k - 1, r = i + k;
}
\end{lstlisting}
\subsection{Suffix Match FFT}
\begin{lstlisting}
// Find occurrences of t in s where '?'s are automatically matched with any character
// res[i + m - 1] = sum_j=0 to m - 1 { s[i + j] * t[j] * (s[i + j] - t[j]) }
vector<int> string_matching(string &s, string &t) {
  int n = s.size(), m = t.size();
  vector<int> s1(n), s2(n), s3(n);
  for(int i = 0; i < n; i++)
    s1[i] = s[i] == '?' ? 0 : s[i] - 'a' + 1; // assign any non zero number for non '?'s
  for(int i = 0; i < n; i++)
    s2[i] = s1[i] * s1[i];
  for(int i = 0; i < n; i++)
    s3[i] = s1[i] * s2[i];
  
  vector<int> t1(m), t2(m), t3(m);
  for(int i = 0; i < m; i++)
    t1[i] = t[i] == '?' ? 0 : t[i] - 'a' + 1;
  for(int i = 0; i < m; i++)
    t2[i] = t1[i] * t1[i];
  for(int i = 0; i < m; i++)
    t3[i] = t1[i] * t2[i];
  
  reverse(t1.begin(), t1.end());
  reverse(t2.begin(), t2.end());
  reverse(t3.begin(), t3.end());
  
  vector<int> s1t3 = multiply(s1, t3);
  vector<int> s2t2 = multiply(s2, t2);
  vector<int> s3t1 = multiply(s3, t1);
  
  vector<int> res(n);
  for(int i = 0; i < n; i++)
    res[i] = s1t3[i] - s2t2[i] * 2 + s3t1[i];
  
  vector<int> oc;
  for(int i = m - 1; i < n; i++)
    if(res[i] == 0)
      oc.push_back(i - m + 1);
  
  return oc;
}
\end{lstlisting}
\subsection{Suffix Array}
\begin{lstlisting}
vector<VI> c;
VI sort_cyclic_shifts(const string &s) {
  int n = s.size();
  const int alphabet = 256;
  VI p(n), cnt(alphabet, 0);

  c.clear();
  c.emplace_back();
  c[0].resize(n);

  for (int i = 0; i < n; i++) cnt[s[i]]++;
  for (int i = 1; i < alphabet; i++) cnt[i] += cnt[i - 1];
  for (int i = 0; i < n; i++) p[--cnt[s[i]]] = i;

  c[0][p[0]] = 0;
  int classes = 1;

  for (int i = 1; i < n; i++) {
    if (s[p[i]] != s[p[i - 1]]) classes++;
    c[0][p[i]] = classes - 1;
  }

  VI pn(n), cn(n);
  cnt.resize(n);
  for (int h = 0; (1 << h) < n; h++) {
    for (int i = 0; i < n; i++) {
      pn[i] = p[i] - (1 << h);
      if (pn[i] < 0) pn[i] += n;
    }
    fill(cnt.begin(), cnt.end(), 0);
    /// radix sort
    for (int i = 0; i < n; i++) cnt[c[h][pn[i]]]++;
    for (int i = 1; i < classes; i++) cnt[i] += cnt[i - 1];
    for (int i = n - 1; i >= 0; i--) p[--cnt[c[h][pn[i]]]] = pn[i];

    cn[p[0]] = 0;
    classes = 1;

    for (int i = 1; i < n; i++) {
      PII cur = {c[h][p[i]], c[h][(p[i] + (1 << h)) % n]};
      PII prev = {c[h][p[i - 1]], c[h][(p[i - 1] + (1 << h)) % n]};
      if (cur != prev) ++classes;
      cn[p[i]] = classes - 1;
    }
    c.push_back(cn);
  }
  return p;
}
VI suffix_array_construction(string s) {
  s += "!";
  VI sorted_shifts = sort_cyclic_shifts(s);
  sorted_shifts.erase(sorted_shifts.begin());
  return sorted_shifts;
}
/// LCP between the ith and jth (i != j) suffix of the STRING
int suffixLCP(int i, int j) {
  assert(i != j);
  int log_n = c.size() - 1;

  int ans = 0;
  for (int k = log_n; k >= 0; k--) {
    if (c[k][i] == c[k][j]) {
      ans += 1 << k;
      i += 1 << k;
      j += 1 << k;
    }
  }
  return ans;
}

VI lcp_construction(const string &s, const VI &sa) {
  int n = s.size();
  VI rank(n, 0);
  VI lcp(n - 1, 0);

  for (int i = 0; i < n; i++) rank[sa[i]] = i;

  for (int i = 0, k = 0; i < n; i++, k -= (k != 0)) {
    if (rank[i] == n - 1) {
      k = 0;
      continue;
    }
    int j = sa[rank[i] + 1];
    while (i + k < n && j + k < n && s[i + k] == s[j + k]) k++;
    lcp[rank[i]] = k;
  }
  return lcp;
}
\end{lstlisting}
\subsection{Trie}
\begin{lstlisting}
template<int sz>
struct Trie {
  Trie() : id(1) {
    memset(endMark, 0, sizeof endMark);
    for_each(all(trie), [](vector<int> &v) { v.assign(sz, 0); });
  }

  void insert(const string &s) {
    int cur = 0;
    for (auto c : s) {
      int val = c - 'a';
      if (!trie[cur][val])
        trie[cur][val] = id++;
      cur = trie[cur][val];
    }
    endMark[cur] = true;
  }

  bool search(const string &s) {
    int cur = 0;
    for (auto c : s) {
      int val = c - 'a';
      if (!trie[cur][val])
        return false;
      cur = trie[cur][val];
    }
    return endMark[cur];
  }

private:
  int id, endMark[100005];
  vector<int> trie[100005];
};
\end{lstlisting}
\subsection{Z Algo}
\begin{lstlisting}
vector<int> calcz(string s) {
  int n = s.size();
  vector<int> z(n);
  int l = 0, r = 0;
  for (int i = 1; i < n; i++) {
    if (i > r) {
      l = r = i;
      while (r < n && s[r] == s[r - l]) r++;
      z[i] = r - l, r--;
    } else {
      int k = i - l;
      if (z[k] < r - i + 1) z[i] = z[k];
      else {
        l = i;
        while (r < n && s[r] == s[r - l]) r++;
        z[i] = r - l, r--;
      }
    }
  }
  return z;
}
\end{lstlisting}
\section{DP}
\subsection{Bitmask}
\begin{lstlisting}
for(int mask= 0; mask < (1 << 4); mask++){
  ll sum_of_set = 0;
  for(int i = 0; (1ll << i) <= mask; i++) if(mask& (1ll << i)) sum_of_set += v[i];
  if(sum_of_set == S){
    cout << "Yes\n";
    flg = true;
    break;
  }
}
if(!flg) cout << "No\n";\end{lstlisting}
\subsection{LIS}
\begin{lstlisting}
vector<pair<ll, ll>> LIS(vector<ll> &v){
  ll n=v.size();
  vector<pair<ll, ll>> seq(n); //{size, last element}
  set<ll> s; //multiset for non_dcrs
  for(int i=0; i<n; ++i){
    auto it=s.lower_bound(v[i]);
    if(it==s.end()) s.insert(v[i]);
    else{
      s.erase(it);
      s.insert(v[i]);
    } 
    seq[i]={s.size(), *(s.rbegin())};
  }
  return seq;
} //seq[i] = {size of LIS in v[0, i], largest element in that sequence}\end{lstlisting}
\subsection{Divide and Conquer DP}
\begin{lstlisting}
const int K = 805, N = 4005;
LL dp[2][N], _cost[N][N];
// 1-indexed for convenience
LL cost(int l, int r) {
  return _cost[r][r] - _cost[l - 1][r] - _cost[r][l - 1] + _cost[l - 1][l - 1] >> 1;
}
void compute(int cnt, int l, int r, int optl, int optr) {
  if (l > r) return;
  int mid = l + r >> 1;
  LL best = INT_MAX;
  int opt = -1;
  for (int i = optl; i <= min(mid, optr); i++) {
    LL cur = dp[cnt ^ 1][i - 1] + cost(i, mid);
    if (cur < best) best = cur, opt = i;
  }
  dp[cnt][mid] = best;
  compute(cnt, l, mid - 1, optl, opt);
  compute(cnt, mid + 1, r, opt, optr);
}
LL dnc_dp(int k, int n) {
  fill(dp[0] + 1, dp[0] + n + 1, INT_MAX);
  for (int cnt = 1; cnt <= k; cnt++) {
    compute(cnt & 1, 1, n, 1, n);
  }
  return dp[k & 1][n];
}
\end{lstlisting}
\subsection{Knuth Optimization}
\begin{lstlisting}
const int N = 1005;
LL dp[N][N], a[N];
int opt[N][N];
LL cost(int i, int j) { return a[j + 1] - a[i]; }
LL knuth_optimization(int n) {
  for (int i = 0; i < n; i++) {
    dp[i][i] = 0;
    opt[i][i] = i;
  }
  for (int i = n - 2; i >= 0; i--) {
    for (int j = i + 1; j < n; j++) {
      LL mn = LLONG_MAX;
      LL c = cost(i, j);
      for (int k = opt[i][j - 1]; k <= min(j - 1, opt[i + 1][j]); k++) {
        if (mn > dp[i][k] + dp[k + 1][j] + c) {
          mn = dp[i][k] + dp[k + 1][j] + c;
          opt[i][j] = k;
        }
      }
      dp[i][j] = mn;
    }
  }
  return dp[0][n - 1];
}
\end{lstlisting}
\section{Math}
\subsection{BigMod}
\begin{lstlisting}
ll bigmod(ll a, ll b, ll m) {
  if(b  == 0) return 1;
  ll x = bigmod(a, b/2, m);
  x = (x * x) % m;
  if(b % 2) x = (x * a) % m;
  return x;
}\end{lstlisting}
\subsection{Combi}
\begin{lstlisting}
array<int, N + 1> fact, inv, inv_fact;
void init() {
  fact[0] = inv_fact[0] = 1;
  for (int i = 1; i <= N; i++) {
    inv[i] = i == 1 ? 1 : (LL)inv[i - mod % i] * (mod / i + 1) % mod;
    fact[i] = (LL)fact[i - 1] * i % mod;
    inv_fact[i] = (LL)inv_fact[i - 1] * inv[i] % mod;
  }
}
LL C(int n, int r) {
  return (r < 0 or r > n) ? 0 : (LL)fact[n] * inv_fact[r] % mod * inv_fact[n - r] % mod;
}
\end{lstlisting}
\subsection{Sieve}
\begin{lstlisting}
const ll m = 10e6;
vector<ll> lp(m+1);
vector<ll> prime;
void ln_sieve() {
  for(ll i = 2; i <= m; i++){
    if(!lp[i]){
      lp[i] = i;
      prime.push_back(i);
    }

    for(ll j = 0; i * prime[j] <= m; j++){
      lp[i * prime[j]] = prime[j];
      if(prime[j] == lp[i]) break;
    }
  }
}\end{lstlisting}
\subsection{Linear Sieve}
\begin{lstlisting}
const int N = 1e7;
vector<int> primes;
int spf[N + 5], phi[N + 5], NOD[N + 5], cnt[N + 5], POW[N + 5];
bool prime[N + 5];
int SOD[N + 5];
void init() {
  fill(prime + 2, prime + N + 1, 1);
  SOD[1] = NOD[1] = phi[1] = spf[1] = 1;
  for (LL i = 2; i <= N; i++) {
    if (prime[i]) {
      primes.push_back(i), spf[i] = i;
      phi[i] = i - 1;
      NOD[i] = 2, cnt[i] = 1;
      SOD[i] = i + 1, POW[i] = i;
    }
    for (auto p : primes) {
      if (p * i > N or p > spf[i]) break;
      prime[p * i] = false, spf[p * i] = p;
      if (i % p == 0) {
        phi[p * i] = p * phi[i];
        NOD[p * i] = NOD[i] / (cnt[i] + 1) * (cnt[i] + 2),
                cnt[p * i] = cnt[i] + 1;
        SOD[p * i] = SOD[i] / SOD[POW[i]] * (SOD[POW[i]] + p * POW[i]),
                POW[p * i] = p * POW[i];
        break;
      } else {
        phi[p * i] = phi[p] * phi[i];
        NOD[p * i] = NOD[p] * NOD[i], cnt[p * i] = 1;
        SOD[p * i] = SOD[p] * SOD[i], POW[p * i] = p;
      }
    }
  }
}

\end{lstlisting}
\subsection{Pollard Rho}
\begin{lstlisting}
LL mul(LL a, LL b, LL mod) {
    return (__int128)a * b % mod;
    // LL ans = a * b - mod * (LL) (1.L / mod * a * b);
    // return ans + mod * (ans < 0) - mod * (ans >= (LL) mod);
}
LL bigmod(LL num, LL pow, LL mod) {
    LL ans = 1;
    for (; pow > 0; pow >>= 1, num = mul(num, num, mod))
        if (pow & 1) ans = mul(ans, num, mod);
    return ans;
}
bool is_prime(LL n) {
    if (n < 2 or n % 6 % 4 != 1) return (n | 1) == 3;
    LL a[] = {2, 325, 9375, 28178, 450775, 9780504, 1795265022};
    LL s = __builtin_ctzll(n - 1), d = n >> s;
    for (LL x : a) {
        LL p = bigmod(x % n, d, n), i = s;
        for (; p != 1 and p != n - 1 and x % n and i--; p = mul(p, p, n))
            ;
        if (p != n - 1 and i != s) return false;
    }
    return true;
}
LL get_factor(LL n) {
    auto f = [&](LL x) { return mul(x, x, n) + 1; };
    LL x = 0, y = 0, t = 0, prod = 2, i = 2, q;
    for (; t++ % 40 or gcd(prod, n) == 1; x = f(x), y = f(f(y))) {
        (x == y) ? x = i++, y = f(x) : 0;
        prod = (q = mul(prod, max(x, y) - min(x, y), n)) ? q : prod;
    }
    return gcd(prod, n);
}
map<LL, int> factorize(LL n) {
    map<LL, int> res;
    if (n < 2) return res;
    LL small_primes[] = {2,  3,  5,  7,  11, 13, 17, 19, 23, 29, 31, 37, 41,
                                              43, 47, 53, 59, 61, 67, 71, 73, 79, 83, 89, 97};
    for (LL p : small_primes)
        for (; n % p == 0; n /= p, res[p]++)
            ;

    auto _factor = [&](LL n, auto &_factor) {
        if (n == 1) return;
        if (is_prime(n))
            res[n]++;
        else {
            LL x = get_factor(n);
            _factor(x, _factor);
            _factor(n / x, _factor);
        }
    };
    _factor(n, _factor);
    return res;
}
\end{lstlisting}
\subsection{Chinese Remainder Theorem}
\begin{lstlisting}
// given a, b will find solutions for
// ax + by = 1
tuple<LL, LL, LL> EGCD(LL a, LL b) {
  if (b == 0)
    return {1, 0, a};
  else {
    auto [x, y, g] = EGCD(b, a % b);
    return {y, x - a / b * y, g};
  }
}
// given modulo equations, will apply CRT
PLL CRT(vector<PLL> &v) {
  LL V = 0, M = 1;
  for (auto &[v, m] : v) {  // value % mod
    auto [x, y, g] = EGCD(M, m);
    if ((v - V) % g != 0) return {-1, 0};
    V += x * (v - V) / g % (m / g) * M, M *= m / g;
    V = (V % M + M) % M;
  }
  return make_pair(V, M);
}
\end{lstlisting}
\subsection{Mobius Function}
\begin{lstlisting}
const int N = 1e6 + 5;
int mob[N];
void mobius() {
  memset(mob, -1, sizeof mob);
  mob[1] = 1;
  for (int i = 2; i < N; i++)
    if (mob[i]) {
      for (int j = i + i; j < N; j += i) mob[j] -= mob[i];
    }
}

\end{lstlisting}
\subsection{FFT}
\begin{lstlisting}
using CD = complex<double>;
typedef long long LL;
const double PI = acos(-1.0L);

int N;
vector<int> perm;
vector<CD> wp[2];
void precalculate(int n) {
  assert((n & (n - 1)) == 0), N = n;
  perm = vector<int>(N, 0);
  for (int k = 1; k < N; k <<= 1) {
    for (int i = 0; i < k; i++) {
      perm[i] <<= 1;
      perm[i + k] = 1 + perm[i];
    }
  }
  wp[0] = wp[1] = vector<CD>(N);
  for (int i = 0; i < N; i++) {
    wp[0][i] = CD(cos(2 * PI * i / N), sin(2 * PI * i / N));
    wp[1][i] = CD(cos(2 * PI * i / N), -sin(2 * PI * i / N));
  }
}
void fft(vector<CD> &v, bool invert = false) {
  if (v.size() != perm.size()) precalculate(v.size());
  for (int i = 0; i < N; i++)
    if (i < perm[i]) swap(v[i], v[perm[i]]);
  for (int len = 2; len <= N; len *= 2) {
    for (int i = 0, d = N / len; i < N; i += len) {
      for (int j = 0, idx = 0; j < len / 2; j++, idx += d) {
        CD x = v[i + j];
        CD y = wp[invert][idx] * v[i + j + len / 2];
        v[i + j] = x + y;
        v[i + j + len / 2] = x - y;
      }
    }
  }
  if (invert) {
    for (int i = 0; i < N; i++) v[i] /= N;
  }
}
void pairfft(vector<CD> &a, vector<CD> &b, bool invert = false) {
  int N = a.size();
  vector<CD> p(N);
  for (int i = 0; i < N; i++) p[i] = a[i] + b[i] * CD(0, 1);
  fft(p, invert);
  p.push_back(p[0]);
  for (int i = 0; i < N; i++) {
    if (invert) {
      a[i] = CD(p[i].real(), 0);
      b[i] = CD(p[i].imag(), 0);
    } else {
      a[i] = (p[i] + conj(p[N - i])) * CD(0.5, 0);
      b[i] = (p[i] - conj(p[N - i])) * CD(0, -0.5);
    }
  }
}
vector<LL> multiply(const vector<LL> &a, const vector<LL> &b) {
  int n = 1;
  while (n < a.size() + b.size()) n <<= 1;
  vector<CD> fa(a.begin(), a.end()), fb(b.begin(), b.end());
  fa.resize(n);
  fb.resize(n);
  //        fft(fa); fft(fb);
  pairfft(fa, fb);
  for (int i = 0; i < n; i++) fa[i] = fa[i] * fb[i];
  fft(fa, true);
  vector<LL> ans(n);
  for (int i = 0; i < n; i++) ans[i] = round(fa[i].real());
  return ans;
}
const int M = 1e9 + 7, B = sqrt(M) + 1;
vector<LL> anyMod(const vector<LL> &a, const vector<LL> &b) {
  int n = 1;
  while (n < a.size() + b.size()) n <<= 1;
  vector<CD> al(n), ar(n), bl(n), br(n);
  for (int i = 0; i < a.size(); i++) al[i] = a[i] % M / B, ar[i] = a[i] % M % B;
  for (int i = 0; i < b.size(); i++) bl[i] = b[i] % M / B, br[i] = b[i] % M % B;
  pairfft(al, ar);
  pairfft(bl, br);
  //        fft(al); fft(ar); fft(bl); fft(br);
  for (int i = 0; i < n; i++) {
    CD ll = (al[i] * bl[i]), lr = (al[i] * br[i]);
    CD rl = (ar[i] * bl[i]), rr = (ar[i] * br[i]);
    al[i] = ll;
    ar[i] = lr;
    bl[i] = rl;
    br[i] = rr;
  }
  pairfft(al, ar, true);
  pairfft(bl, br, true);
  //        fft(al, true); fft(ar, true); fft(bl, true); fft(br, true);
  vector<LL> ans(n);
  for (int i = 0; i < n; i++) {
    LL right = round(br[i].real()), left = round(al[i].real());
    ;
    LL mid = round(round(bl[i].real()) + round(ar[i].real()));
    ans[i] = ((left % M) * B * B + (mid % M) * B + right) % M;
  }
  return ans;
}
\end{lstlisting}
\subsection{NTT}
\begin{lstlisting}
const LL N = 1 << 18;
const LL MOD = 786433;

vector<LL> P[N];
LL rev[N], w[N | 1], a[N], b[N], inv_n, g;
LL Pow(LL b, LL p) {
  LL ret = 1;
  while (p) {
    if (p & 1) ret = (ret * b) % MOD;
    b = (b * b) % MOD;
    p >>= 1;
  }
  return ret;
}
LL primitive_root(LL p) {
  vector<LL> factor;
  LL phi = p - 1, n = phi;
  for (LL i = 2; i * i <= n; i++) {
    if (n % i) continue;
    factor.emplace_back(i);
    while (n % i == 0) n /= i;
  }
  if (n > 1) factor.emplace_back(n);
  for (LL res = 2; res <= p; res++) {
    bool ok = true;
    for (LL i = 0; i < factor.size() && ok; i++)
      ok &= Pow(res, phi / factor[i]) != 1;
    if (ok) return res;
  }
  return -1;
}
void prepare(LL n) {
  LL sz = abs(31 - __builtin_clz(n));
  LL r = Pow(g, (MOD - 1) / n);
  inv_n = Pow(n, MOD - 2);
  w[0] = w[n] = 1;
  for (LL i = 1; i < n; i++) w[i] = (w[i - 1] * r) % MOD;
  for (LL i = 1; i < n; i++)
    rev[i] = (rev[i >> 1] >> 1) | ((i & 1) << (sz - 1));
}
void NTT(LL *a, LL n, LL dir = 0) {
  for (LL i = 1; i < n - 1; i++)
    if (i < rev[i]) swap(a[i], a[rev[i]]);
  for (LL m = 2; m <= n; m <<= 1) {
    for (LL i = 0; i < n; i += m) {
      for (LL j = 0; j < (m >> 1); j++) {
        LL &u = a[i + j], &v = a[i + j + (m >> 1)];
        LL t = v * w[dir ? n - n / m * j : n / m * j] % MOD;
        v = u - t < 0 ? u - t + MOD : u - t;
        u = u + t >= MOD ? u + t - MOD : u + t;
      }
    }
  }
  if (dir)
    for (LL i = 0; i < n; i++) a[i] = (inv_n * a[i]) % MOD;
}
vector<LL> mul(vector<LL> p, vector<LL> q) {
  LL n = p.size(), m = q.size();
  LL t = n + m - 1, sz = 1;
  while (sz < t) sz <<= 1;
  prepare(sz);

  for (LL i = 0; i < n; i++) a[i] = p[i];
  for (LL i = 0; i < m; i++) b[i] = q[i];
  for (LL i = n; i < sz; i++) a[i] = 0;
  for (LL i = m; i < sz; i++) b[i] = 0;

  NTT(a, sz);
  NTT(b, sz);
  for (LL i = 0; i < sz; i++) a[i] = (a[i] * b[i]) % MOD;
  NTT(a, sz, 1);

  vector<LL> c(a, a + sz);
  while (c.size() && c.back() == 0) c.pop_back();
  return c;
}
\end{lstlisting}
\subsection{ModInverse}
\begin{lstlisting}

//solves ax+by=gcd(a, b) i guess
int gcd(int a, int b, int& x, int& y) {
  x = 1, y = 0;
  int x1 = 0, y1 = 1, a1 = a, b1 = b;
  while (b1) {
    int q = a1 / b1;
    tie(x, x1) = make_tuple(x1, x - q * x1);
    tie(y, y1) = make_tuple(y1, y - q * y1);
    tie(a1, b1) = make_tuple(b1, a1 - q * b1);
  }
  return a1;
}

//finds mod inverse?
int x, y;
int g = gcd(a, m, x, y);
if (g != 1) {
  cout << "No solution!";
}
else {
  x = (x % m + m) % m;
  cout << x << endl;
}
\end{lstlisting}
\subsection{Diophantine}
\begin{lstlisting}
int gcd(int a, int b, int& x, int& y) {
  if (b == 0) {
    x = 1;
    y = 0;
    return a;
  }
  int x1, y1;
  int d = gcd(b, a % b, x1, y1);
  x = y1;
  y = x1 - y1 * (a / b);
  return d;
}

bool find_any_solution(int a, int b, int c, int &x0, int &y0, int &g) {
  g = gcd(abs(a), abs(b), x0, y0);
  if (c % g) {
    return false;
  }

  x0 *= c / g;
  y0 *= c / g;
  if (a < 0) x0 = -x0;
  if (b < 0) y0 = -y0;
  return true;
}

void shift_solution(int & x, int & y, int a, int b, int cnt) {
  x += cnt * b;
  y -= cnt * a;
}

int find_all_solutions(int a, int b, int c, int minx, int maxx, int miny, int maxy) {
  int x, y, g;
  if (!find_any_solution(a, b, c, x, y, g))
    return 0;
  a /= g;
  b /= g;

  int sign_a = a > 0 ? +1 : -1;
  int sign_b = b > 0 ? +1 : -1;

  shift_solution(x, y, a, b, (minx - x) / b);
  if (x < minx)
    shift_solution(x, y, a, b, sign_b);
  if (x > maxx)
    return 0;
  int lx1 = x;

  shift_solution(x, y, a, b, (maxx - x) / b);
  if (x > maxx)
    shift_solution(x, y, a, b, -sign_b);
  int rx1 = x;

  shift_solution(x, y, a, b, -(miny - y) / a);
  if (y < miny)
    shift_solution(x, y, a, b, -sign_a);
  if (y > maxy)
    return 0;
  int lx2 = x;

  shift_solution(x, y, a, b, -(maxy - y) / a);
  if (y > maxy)
    shift_solution(x, y, a, b, sign_a);
  int rx2 = x;

  if (lx2 > rx2)
    swap(lx2, rx2);
  int lx = max(lx1, lx2);
  int rx = min(rx1, rx2);

  if (lx > rx)
    return 0;
  return (rx - lx) / abs(b) + 1;
}\end{lstlisting}
\section{Geometry}
\subsection{Point}
\begin{lstlisting}
typedef double Tf;
typedef double Ti;  /// use long long for exactness
const Tf PI = acos(-1), EPS = 1e-9;
int dcmp(Tf x) { return abs(x) < EPS ? 0 : (x < 0 ? -1 : 1); }

struct Point {
    Ti x, y;
    Point(Ti x = 0, Ti y = 0) : x(x), y(y) {}

    Point operator+(const Point& u) const { return Point(x + u.x, y + u.y); }
    Point operator-(const Point& u) const { return Point(x - u.x, y - u.y); }
    Point operator*(const LL u) const { return Point(x * u, y * u); }
    Point operator*(const Tf u) const { return Point(x * u, y * u); }
    Point operator/(const Tf u) const { return Point(x / u, y / u); }

    bool operator==(const Point& u) const {
        return dcmp(x - u.x) == 0 && dcmp(y - u.y) == 0;
    }
    bool operator!=(const Point& u) const { return !(*this == u); }
    bool operator<(const Point& u) const {
        return dcmp(x - u.x) < 0 || (dcmp(x - u.x) == 0 && dcmp(y - u.y) < 0);
    }
};
Ti dot(Point a, Point b) { return a.x * b.x + a.y * b.y; }
Ti cross(Point a, Point b) { return a.x * b.y - a.y * b.x; }
Tf length(Point a) { return sqrt(dot(a, a)); }
Ti sqLength(Point a) { return dot(a, a); }
Tf distance(Point a, Point b) { return length(a - b); }
Tf angle(Point u) { return atan2(u.y, u.x); }

// returns angle between oa, ob in (-PI, PI]
Tf angleBetween(Point a, Point b) {
    Tf ans = angle(b) - angle(a);
    return ans <= -PI ? ans + 2 * PI : (ans > PI ? ans - 2 * PI : ans);
}
// Rotate a ccw by rad radians, Tf Ti same
Point rotate(Point a, Tf rad) {
    return Point(a.x * cos(rad) - a.y * sin(rad),
                              a.x * sin(rad) + a.y * cos(rad));
}
// rotate a ccw by angle th with cos(th) = co && sin(th) = si, tf ti same
Point rotatePrecise(Point a, Tf co, Tf si) {
    return Point(a.x * co - a.y * si, a.y * co + a.x * si);
}
Point rotate90(Point a) { return Point(-a.y, a.x); }
// scales vector a by s such that length of a becomes s, Tf Ti same
Point scale(Point a, Tf s) { return a / length(a) * s; }
// returns an unit vector perpendicular to vector a, Tf Ti same
Point normal(Point a) {
    Tf l = length(a);
    return Point(-a.y / l, a.x / l);
}
// returns 1 if c is left of ab, 0 if on ab && -1 if right of ab
int orient(Point a, Point b, Point c) { return dcmp(cross(b - a, c - a)); }
/// Use as sort(v.begin(), v.end(), polarComp(O, dir))
/// Polar comparator around O starting at direction dir
struct polarComp {
    Point O, dir;
    polarComp(Point O = Point(0, 0), Point dir = Point(1, 0)) : O(O), dir(dir) {}
    bool half(Point p) {
        return dcmp(cross(dir, p)) < 0 ||
                      (dcmp(cross(dir, p)) == 0 && dcmp(dot(dir, p)) > 0);
    }
    bool operator()(Point p, Point q) {
        return make_tuple(half(p), 0) < make_tuple(half(q), cross(p, q));
    }
};
struct Segment {
    Point a, b;
    Segment(Point aa, Point bb) : a(aa), b(bb) {}
};
typedef Segment Line;
struct Circle {
    Point o;
    Tf r;
    Circle(Point o = Point(0, 0), Tf r = 0) : o(o), r(r) {}
    // returns true if point p is in || on the circle
    bool contains(Point p) { return dcmp(sqLength(p - o) - r * r) <= 0; }
    // returns a point on the circle rad radians away from +X CCW
    Point point(Tf rad) {
        static_assert(is_same<Tf, Ti>::value);
        return Point(o.x + cos(rad) * r, o.y + sin(rad) * r);
    }
    // area of a circular sector with central angle rad
    Tf area(Tf rad = PI + PI) { return rad * r * r / 2; }
    // area of the circular sector cut by a chord with central angle alpha
    Tf sector(Tf alpha) { return r * r * 0.5 * (alpha - sin(alpha)); }
};
\end{lstlisting}
\subsection{Linear}
\begin{lstlisting}
// **** LINE LINE INTERSECTION START ****
// returns true if point p is on segment s
bool onSegment(Point p, Segment s) {
  return dcmp(cross(s.a - p, s.b - p)) == 0 && dcmp(dot(s.a - p, s.b - p)) <= 0;
}
// returns true if segment p && q touch or intersect
bool segmentsIntersect(Segment p, Segment q) {
  if (onSegment(p.a, q) || onSegment(p.b, q)) return true;
  if (onSegment(q.a, p) || onSegment(q.b, p)) return true;

  Ti c1 = cross(p.b - p.a, q.a - p.a);
  Ti c2 = cross(p.b - p.a, q.b - p.a);
  Ti c3 = cross(q.b - q.a, p.a - q.a);
  Ti c4 = cross(q.b - q.a, p.b - q.a);
  return dcmp(c1) * dcmp(c2) < 0 && dcmp(c3) * dcmp(c4) < 0;
}
bool linesParallel(Line p, Line q) {
  return dcmp(cross(p.b - p.a, q.b - q.a)) == 0;
}
// lines are represented as a ray from a point: (point, vector)
// returns false if two lines (p, v) && (q, w) are parallel or collinear
// true otherwise, intersection point is stored at o via reference, Tf Ti Same
bool lineLineIntersection(Point p, Point v, Point q, Point w, Point& o) {
  if (dcmp(cross(v, w)) == 0) return false;
  Point u = p - q;
  o = p + v * (cross(w, u) / cross(v, w));
  return true;
}
// returns false if two lines p && q are parallel or collinear
// true otherwise, intersection point is stored at o via reference
bool lineLineIntersection(Line p, Line q, Point& o) {
  return lineLineIntersection(p.a, p.b - p.a, q.a, q.b - q.a, o);
}
// returns the distance from point a to line l
// **** LINE LINE INTERSECTION FINISH ****
Tf distancePointLine(Point p, Line l) {
  return abs(cross(l.b - l.a, p - l.a) / length(l.b - l.a));
}
// returns the shortest distance from point a to segment s
Tf distancePointSegment(Point p, Segment s) {
  if (s.a == s.b) return length(p - s.a);
  Point v1 = s.b - s.a, v2 = p - s.a, v3 = p - s.b;
  if (dcmp(dot(v1, v2)) < 0)
    return length(v2);
  else if (dcmp(dot(v1, v3)) > 0)
    return length(v3);
  else
    return abs(cross(v1, v2) / length(v1));
}
// returns the shortest distance from segment p to segment q
Tf distanceSegmentSegment(Segment p, Segment q) {
  if (segmentsIntersect(p, q)) return 0;
  Tf ans = distancePointSegment(p.a, q);
  ans = min(ans, distancePointSegment(p.b, q));
  ans = min(ans, distancePointSegment(q.a, p));
  ans = min(ans, distancePointSegment(q.b, p));
  return ans;
}
// returns the projection of point p on line l, Tf Ti Same
Point projectPointLine(Point p, Line l) {
  Point v = l.b - l.a;
  return l.a + v * ((Tf)dot(v, p - l.a) / dot(v, v));
}
\end{lstlisting}
\subsection{Circular}
\begin{lstlisting}
// Extremely inaccurate for finding near touches
// compute intersection of line l with circle c
// The intersections are given in order of the ray (l.a, l.b), Tf Ti same
vector<Point> circleLineIntersection(Circle c, Line l) {
    vector<Point> ret;
    Point b = l.b - l.a, a = l.a - c.o;
    Tf A = dot(b, b), B = dot(a, b);
    Tf C = dot(a, a) - c.r * c.r, D = B * B - A * C;
    if (D < -EPS) return ret;
    ret.push_back(l.a + b * (-B - sqrt(D + EPS)) / A);
    if (D > EPS) ret.push_back(l.a + b * (-B + sqrt(D)) / A);
    return ret;
}
// signed area of intersection of circle(c.o, c.r) &&
// triangle(c.o, s.a, s.b) [cross(a-o, b-o)/2]
Tf circleTriangleIntersectionArea(Circle c, Segment s) {
    using Linear::distancePointSegment;
    Tf OA = length(c.o - s.a);
    Tf OB = length(c.o - s.b);
    // sector
    if (dcmp(distancePointSegment(c.o, s) - c.r) >= 0)
        return angleBetween(s.a - c.o, s.b - c.o) * (c.r * c.r) / 2.0;
    // triangle
    if (dcmp(OA - c.r) <= 0 && dcmp(OB - c.r) <= 0)
        return cross(c.o - s.b, s.a - s.b) / 2.0;
    // three part: (A, a) (a, b) (b, B)
    vector<Point> Sect = circleLineIntersection(c, s);
    return circleTriangleIntersectionArea(c, Segment(s.a, Sect[0])) +
                  circleTriangleIntersectionArea(c, Segment(Sect[0], Sect[1])) +
                  circleTriangleIntersectionArea(c, Segment(Sect[1], s.b));
}
// area of intersecion of circle(c.o, c.r) && simple polyson(p[])
Tf circlePolyIntersectionArea(Circle c, Polygon p) {
    Tf res = 0;
    int n = p.size();
    for (int i = 0; i < n; ++i)
        res += circleTriangleIntersectionArea(c, Segment(p[i], p[(i + 1) % n]));
    return abs(res);
}
// locates circle c2 relative to c1
// interior             (d < R - r)         ----> -2
// interior tangents (d = R - r)         ----> -1
// concentric        (d = 0)
// secants             (R - r < d < R + r) ---->  0
// exterior tangents (d = R + r)         ---->  1
// exterior             (d > R + r)         ---->  2
int circleCirclePosition(Circle c1, Circle c2) {
    Tf d = length(c1.o - c2.o);
    int in = dcmp(d - abs(c1.r - c2.r)), ex = dcmp(d - (c1.r + c2.r));
    return in < 0 ? -2 : in == 0 ? -1 : ex == 0 ? 1 : ex > 0 ? 2 : 0;
}
// compute the intersection points between two circles c1 && c2, Tf Ti same
vector<Point> circleCircleIntersection(Circle c1, Circle c2) {
    vector<Point> ret;
    Tf d = length(c1.o - c2.o);
    if (dcmp(d) == 0) return ret;
    if (dcmp(c1.r + c2.r - d) < 0) return ret;
    if (dcmp(abs(c1.r - c2.r) - d) > 0) return ret;

    Point v = c2.o - c1.o;
    Tf co = (c1.r * c1.r + sqLength(v) - c2.r * c2.r) / (2 * c1.r * length(v));
    Tf si = sqrt(abs(1.0 - co * co));
    Point p1 = scale(rotatePrecise(v, co, -si), c1.r) + c1.o;
    Point p2 = scale(rotatePrecise(v, co, si), c1.r) + c1.o;

    ret.push_back(p1);
    if (p1 != p2) ret.push_back(p2);
    return ret;
}
// intersection area between two circles c1, c2
Tf circleCircleIntersectionArea(Circle c1, Circle c2) {
    Point AB = c2.o - c1.o;
    Tf d = length(AB);
    if (d >= c1.r + c2.r) return 0;
    if (d + c1.r <= c2.r) return PI * c1.r * c1.r;
    if (d + c2.r <= c1.r) return PI * c2.r * c2.r;

    Tf alpha1 = acos((c1.r * c1.r + d * d - c2.r * c2.r) / (2.0 * c1.r * d));
    Tf alpha2 = acos((c2.r * c2.r + d * d - c1.r * c1.r) / (2.0 * c2.r * d));
    return c1.sector(2 * alpha1) + c2.sector(2 * alpha2);
}
// returns tangents from a point p to circle c, Tf Ti same
vector<Point> pointCircleTangents(Point p, Circle c) {
    vector<Point> ret;
    Point u = c.o - p;
    Tf d = length(u);
    if (d < c.r)
        ;
    else if (dcmp(d - c.r) == 0) {
        ret = {rotate(u, PI / 2)};
    } else {
        Tf ang = asin(c.r / d);
        ret = {rotate(u, -ang), rotate(u, ang)};
    }
    return ret;
}
// returns the points on tangents that touches the circle, Tf Ti Same
vector<Point> pointCircleTangencyPoints(Point p, Circle c) {
    Point u = p - c.o;
    Tf d = length(u);
    if (d < c.r)
        return {};
    else if (dcmp(d - c.r) == 0)
        return {c.o + u};
    else {
        Tf ang = acos(c.r / d);
        u = u / length(u) * c.r;
        return {c.o + rotate(u, -ang), c.o + rotate(u, ang)};
    }
}
// for two circles c1 && c2, returns two list of points a && b
// such that a[i] is on c1 && b[i] is c2 && for every i
// Line(a[i], b[i]) is a tangent to both circles
// CAUTION: a[i] = b[i] in case they touch | -1 for c1 = c2
int circleCircleTangencyPoints(Circle c1, Circle c2, vector<Point> &a, vector<Point> &b) {
        a.clear(), b.clear();
        int cnt = 0;
        
        if (dcmp(c1.r - c2.r) < 0) {
                swap(c1, c2);
                swap(a, b);
        }
        
        Tf d2 = sqLength(c1.o - c2.o);
        Tf rdif = c1.r - c2.r, rsum = c1.r + c2.r;
        
        if (dcmp(d2 - rdif * rdif) < 0) 
                return 0;
        if (dcmp(d2) == 0 && dcmp(c1.r - c2.r) == 0) 
                return -1;

        Tf base = angle(c2.o - c1.o);
        
        if (dcmp(d2 - rdif * rdif) == 0) {
                a.push_back(c1.point(base));
                b.push_back(c2.point(base));
                cnt++;
                return cnt;
        }

        Tf ang = acos((c1.r - c2.r) / sqrt(d2));
        a.push_back(c1.point(base + ang));
        b.push_back(c2.point(base + ang));
        cnt++;
        a.push_back(c1.point(base - ang));
        b.push_back(c2.point(base - ang));
        cnt++;

        if (dcmp(d2 - rsum * rsum) == 0) {
                a.push_back(c1.point(base));
                b.push_back(c2.point(PI + base));
                cnt++;
        } else if (dcmp(d2 - rsum * rsum) > 0) {
                Tf ang = acos((c1.r + c2.r) / sqrt(d2));
                a.push_back(c1.point(base + ang));
                b.push_back(c2.point(PI + base + ang));
                cnt++;
                a.push_back(c1.point(base - ang));
                b.push_back(c2.point(PI + base - ang));
                cnt++;
        }
        return cnt;
}
\end{lstlisting}
\subsection{Convex}
\begin{lstlisting}
/// minkowski sum of two polygons in O(n)
Polygon minkowskiSum(Polygon A, Polygon B) {
    int n = A.size(), m = B.size();
    rotate(A.begin(), min_element(A.begin(), A.end()), A.end());
    rotate(B.begin(), min_element(B.begin(), B.end()), B.end());

    A.push_back(A[0]);
    B.push_back(B[0]);
    for (int i = 0; i < n; i++) A[i] = A[i + 1] - A[i];
    for (int i = 0; i < m; i++) B[i] = B[i + 1] - B[i];

    Polygon C(n + m + 1);
    C[0] = A.back() + B.back();
    merge(A.begin(), A.end() - 1, B.begin(), B.end() - 1, C.begin() + 1,
                polarComp(Point(0, 0), Point(0, -1)));
    for (int i = 1; i < C.size(); i++) C[i] = C[i] + C[i - 1];
    C.pop_back();
    return C;
}
// finds the rectangle with minimum area enclosing a convex polygon and
// the rectangle with minimum perimeter enclosing a convex polygon
// Tf Ti Same
pair<Tf, Tf> rotatingCalipersBoundingBox(const Polygon &p) {
    using Linear::distancePointLine;
    int n = p.size();
    int l = 1, r = 1, j = 1;
    Tf area = 1e100;
    Tf perimeter = 1e100;
    for (int i = 0; i < n; i++) {
        Point v = (p[(i + 1) % n] - p[i]) / length(p[(i + 1) % n] - p[i]);
        while (dcmp(dot(v, p[r % n] - p[i]) - dot(v, p[(r + 1) % n] - p[i])) < 0)
            r++;
        while (j < r || dcmp(cross(v, p[j % n] - p[i]) -
                                                  cross(v, p[(j + 1) % n] - p[i])) < 0)
            j++;
        while (l < j ||
                      dcmp(dot(v, p[l % n] - p[i]) - dot(v, p[(l + 1) % n] - p[i])) > 0)
            l++;
        Tf w = dot(v, p[r % n] - p[i]) - dot(v, p[l % n] - p[i]);
        Tf h = distancePointLine(p[j % n], Line(p[i], p[(i + 1) % n]));
        area = min(area, w * h);
        perimeter = min(perimeter, 2 * w + 2 * h);
    }
    return make_pair(area, perimeter);
}
// returns the left side of polygon u after cutting it by ray a->b
Polygon cutPolygon(Polygon u, Point a, Point b) {
    using Linear::lineLineIntersection;
    using Linear::onSegment;

    Polygon ret;
    int n = u.size();
    for (int i = 0; i < n; i++) {
        Point c = u[i], d = u[(i + 1) % n];
        if (dcmp(cross(b - a, c - a)) >= 0) ret.push_back(c);
        if (dcmp(cross(b - a, d - c)) != 0) {
            Point t;
            lineLineIntersection(a, b - a, c, d - c, t);
            if (onSegment(t, Segment(c, d))) ret.push_back(t);
        }
    }
    return ret;
}
// returns true if point p is in or on triangle abc
bool pointInTriangle(Point a, Point b, Point c, Point p) {
    return dcmp(cross(b - a, p - a)) >= 0 && dcmp(cross(c - b, p - b)) >= 0 &&
                  dcmp(cross(a - c, p - c)) >= 0;
}
// pt must be in ccw order with no three collinear points
// returns inside = -1, on = 0, outside = 1
int pointInConvexPolygon(const Polygon &pt, Point p) {
    int n = pt.size();
    assert(n >= 3);

    int lo = 1, hi = n - 1;
    while (hi - lo > 1) {
        int mid = (lo + hi) / 2;
        if (dcmp(cross(pt[mid] - pt[0], p - pt[0])) > 0)
            lo = mid;
        else
            hi = mid;
    }

    bool in = pointInTriangle(pt[0], pt[lo], pt[hi], p);
    if (!in) return 1;

    if (dcmp(cross(pt[lo] - pt[lo - 1], p - pt[lo - 1])) == 0) return 0;
    if (dcmp(cross(pt[hi] - pt[lo], p - pt[lo])) == 0) return 0;
    if (dcmp(cross(pt[hi] - pt[(hi + 1) % n], p - pt[(hi + 1) % n])) == 0)
        return 0;
    return -1;
}
// Extreme Point for a direction is the farthest point in that direction
// u is the direction for extremeness
int extremePoint(const Polygon &poly, Point u) {
    int n = (int)poly.size();
    int a = 0, b = n;
    while (b - a > 1) {
        int c = (a + b) / 2;
        if (dcmp(dot(poly[c] - poly[(c + 1) % n], u)) >= 0 &&
                dcmp(dot(poly[c] - poly[(c - 1 + n) % n], u)) >= 0) {
            return c;
        }

        bool a_up = dcmp(dot(poly[(a + 1) % n] - poly[a], u)) >= 0;
        bool c_up = dcmp(dot(poly[(c + 1) % n] - poly[c], u)) >= 0;
        bool a_above_c = dcmp(dot(poly[a] - poly[c], u)) > 0;

        if (a_up && !c_up)
            b = c;
        else if (!a_up && c_up)
            a = c;
        else if (a_up && c_up) {
            if (a_above_c)
                b = c;
            else
                a = c;
        } else {
            if (!a_above_c)
                b = c;
            else
                a = c;
        }
    }

    if (dcmp(dot(poly[a] - poly[(a + 1) % n], u)) > 0 &&
            dcmp(dot(poly[a] - poly[(a - 1 + n) % n], u)) > 0)
        return a;
    return b % n;
}
// For a convex polygon p and a line l, returns a list of segments
// of p that touch or intersect line l.
// the i'th segment is considered (p[i], p[(i + 1) modulo |p|])
// #1 If a segment is collinear with the line, only that is returned
// #2 Else if l goes through i'th point, the i'th segment is added
// Complexity: O(lg |p|)
vector<int> lineConvexPolyIntersection(const Polygon &p, Line l) {
    assert((int)p.size() >= 3);
    assert(l.a != l.b);

    int n = p.size();
    vector<int> ret;

    Point v = l.b - l.a;
    int lf = extremePoint(p, rotate90(v));
    int rt = extremePoint(p, rotate90(v) * Ti(-1));
    int olf = orient(l.a, l.b, p[lf]);
    int ort = orient(l.a, l.b, p[rt]);

    if (!olf || !ort) {
        int idx = (!olf ? lf : rt);
        if (orient(l.a, l.b, p[(idx - 1 + n) % n]) == 0)
            ret.push_back((idx - 1 + n) % n);
        else
            ret.push_back(idx);
        return ret;
    }
    if (olf == ort) return ret;

    for (int i = 0; i < 2; ++i) {
        int lo = i ? rt : lf;
        int hi = i ? lf : rt;
        int olo = i ? ort : olf;

        while (true) {
            int gap = (hi - lo + n) % n;
            if (gap < 2) break;

            int mid = (lo + gap / 2) % n;
            int omid = orient(l.a, l.b, p[mid]);
            if (!omid) {
                lo = mid;
                break;
            }
            if (omid == olo)
                lo = mid;
            else
                hi = mid;
        }
        ret.push_back(lo);
    }
    return ret;
}
// Calculate [ACW, CW] tangent pair from an external point
constexpr int CW = -1, ACW = 1;
bool isGood(Point u, Point v, Point Q, int dir) {
    return orient(Q, u, v) != -dir;
}
Point better(Point u, Point v, Point Q, int dir) {
    return orient(Q, u, v) == dir ? u : v;
}
Point pointPolyTangent(const Polygon &pt, Point Q, int dir, int lo, int hi) {
    while (hi - lo > 1) {
        int mid = (lo + hi) / 2;
        bool pvs = isGood(pt[mid], pt[mid - 1], Q, dir);
        bool nxt = isGood(pt[mid], pt[mid + 1], Q, dir);

        if (pvs && nxt) return pt[mid];
        if (!(pvs || nxt)) {
            Point p1 = pointPolyTangent(pt, Q, dir, mid + 1, hi);
            Point p2 = pointPolyTangent(pt, Q, dir, lo, mid - 1);
            return better(p1, p2, Q, dir);
        }

        if (!pvs) {
            if (orient(Q, pt[mid], pt[lo]) == dir)
                hi = mid - 1;
            else if (better(pt[lo], pt[hi], Q, dir) == pt[lo])
                hi = mid - 1;
            else
                lo = mid + 1;
        }
        if (!nxt) {
            if (orient(Q, pt[mid], pt[lo]) == dir)
                lo = mid + 1;
            else if (better(pt[lo], pt[hi], Q, dir) == pt[lo])
                hi = mid - 1;
            else
                lo = mid + 1;
        }
    }

    Point ret = pt[lo];
    for (int i = lo + 1; i <= hi; i++) ret = better(ret, pt[i], Q, dir);
    return ret;
}
// [ACW, CW] Tangent
pair<Point, Point> pointPolyTangents(const Polygon &pt, Point Q) {
    int n = pt.size();
    Point acw_tan = pointPolyTangent(pt, Q, ACW, 0, n - 1);
    Point cw_tan = pointPolyTangent(pt, Q, CW, 0, n - 1);
    return make_pair(acw_tan, cw_tan);
}
\end{lstlisting}
\subsection{Polygon}
\begin{lstlisting}
typedef vector<Point> Polygon;
// removes redundant colinear points
// polygon can't be all colinear points
Polygon RemoveCollinear(const Polygon &poly) {
    Polygon ret;
    int n = poly.size();
    for (int i = 0; i < n; i++) {
        Point a = poly[i];
        Point b = poly[(i + 1) % n];
        Point c = poly[(i + 2) % n];
        if (dcmp(cross(b - a, c - b)) != 0 && (ret.empty() || b != ret.back()))
            ret.push_back(b);
    }
    return ret;
}
// returns the signed area of polygon p of n vertices
Tf signedPolygonArea(const Polygon &p) {
    Tf ret = 0;
    for (int i = 0; i < (int)p.size() - 1; i++)
        ret += cross(p[i] - p[0], p[i + 1] - p[0]);
    return ret / 2;
}
// given a polygon p of n vertices, generates the convex hull in in CCW
// Tested on https://acm.timus.ru/problem.aspx?space=1&num=1185
// Caution: when all points are colinear AND removeRedundant == false
// output will be contain duplicate points (from upper hull) at back
Polygon convexHull(Polygon p, bool removeRedundant) {
    int check = removeRedundant ? 0 : -1;
    sort(p.begin(), p.end());
    p.erase(unique(p.begin(), p.end()), p.end());

    int n = p.size();
    Polygon ch(n + n);
    int m = 0;  // preparing lower hull
    for (int i = 0; i < n; i++) {
        while (m > 1 &&
                      dcmp(cross(ch[m - 1] - ch[m - 2], p[i] - ch[m - 1])) <= check)
            m--;
        ch[m++] = p[i];
    }
    int k = m;  // preparing upper hull
    for (int i = n - 2; i >= 0; i--) {
        while (m > k &&
                      dcmp(cross(ch[m - 1] - ch[m - 2], p[i] - ch[m - 2])) <= check)
            m--;
        ch[m++] = p[i];
    }
    if (n > 1) m--;
    ch.resize(m);
    return ch;
}
// returns inside = -1, on = 0, outside = 1
int pointInPolygon(const Polygon &p, Point o) {
    using Linear::onSegment;
    int wn = 0, n = p.size();
    for (int i = 0; i < n; i++) {
        int j = (i + 1) % n;
        if (onSegment(o, Segment(p[i], p[j])) || o == p[i]) return 0;
        int k = dcmp(cross(p[j] - p[i], o - p[i]));
        int d1 = dcmp(p[i].y - o.y);
        int d2 = dcmp(p[j].y - o.y);
        if (k > 0 && d1 <= 0 && d2 > 0) wn++;
        if (k < 0 && d2 <= 0 && d1 > 0) wn--;
    }
    return wn ? -1 : 1;
}
// Given a simple polygon p, and a line l, returns (x, y)
// x = longest segment of l in p, y = total length of l in p.
pair<Tf, Tf> linePolygonIntersection(Line l, const Polygon &p) {
    using Linear::lineLineIntersection;
    int n = p.size();
    vector<pair<Tf, int>> ev;
    for (int i = 0; i < n; ++i) {
        Point a = p[i], b = p[(i + 1) % n], z = p[(i - 1 + n) % n];
        int ora = orient(l.a, l.b, a), orb = orient(l.a, l.b, b),
                orz = orient(l.a, l.b, z);
        if (!ora) {
            Tf d = dot(a - l.a, l.b - l.a);
            if (orz && orb) {
                if (orz != orb) ev.emplace_back(d, 0);
                // else  // Point Touch
            } else if (orz)
                ev.emplace_back(d, orz);
            else if (orb)
                ev.emplace_back(d, orb);
        } else if (ora == -orb) {
            Point ins;
            lineLineIntersection(l, Line(a, b), ins);
            ev.emplace_back(dot(ins - l.a, l.b - l.a), 0);
        }
    }
    sort(ev.begin(), ev.end());

    Tf ans = 0, len = 0, last = 0, tot = 0;
    bool active = false;
    int sign = 0;
    for (auto &qq : ev) {
        int tp = qq.second;
        Tf d = qq.first;  /// current Segment is (last, d)
        if (sign) {       /// On Border
            len += d - last;
            tot += d - last;
            ans = max(ans, len);
            if (tp != sign) active = !active;
            sign = 0;
        } else {
            if (active) {  /// Strictly Inside
                len += d - last;
                tot += d - last;
                ans = max(ans, len);
            }
            if (tp == 0)
                active = !active;
            else
                sign = tp;
        }
        last = d;
        if (!active) len = 0;
    }
    ans /= length(l.b - l.a);
    tot /= length(l.b - l.a);
    return {ans, tot};
}
\end{lstlisting}
\subsection{Half Plane}
\begin{lstlisting}
using Linear::lineLineIntersection;
struct DirLine {
    Point p, v;
    Tf ang;
    DirLine() {}
    /// Directed line containing point P in the direction v
    DirLine(Point p, Point v) : p(p), v(v) { ang = atan2(v.y, v.x); }
    bool operator<(const DirLine& u) const { return ang < u.ang; }
};
// returns true if point p is on the ccw-left side of ray l
bool onLeft(DirLine l, Point p) { return dcmp(cross(l.v, p - l.p)) >= 0; }

// Given a set of directed lines returns a polygon such that
// the polygon is the intersection by halfplanes created by the
// left side of the directed lines. MAY CONTAIN DUPLICATE POINTS
int halfPlaneIntersection(vector<DirLine>& li, Polygon& poly) {
    int n = li.size();
    sort(li.begin(), li.end());

    int first, last;
    Point* p = new Point[n];
    DirLine* q = new DirLine[n];
    q[first = last = 0] = li[0];

    for (int i = 1; i < n; i++) {
        while (first < last && !onLeft(li[i], p[last - 1])) last--;
        while (first < last && !onLeft(li[i], p[first])) first++;
        q[++last] = li[i];

        if (dcmp(cross(q[last].v, q[last - 1].v)) == 0) {
            last--;
            if (onLeft(q[last], li[i].p)) q[last] = li[i];
        }

        if (first < last)
            lineLineIntersection(q[last - 1].p, q[last - 1].v, q[last].p, q[last].v,
                                                      p[last - 1]);
    }

    while (first < last && !onLeft(q[first], p[last - 1])) last--;
    if (last - first <= 1) {
        delete[] p;
        delete[] q;
        poly.clear();
        return 0;
    }
    lineLineIntersection(q[last].p, q[last].v, q[first].p, q[first].v, p[last]);

    int m = 0;
    poly.resize(last - first + 1);
    for (int i = first; i <= last; i++) poly[m++] = p[i];
    delete[] p;
    delete[] q;
    return m;
}
\end{lstlisting}
\end{multicols*}
\begin{multicols*}{3}
\newpage
\section{Equations and Formulas}
\subsection{Catalan Numbers}
$\displaystyle C_n=\frac{1}{n+1}{2n \choose n}$
$\displaystyle C_0=1,C_1=1\text{ and }C_n=\sum \limits_{k=0}^{n-1}C_k C_{n-1-k}$ \\
The number of ways to completely parenthesize $n$+$\displaystyle 1$ factors. \\
The number of triangulations of a convex polygon with $n$+$\displaystyle 2$ sides (i.e. the number of partitions of polygon into disjoint triangles by using the diagonals). \\
The number of ways to connect the $\displaystyle 2n$ points on a circle to form $n$ disjoint i.e. non-intersecting chords. \\
The number of rooted full binary trees with $n$+$\displaystyle 1$ leaves (vertices are not numbered). A rooted binary tree is full if every vertex has either two children or no children. \\
Number of permutations of $\displaystyle {1, …, n}$ that avoid the pattern $\displaystyle 123$ (or any of the other patterns of length $3$); that is, the number of permutations with no three-term increasing sub-sequence. For $n = 3$, these permutations are $\displaystyle 132,\ 213,\ 231,\ 312$ and $321.$

\subsection{Stirling Numbers First Kind}
The Stirling numbers of the first kind count permutations according to their number of cycles (counting fixed points as cycles of length one). \\
$S(n,k)$ counts the number of permutations of $n$ elements with $\displaystyle \displaystyle k$ disjoint cycles. \\
$S(n,k)=(n-1) \cdot S(n-1,k)+S(n-1,k-1),$ \(where,\; S(0,0)=1,S(n,0)=S(0,n)=0\)
$\displaystyle \displaystyle\sum_{k=0}^{n}S(n,k) = n!$ \\
The unsigned Stirling numbers may also be defined algebraically, as the coefficient of the rising factorial:
\[\displaystyle x^{\bar{n}} = x(x+1)...(x+n-1) = \sum_{k=0}^{n}{ S(n, k) x^k}\]
Lets $[n, k]$ be the stirling number of the first kind, then

\[\displaystyle \bigl[\!\begin{smallmatrix} n \\ n\ -\ k \end{smallmatrix}\!\bigr] = \sum_{0 \leq i_1 < i_2 < i_k < n}{i_1i_2....i_k.}\]

\subsection{Stirling Numbers Second Kind}
Stirling number of the second kind is the number of ways to partition a set of n objects into k non-empty subsets. \\
$S(n,k)=k \cdot S(n-1,k)+S(n-1,k-1)$, \(where \; S(0,0)=1,S(n,0)=S(0,n)=0\)
$S(n,2)=2^{n-1}-1$ 
$S(n,k) \cdot k!$ = number of ways to color $n$ nodes using colors from $\displaystyle 1$ to $\displaystyle \displaystyle k$ such that each color is used at least once. \\
An $r$-associated Stirling number of the second kind is the number of ways to partition a set of $n$ objects into $\displaystyle \displaystyle k$ subsets, with each subset containing at least $r$ elements. It is denoted by $S_r( n , k )$ and obeys the recurrence relation. $\displaystyle \displaystyle S_r(n+1,k) = k S_r(n,k) + \binom{n}{r-1} S_r(n-r+1,k-1)$ \\ 
Denote the n objects to partition by the integers $\displaystyle 1, 2, …., n$. Define the reduced Stirling numbers of the second kind, denoted $S^d(n, k)$, to be the number of ways to partition the integers $\displaystyle 1, 2, …., n$ into k nonempty subsets such that all elements in each subset have pairwise distance at least d. That is, for any integers i and j in a given subset, it is required that $|i - j| \geq d$. It has been shown that these numbers satisfy, \(S^d(n, k) = S(n - d + 1, k - d + 1), n \geq k \geq d\)
\subsection{Other Combinatorial Identities}
$\displaystyle \displaystyle {n \choose k}=\frac{n}{k}{n-1 \choose k-1}$ \\
$\displaystyle \sum \limits_{i= 0}^k{n+i \choose i}= \sum \limits_{i= 0}^k{n+i \choose n} = {n+k+1 \choose k}$ \\
$\displaystyle \ n,r \in N, n > r, \sum \limits_{i=r}^n{i \choose r}={n+1 \choose r+1}$ \\
If $\displaystyle P(n)=\sum_{k=0}^{n}{n \choose k} \cdot Q(k)$, then,
\[Q(n)=\sum_{k=0}^{n}(-1)^{n-k}{n \choose k} \cdot P(k)\] \\
If $\displaystyle P(n)=\sum_{k=0}^{n}(-1)^{k}{n \choose k} \cdot Q(k)$ , then,
\[Q(n)=\sum_{k=0}^{n}(-1)^{k}{n \choose k} \cdot P(k)\]

\subsection{Different Math Formulas}
\textbf{Picks Theorem : } $ A = i + b / 2 - 1 $ \\ 
\textbf{Deragements : } $ d(i) = (i - 1) \times \left( d(i - 1) + d(i - 2) \right) $ \\ 
\begin{multline*}
\displaystyle \frac{n}{ab}-\Big\{\frac{b{\prime} n}{a}\Big\}-\Big\{\frac{a{\prime} n}{b}\Big\} + 1
\end{multline*}
\subsection{GCD and LCM}
if $m$ is any integer, then $\displaystyle \gcd(a + m {\cdot} b, b) = \gcd(a, b)$ \\
The gcd is a multiplicative function in the following sense: if $\displaystyle a_1$ and $\displaystyle a_2$ are relatively prime, then $\displaystyle \gcd(a_1 \cdot a_2, b) = \gcd(a_1, b) \cdot \gcd(a_2,b )$. \\
$\displaystyle \gcd(a, \operatorname{lcm}(b, c)) = \operatorname{lcm}(\gcd(a, b), \gcd(a, c))$. \\
$\displaystyle \operatorname{lcm}(a, \gcd(b, c)) = \gcd(\operatorname{lcm}(a, b), \operatorname{lcm}(a, c))$. \\
For non-negative integers $\displaystyle a$ and $b$, where $\displaystyle a$ and $b$ are not both zero, $\displaystyle \gcd({n^a} - 1, {n^b} - 1) = n^{\gcd(a,b)} - 1$ \\
$\displaystyle \gcd(a, b) = \displaystyle \sum_{k|a \, \text{and} \, k|b} {\phi(k)}$ \\
$\displaystyle \displaystyle \sum_{i=1}^n [\gcd(i, n) = k] = { \phi{\left(\frac{n}{k}\right)}}$ \\
$\displaystyle \displaystyle \sum_{k=1}^n \gcd(k, n) = \displaystyle \sum_{d|n} d \cdot {\phi{\left(\frac{n}{d}\right)}}$ \\
$\displaystyle \displaystyle \sum_{k=1}^n x^{\gcd(k,n)} = \displaystyle \sum_{d|n} x^d \cdot {\phi{\left(\frac{n}{d}\right)}}$ \\
$\displaystyle \displaystyle \sum_{k=1}^n \frac{1}{\gcd(k, n)} = \displaystyle \sum_{d|n} \frac{1}{d} \cdot {\phi{\left(\frac{n}{d}\right)}} = \frac{1}{n} \displaystyle \sum_{d|n} d \cdot \phi(d)$ \\
$\displaystyle \displaystyle \sum_{k=1}^n \frac{k}{\gcd(k, n)} = \frac{n}{2} \cdot \displaystyle \sum_{d|n} \frac{1}{d} \cdot {\phi{\left(\frac{n}{d}\right)}} = \frac{n}{2} \cdot \frac{1}{n} \cdot \displaystyle \sum_{d|n} d \cdot \phi(d)$ \\
$\displaystyle \displaystyle \sum_{k=1}^n \frac{n}{\gcd(k, n)} = 2 * \displaystyle \sum_{k=1}^n \frac{k}{\gcd(k, n)} - 1$, for $n > 1$ \\
$\displaystyle \displaystyle \sum_{i=1}^n \sum_{j=1}^n [\gcd(i, j) = 1] = \displaystyle \sum_{d=1}^n \mu(d) \lfloor {\frac{n}{d} \rfloor}^2$ \\
$\displaystyle \displaystyle \sum_{i=1}^n \displaystyle\sum_{j=1}^n \gcd(i, j) = \displaystyle \sum_{d=1}^n \phi(d) \lfloor {\frac{n}{d} \rfloor}^2$ \\
$\displaystyle \sum_{i=1}^n \sum_{j=1}^n i \cdot j[\gcd(i, j) = 1] = \sum_{i=1}^n \phi(i)i^2$ \\
$\displaystyle F(n) = \displaystyle \sum_{i=1}^n \displaystyle \sum_{j=1}^n \operatorname{lcm}(i, j) = \displaystyle \sum_{l=1}^n {\left(\frac{\left( 1 + \lfloor{\frac{n}{l} \rfloor} \right) \left( \lfloor{\frac{n}{l} \rfloor} \right)} {2} \right)}^2 \displaystyle \sum_{d|l} \mu(d)ld$ \\



\subsection{Geometry}
\textbf{Cone:} \( V = \frac{1}{3} \pi r^2 h \), \( A = \pi r (r + \sqrt{h^2 + r^2}) \) \\
\textbf{Pyramid:} \( V = \frac{1}{3} \times \text{base} \times \text{height} \), \( A = \text{base area} + \frac{1}{2} \times \text{perimeter} \times \text{slant height} \) \\
\textbf{Triangular Prism:} \( V = \frac{1}{2} \times \text{base} \times \text{height} \times \text{depth} \), \( A = \text{base} \times \text{height} + 3 \times \left(\frac{1}{2} \times \text{side} \times \text{perimeter} \right) \) \\
\textbf{Torus:} \( V = 2 \pi^2 R r^2 \), \( A = 4 \pi^2 R r \) \\
\textbf{Ellipsoid:} \( V = \frac{4}{3} \pi a b c \), \( A = 4 \pi \left( \frac{(ab)^{1.6} + (bc)^{1.6} + (ca)^{1.6}}{3} \right)^{1/1.6} \)



\end{multicols*}

\end{document}
