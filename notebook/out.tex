\newcommand\TEAM{IUT Songkolpo}
\newcommand\UNI{Islamic University of Technology}
\newcommand\COLS{3}
\newcommand\ORT{landscape}
\newcommand\FSZ{10}
\documentclass[FSZ,a4paper,onesided]{article}
\usepackage[utf8]{inputenc}
\usepackage{amsmath}
\usepackage{listings}
\usepackage{graphicx}
\usepackage{multicol}
\usepackage[utf8]{inputenc}
\usepackage[english]{babel}
\usepackage[usenames,dvipsnames]{color}
\usepackage{verbatim}
\usepackage{hyperref}
\usepackage{geometry}
\usepackage{fancyhdr}
\usepackage{titlesec}
\usepackage{pdflscape}
\geometry{verbose,\ORT,a4paper,tmargin=1.0cm,bmargin=.5cm,lmargin=.5cm,rmargin=.5cm, headsep=.5cm}

\definecolor{dkgreen}{rgb}{0,0.6,0}
\definecolor{gray}{rgb}{0.5,0.5,0.5}
\definecolor{mauve}{rgb}{0.58,0,0.82}

\lstset{frame=tb,
  language=C++,
  aboveskip=1mm,
  belowskip=1mm,
  showstringspaces=false,
  columns=flexible,
  basicstyle={\small\ttfamily},
  numbers=none,
  numberstyle=\footnotesize\color{gray},
  keywordstyle=\color{blue},
  commentstyle=\color{dkgreen},
  stringstyle=\color{mauve},
  breaklines=true,
  breakatwhitespace=false,
  tabsize=1
}

\fancyhf{}
\renewcommand{\headrulewidth}{1pt}
\pagestyle{fancy}
\lhead{\large{\textbf{\TEAM}, \textbf{\UNI}}}
\rhead{\thepage}

\titleformat*{\section}{\large\bfseries}
\titleformat*{\subsection}{\normalsize\bfseries}
\titleformat*{\subsubsection}{\normalsize}
\titlespacing*{\section}
{0pt}{0ex}{0ex}
\titlespacing*{\subsection}
{0pt}{0ex}{0ex}
\titlespacing*{\subsubsection}
{0pt}{0ex}{0ex}
\setlength{\columnsep}{0.05in}
\setlength{\columnseprule}{1px}


\begin{document}

\begin{multicols*}{\COLS}
\pagenumbering{gobble}
\tableofcontents
\newpage
\pagenumbering{arabic}
\lstloadlanguages{C++,Java}
\subsection*{Sublime Build}
\begin{lstlisting}[language= Pascal, commentstyle=\color{black}, numberstyle=\tiny\color{black}, keywordstyle=\color{black}, stringstyle=\color{black},
]
{
    "shell_cmd": "g++ -std=c++17 -o ${file_path}/${file_base_name} ${file} && ${file_path}/${file_base_name} < input.txt > output.txt",
    "working_dir": "${file_path}",
    "selector": "source.cpp"
}
\end{lstlisting}
\subsection*{Sublime Build Ubuntu}

\begin{lstlisting}[language= Pascal, commentstyle=\color{black}, numberstyle=\tiny\color{black}, keywordstyle=\color{black}, stringstyle=\color{black},
]
{
"cmd" : ["g++ -std=c++20 -DLOCAL $file_name -o $file_base_name 
&&timeout 4s ./$file_base_name<inputf.in>outputf.in"],
"selector" : "source.cpp",
"file_regex": "^(..[^:]*):([0-9]+):?([0-9]+)?:? (.*)$",
"shell": true,
"working_dir" : "$file_path"
}
\end{lstlisting}

\subsection*{Stress-tester}
\begin{lstlisting}[language= Pascal, commentstyle=\color{black}, numberstyle=\tiny\color{black}, keywordstyle=\color{black}, stringstyle=\color{black},
]
#!/bin/bash
# Call as stresstester ITERATIONS [--count]

g++ gen.cpp -o gen
g++ sol.cpp -o sol
g++ brute.cpp -o brute

for i in $(seq 1 "$1") ; do
    echo "Attempt $i/$1"
    ./gen > in.txt
    ./sol < in.txt > out1.txt
    ./brute < in.txt > out2.txt
    diff -y out1.txt out2.txt > diff.txt
    if [ $? -ne 0 ] ; then
        echo "Differing Testcase Found:"; cat in.txt
        echo -e "\nOutputs:"; cat diff.txt
        break
    fi
done
\end{lstlisting}


\section{All Macros}
\begin{lstlisting}
/*--- DEBUG TEMPLATE STARTS HERE ---*/
void show(int x) {cerr << x;}
void show(long long x) {cerr << x;}
void show(double x) {cerr << x;}
void show(char x) {cerr << '\'' << x << '\'';}
void show(const string &x) {cerr << '\"' << x << '\"';}
void show(bool x) {cerr << (x ? "true" : "false");}

template<typename T, typename V>
void show(pair<T, V> x) { cerr << '{'; show(x.first); cerr << ", "; show(x.second); cerr << '}'; }
template<typename T>
void show(T x) {int f = 0; cerr << "{"; for (auto &i: x) cerr << (f++ ? ", " : ""), show(i); cerr << "}";}

void debug_out(string s) {
  s.clear();
  cerr << s << '\n';
}
template <typename T, typename... V>
void debug_out(string s, T t, V... v) {
  s.erase(remove(s.begin(), s.end(), ' '), s.end());
  cerr << "        "; // 8 spaces
  cerr << s.substr(0, s.find(','));
  s = s.substr(s.find(',') + 1);
  cerr << " = ";
  show(t);
  cerr << endl;
  if(sizeof...(v)) debug_out(s, v...);
}
#define dbg(x...) cerr << "LINE: " << __LINE__ << endl; debug_out(#x, x); cerr << endl; 
/*--- DEBUG TEMPLATE ENDS HERE ---*/
//#pragma GCC optimize("Ofast")
//#pragma GCC optimization ("O3")
//#pragma comment(linker, "/stack:200000000")
//#pragma GCC optimize("unroll-loops")
//#pragma GCC target("sse,sse2,sse3,ssse3,sse4,popcnt,abm,mmx,avx,tune=native")

#include <bits/stdc++.h>
#include <ext/pb_ds/assoc_container.hpp>
#include <ext/pb_ds/tree_policy.hpp>
using namespace std;
using namespace __gnu_pbds;
  //find_by_order(k) --> returns iterator to the kth largest element counting from 0
  //order_of_key(val) --> returns the number of items in a set that are strictly smaller than our item
template <typename DT> 
using ordered_set = tree <DT, null_type, less<DT>, rb_tree_tag,tree_order_statistics_node_update>;
mt19937 rnd(chrono::steady_clock::now().time_since_epoch().count());

#ifdef LOCAL
#include "dbg.h"
#else
#define dbg(x...)
#endif

int main() {
  cin.tie(0) -> sync_with_stdio(0);
}
\end{lstlisting}
\section{Data Structure}
\subsection{Segment Tree}
\begin{lstlisting}
template <typename VT>
struct segtree {
  using DT = typename VT::DT;
  using LT = typename VT::LT;

  int L, R;
  vector <VT> tr;
  segtree(int n): L(0), R(n - 1), tr(n << 2) {}
  segtree(int l, int r): L(l), R(r), tr((r - l + 1) << 2) {}

  void propagate(int l, int r, int u) {
    if(l == r) return;
    VT :: apply(tr[u << 1], tr[u].lz, l, (l + r) >> 1);
    VT :: apply(tr[u << 1 | 1], tr[u].lz, (l + r + 2) >> 1, r);
    tr[u].lz = VT :: None;
  }

  void build(int l, int r, vector <DT> &v, int u = 1 ) {
    if(l == r) {
      tr[u].val = v[l];
      return;
    }
    int m = (l + r) >> 1, lft = u << 1, ryt = u << 1 | 1;
    build(l, m, v, lft);
    build(m + 1, r, v, ryt);
    tr[u].val = VT :: merge(tr[lft].val, tr[ryt].val, l, r);
  }

  void update(int ql,int qr, LT up, int l, int r, int u = 1) {
    if(ql > qr) return;
    if(ql == l and qr == r) {
      VT :: apply(tr[u], up, l, r);
      return;
    }
    propagate(l, r, u);
    int m = (l + r) >> 1, lft = u << 1, ryt = u << 1 | 1;
    update(ql, min(m, qr), up,  l,  m, lft);
    update(max(ql, m + 1), qr, up, m + 1, r, ryt);
    tr[u].val = VT :: merge(tr[lft].val, tr[ryt].val, l, r);
  }

  DT query(int ql, int qr, int l, int r, int u = 1) {
    if(ql > qr) return VT::I;
    if(l == ql and r == qr)
      return tr[u].val;
    propagate(l, r, u);
    int m = (l + r) >> 1, lft = u << 1, ryt = u << 1 | 1;
    DT ansl = query(ql, min(m, qr), l, m, lft);
    DT ansr = query(max(ql, m + 1), qr, m + 1, r, ryt);
    return tr[u].merge(ansl, ansr, l, r);
  }

  void build(vector <DT> &v) { build(L, R, v); }
  void update(int ql, int qr, LT U) { update(ql, qr, U, L, R); }
  DT query(int ql, int qr) { return query(ql, qr, L, R); }
};


struct add_sum {
  using DT = LL;
  using LT = LL;
  DT val;
  LT lz;

  static constexpr DT I = 0; 
  static constexpr LT None = 0;
  
  add_sum(DT _val = I, LT _lz = None): val(_val), lz(_lz) {}

  static void apply(add_sum &u, const LT &up, int l, int r) {
    u.val += (r - l + 1) * up;
    u.lz += up;
  }

  static DT merge(const DT &a, const DT &b, int l, int r) {
    return a + b;
  }
};
\end{lstlisting}
\subsection{Spare Table}
\begin{lstlisting}
template <typename T> struct sparse_table {
  vector <vector<T>> tbl;
  function < T(T, T) > f;
  T id;

  sparse_table(const vector <T> &v, function <T(T, T)> _f, T _id) : f(_f), id(_id) {
    int n = (int) v.size(), b = __lg(n);
    tbl.assign(b + 1, v);
    for(int k = 1; k <= b; k++) {
      for(int i = 0; i + (1 << k) <= n; i++) {
        tbl[k][i] = f(tbl[k - 1][i], tbl[k - 1][i + (1 << (k - 1))]);
      }
    }
  }
  T query(int l, int r) {
    if(l > r) return id;
    int pow = __lg(r - l + 1);
    return f(tbl[pow][l], tbl[pow][r - (1 << pow) + 1]);
  }
};
\end{lstlisting}
\subsection{Persistent Segment Tree}
\begin{lstlisting}
struct Node {
  int l = 0, r = 0, val = 0;
} tr[20 * N];
int ptr = 0;
int build(int st, int en) {
  int u = ++ptr;
  if (st == en) return u;
  int mid = (st + en) / 2;
  auto& [l, r, val] = tr[u];
  l = build(st, mid);
  r = build(mid + 1, en);
  val = tr[l].val + tr[r].val;
  return u;
}

int update(int pre, int st, int en, int idx, int v) {
  int u = ++ptr;
  tr[u] = tr[pre];
  if (st == en) {
    tr[u].val += v;
    return u;
  }
  int mid = (st + en) / 2;
  auto& [l, r, val] = tr[u];
  if (idx <= mid) {
    r = tr[pre].r;
    l = update(tr[pre].l, st, mid, idx, v);
  } else {
    l = tr[pre].l;
    r = update(tr[pre].r, mid + 1, en, idx, v);
  }
  tr[u].val = tr[l].val + tr[r].val;
  return u;
}
// finding the kth elelment in a range
int query(int left, int right, int st, int en, int k) {
  if (st == en) return st;
  int cnt = tr[tr[right].l].val - tr[tr[left].l].val;
  int mid = (st + en) / 2;
  if (cnt >= k) return query(tr[left].l, tr[right].l, st, mid, k);
  else return query(tr[left].r, tr[right].r, mid + 1, en, k - cnt);
}
int V[N], root[N], a[N];
int main() {
  map<int, int> mp; int n, q;
  cin >> n >> q;
  for (int i = 1; i <= n; i++) cin >> a[i], mp[a[i]];
  int c = 0;
  for (auto x : mp) mp[x.first] = ++c, V[c] = x.first;
  root[0] = build(1, n);
  for (int i = 1; i <= n; i++) {
    root[i] = update(root[i - 1], 1, n, mp[a[i]], 1);
  }
  while (q--) {
    int l, r, k; cin >> l >> r >> k; l++, k++;
    cout << V[query(root[l - 1], root[r], 1, n, k)] << '\n';
  }
  return 0;
}
\end{lstlisting}
\subsection{SegTree Beats}
\begin{lstlisting}
const int N = 2e5 + 5;
LL mx[4 * N], mn[4 * N], smx[4 * N], smn[4 * N], sum[4 * N], add[4 * N];
int mxcnt[4 * N], mncnt[4 * N];

int L, R;

void applyMax(int u, LL x) {
  sum[u] += mncnt[u] * (x - mn[u]);
  if (mx[u] == mn[u]) mx[u] = x;
  if (smx[u] == mn[u]) smx[u] = x;
  mn[u] = x;
}
void applyMin(int u, LL x) {
  sum[u] -= mxcnt[u] * (mx[u] - x);
  if (mn[u] == mx[u]) mn[u] = x;
  if (smn[u] == mx[u]) smn[u] = x;
  mx[u] = x;
}
void applyAdd(int u, LL x, int tl, int tr) {
  sum[u] += (tr - tl + 1) * x;
  add[u] += x;
  mx[u] += x, mn[u] += x;
  if (smx[u] != -INF) smx[u] += x;
  if (smn[u] != INF) smn[u] += x;
}
void push(int u, int tl, int tr) {
  int lft = u << 1, ryt = lft | 1, mid = tl + tr >> 1;
  if (add[u] != 0) {
    applyAdd(lft, add[u], tl, mid);
    applyAdd(ryt, add[u], mid + 1, tr);
    add[u] = 0;
  }
  if (mx[u] < mx[lft]) applyMin(lft, mx[u]);
  if (mx[u] < mx[ryt]) applyMin(ryt, mx[u]);

  if (mn[u] > mn[lft]) applyMax(lft, mn[u]);
  if (mn[u] > mn[ryt]) applyMax(ryt, mn[u]);
}
void merge(int u) {
  int lft = u << 1, ryt = lft | 1;
  sum[u] = sum[lft] + sum[ryt];

  mx[u] = max(mx[lft], mx[ryt]);
  smx[u] = max(smx[lft], smx[ryt]);
  if (mx[lft] != mx[ryt]) smx[u] = max(smx[u], min(mx[lft], mx[ryt]));
  mxcnt[u] = (mx[u] == mx[lft]) * mxcnt[lft] + (mx[u] == mx[ryt]) * mxcnt[ryt];

  mn[u] = min(mn[lft], mn[ryt]);
  smn[u] = min(smn[lft], smn[ryt]);
  if (mn[lft] != mn[ryt]) smn[u] = min(smn[u], max(mn[lft], mn[ryt]));
  mncnt[u] = (mn[u] == mn[lft]) * mncnt[lft] + (mn[u] == mn[ryt]) * mncnt[ryt];
}
void minimize(int l, int r, LL x, int tl = L, int tr = R, int u = 1) {
  if (l > tr or tl > r or mx[u] <= x) return;
  if (l <= tl and tr <= r and smx[u] < x) {
    applyMin(u, x);
    return;
  }
  push(u, tl, tr);
  int mid = tl + tr >> 1, lft = u << 1, ryt = lft | 1;
  minimize(l, r, x, tl, mid, lft);
  minimize(l, r, x, mid + 1, tr, ryt);
  merge(u);
}
void maximize(int l, int r, LL x, int tl = L, int tr = R, int u = 1) {
  if (l > tr or tl > r or mn[u] >= x) return;
  if (l <= tl and tr <= r and smn[u] > x) {
    applyMax(u, x);
    return;
  }
  push(u, tl, tr);
  int mid = tl + tr >> 1, lft = u << 1, ryt = lft | 1;
  maximize(l, r, x, tl, mid, lft);
  maximize(l, r, x, mid + 1, tr, ryt);
  merge(u);
}
void increase(int l, int r, LL x, int tl = L, int tr = R, int u = 1) {
  if (l > tr or tl > r) return;
  if (l <= tl and tr <= r) {
    applyAdd(u, x, tl, tr);
    return;
  }
  push(u, tl, tr);
  int mid = tl + tr >> 1, lft = u << 1, ryt = lft | 1;
  increase(l, r, x, tl, mid, lft);
  increase(l, r, x, mid + 1, tr, ryt);
  merge(u);
}
LL getSum(int l, int r, int tl = L, int tr = R, int u = 1) {
  if (l > tr or tl > r) return 0;
  if (l <= tl and tr <= r) return sum[u];
  push(u, tl, tr);
  int mid = tl + tr >> 1, lft = u << 1, ryt = lft | 1;
  return getSum(l, r, tl, mid, lft) + getSum(l, r, mid + 1, tr, ryt);
}
void build(LL a[], int tl = L, int tr = R, int u = 1) {
  if (tl == tr) {
    sum[u] = mn[u] = mx[u] = a[tl];
    mxcnt[u] = mncnt[u] = 1;
    smx[u] = -INF;
    smn[u] = INF;
    return;
  }
  int mid = tl + tr >> 1, lft = u << 1, ryt = lft | 1;
  build(a, tl, mid, lft);
  build(a, mid + 1, tr, ryt);
  merge(u);
}
void init(LL a[], int _L, int _R) {
  L = _L, R = _R;
  build(a);
}
\end{lstlisting}
\subsection{HashTable}
\begin{lstlisting}
#include <ext/pb_ds/assoc_container.hpp>
using namespace __gnu_pbds;

const int RANDOM = chrono::high_resolution_clock::now().time_since_epoch().count();
unsigned hash_f(unsigned x) {
  x = ((x >> 16) ^ x) * 0x45d9f3b;
  x = ((x >> 16) ^ x) * 0x45d9f3b;
  return x = (x >> 16) ^ x;
}

unsigned hash_combine(unsigned a, unsigned b) { return a * 31 + b; }
struct chash {
  int operator()(int x) const { return hash_f(x); }
};
typedef gp_hash_table<int, int, chash> gp;
gp table;
\end{lstlisting}
\subsection{DSU With Rollbacks}
\begin{lstlisting}
struct Rollback_DSU {
  int n;
  vector<int> par, sz;
  vector<pair<int, int>> op;
  Rollback_DSU(int n) : par(n), sz(n, 1) {
    iota(par.begin(), par.end(), 0);
    op.reserve(n);
  }
  int Anc(int node) {
    for (; node != par[node]; node = par[node])
      ;  // no path compression
    return node;
  }
  void Unite(int x, int y) {
    if (sz[x = Anc(x)] < sz[y = Anc(y)]) swap(x, y);
    op.emplace_back(x, y);
    par[y] = x;
    sz[x] += sz[y];
  }
  void Undo(int t) {
    for (; op.size() > t; op.pop_back()) {
      par[op.back().second] = op.back().second;
      sz[op.back().first] -= sz[op.back().second];
    }
  }
};
\end{lstlisting}
\subsection{Binary Trie}
\begin{lstlisting}
const int N = 1e7 + 5, b = 30;
int tc = 1;
struct node {
  int vis = 0;
  int to[2] = {0, 0};
  int val[2] = {0, 0};
  void update() {
    to[0] = to[1] = 0;
    val[0] = val[1] = 0;
    vis = tc;
  }
} T[N + 2];
node *root = T;
int ptr = 0;
node *nxt(node *cur, int x) {
  if (cur->to[x] == 0) cur->to[x] = ++ptr;
  assert(ptr < N);
  int idx = cur->to[x];
  if (T[idx].vis < tc) T[idx].update();
  return T + idx;
}
int query(int j, int aj) {
  int ans = 0, jaj = j ^ aj;
  node *cur = root;
  for (int k = b - 1; ~k; k--) {
    maximize(ans, nxt(cur, (jaj >> k & 1) ^ 1)->val[1 ^ (aj >> k & 1)]);
    cur = nxt(cur, (jaj >> k & 1));
  }
  return ans;
}
void insert(int j, int aj, int val) {
  int jaj = j ^ aj;
  node *cur = root;
  for (int k = b - 1; ~k; k--) {
    cur = nxt(cur, (jaj >> k & 1));
    maximize(cur->val[j >> k & 1], val);
  }
}
void clear() {
  tc++;
  ptr = 0;
  root->update();
}
\end{lstlisting}
\subsection{BIT-2D}
\begin{lstlisting}
const int N = 1008;
int bit[N][N], n, m;
int a[N][N], q;
void update(int x, int y, int val) {
  for (; x < N; x += -x & x)
    for (int j = y; j < N; j += -j & j) bit[x][j] += val;
}
int get(int x, int y) {
  int ans = 0;
  for (; x; x -= x & -x)
    for (int j = y; j; j -= j & -j) ans += bit[x][j];
  return ans;
}
int get(int x1, int y1, int x2, int y2) {
  return get(x2, y2) - get(x1 - 1, y2) - get(x2, y1 - 1) + get(x1 - 1, y1 - 1);
}
\end{lstlisting}
\subsection{Divide And Conquer Query Offline}
\begin{lstlisting}
namespace up {
int l[N], r[N], u[N], v[N], tm;
void push(int _l, int _r, int _u, int _v) {
  l[tm] = _l, r[tm] = _r, u[tm] = _u, v[tm] = _v;
  tm++;
}
}  // namespace up
namespace que {
int node[N], tm;
LL ans[N];
void push(int _node) { node[++tm] = _node; }
}  // namespace que
namespace edge_set {
void push(int i) { dsu ::merge(up ::u[i], up ::v[i]); }
void pop(int t) { dsu ::rollback(t); }
int time() { return dsu ::op.size(); }
LL query(int u) { return a[dsu ::root(u)]; }
}  // namespace edge_set
namespace dncq {
vector<int> tree[4 * N];
void update(int idx, int l = 0, int r = que ::tm, int node = 1) {
  int ul = up ::l[idx], ur = up ::r[idx];
  if (r < ul or ur < l) return;
  if (ul <= l and r <= ur) {
    tree[node].push_back(idx);
    return;
  }
  int m = l + r >> 1;
  update(idx, l, m, node << 1);
  update(idx, m + 1, r, node << 1 | 1);
}
void dfs(int l = 0, int r = que ::tm, int node = 1) {
  int cur = edge_set ::time();
  for (int e : tree[node]) edge_set ::push(e);
  if (l == r) {
    que ::ans[l] = edge_set ::query(que ::node[l]);
  } else {
    int m = l + r >> 1;
    dfs(l, m, node << 1);
    dfs(m + 1, r, node << 1 | 1);
  }
  edge_set ::pop(cur);
}
}  // namespace dncq
void push_updates() {
  for (int i = 0; i < up ::tm; i++) dncq ::update(i);
}
\end{lstlisting}
\subsection{MO with Update}
\begin{lstlisting}
const int N = 1e5 + 5, sz = 2700, bs = 25;
int arr[N], freq[2 * N], cnt[2 * N], id[N], ans[N];
struct query {
    int l, r, t, L, R;
    query(int l = 1, int r = 0, int t = 1, int id = -1)
            : l(l), r(r), t(t), L(l / sz), R(r / sz) {}
    bool operator<(const query &rhs) const {
        return (L < rhs.L) or (L == rhs.L and R < rhs.R) or
                      (L == rhs.L and R == rhs.R and t < rhs.t);
    }
} Q[N];
struct update {
    int idx, val, last;
} Up[N];
int qi = 0, ui = 0;
int l = 1, r = 0, t = 0;

void add(int idx) {
    --cnt[freq[arr[idx]]];
    freq[arr[idx]]++;
    cnt[freq[arr[idx]]]++;
}
void remove(int idx) {
    --cnt[freq[arr[idx]]];
    freq[arr[idx]]--;
    cnt[freq[arr[idx]]]++;
}
void apply(int t) {
    const bool f = l <= Up[t].idx and Up[t].idx <= r;
    if (f) remove(Up[t].idx);
    arr[Up[t].idx] = Up[t].val;
    if (f) add(Up[t].idx);
}
void undo(int t) {
    const bool f = l <= Up[t].idx and Up[t].idx <= r;
    if (f) remove(Up[t].idx);
    arr[Up[t].idx] = Up[t].last;
    if (f) add(Up[t].idx);
}
int mex() {
    for (int i = 1; i <= N; i++)
        if (!cnt[i]) return i;
    assert(0);
}
int main() {
    sort(id + 1, id + qi + 1, [&](int x, int y) { return Q[x] < Q[y]; });
    for (int i = 1; i <= qi; i++) {
        int x = id[i];
        while (Q[x].t > t) apply(++t);
        while (Q[x].t < t) undo(t--);
        while (Q[x].l < l) add(--l);
        while (Q[x].r > r) add(++r);
        while (Q[x].l > l) remove(l++);
        while (Q[x].r < r) remove(r--);
        ans[x] = mex();
    }
}
\end{lstlisting}
\subsection{SparseTable (Rectangle Query)}
\begin{lstlisting}
#include <bits/stdc++.h>
using namespace std;

const int MAXN = 505;
const int LOGN = 9;

// O(n^2 (logn)^2
// Supports Rectangular Query
int A[MAXN][MAXN];
int M[MAXN][MAXN][LOGN][LOGN];

void Build2DSparse(int N) {
    for (int i = 1; i <= N; i++) {
        for (int j = 1; j <= N; j++) {
            M[i][j][0][0] = A[i][j];
        }
        for (int q = 1; (1 << q) <= N; q++) {
            int add = 1 << (q - 1);
            for (int j = 1; j + add <= N; j++) {
                M[i][j][0][q] = max(M[i][j][0][q - 1], M[i][j + add][0][q - 1]);
            }
        }
    }

    for (int p = 1; (1 << p) <= N; p++) {
        int add = 1 << (p - 1);
        for (int i = 1; i + add <= N; i++) {
            for (int q = 0; (1 << q) <= N; q++) {
                for (int j = 1; j <= N; j++) {
                    M[i][j][p][q] = max(M[i][j][p - 1][q], M[i + add][j][p - 1][q]);
                }
            }
        }
    }
}

// returns max of all A[i][j], where x1<=i<=x2 and y1<=j<=y2
int Query(int x1, int y1, int x2, int y2) {
    int kX = log2(x2 - x1 + 1);
    int kY = log2(y2 - y1 + 1);
    int addX = 1 << kX;
    int addY = 1 << kY;

    int ret1 = max(M[x1][y1][kX][kY], M[x1][y2 - addY + 1][kX][kY]);
    int ret2 = max(M[x2 - addX + 1][y1][kX][kY],
                                  M[x2 - addX + 1][y2 - addY + 1][kX][kY]);
    return max(ret1, ret2);
}
\end{lstlisting}
\section{Graph}
\subsection{Graph Template}
\begin{lstlisting}
struct edge {
  int u, v;
  edge(int u = 0, int v = 0) : u(u), v(v) {}
  int to(int node) { return u ^ v ^ node; }
};
struct graph {
  int n;
  vector<vector<int>> adj;
  vector<edge> edges;
  graph(int n = 0) : n(n), adj(n) {}
  void addEdge(int u, int v, bool dir = 1) {
    adj[u].push_back(edges.size());
    if (dir) adj[v].push_back(edges.size());
    edges.emplace_back(u, v);
  }
  int addNode() {
    adj.emplace_back();
    return n++;
  }
  edge &operator()(int idx) { return edges[idx]; }
  vector<int> &operator[](int u) { return adj[u]; }
};
\end{lstlisting}
\subsection{Lifting, LCA, HLD}
\begin{lstlisting}
using Tree = vector<vector<int>>;
int anc[B][N], sz[N], lvl[N], st[N], en[N], nxt[N], t = 0;

void initLifting(int n) {
  for (int b = 1; b < B; b++) {
    for (int i = 0; i < n; i++) {
      anc[b][i] = anc[b - 1][anc[b - 1][i]];
    }
  }
}

int kthAncestor(int u, int k) {
  for (int b = 0; b < B; b++) {
    if (k >> b & 1) u = anc[b][u];
  }
  return u;
}

int lca(int u, int v) {
  if (lvl[u] > lvl[v]) swap(u, v);
  v = kthAncestor(v, lvl[v] - lvl[u]);

  if (u == v) return u;

  for (int b = B - 1; b >= 0; b--) {
    if (anc[b][u] != anc[b][v]) u = anc[b][u], v = anc[b][v];
  }
  return anc[0][u];
}

int dis(int u, int v) {
  int g = lca(u, v);
  return lvl[u] + lvl[v] - 2 * lvl[g];
}
bool isAncestor(int u, int v) { return st[v] <= st[u] and en[u] <= en[v]; }

void tour(int u, int p, Tree &T) {
  st[u] = t++;
  int idx = 0;
  for (int v : T[u]) {
    if (v == p) continue;
    nxt[v] = (idx++ ? v : nxt[u]);  // only for hld
    anc[0][v] = u, lvl[v] = lvl[u] + 1;
    tour(v, u, T);
  }
  en[u] = t;  // [st, en] contains subtree range
}

void hld(int u, int p, Tree &T) {
  sz[u] = 1;

  int eld = 0, mx = 0, idx = 0;
  for (int i = 0; i < T[u].size(); i++) {
    int v = T[u][i];
    if (v == p) continue;
    hld(v, u, T);

    if (sz[v] > mx) mx = sz[v], eld = i;
    sz[u] += sz[v];
  }
  swap(T[u][0], T[u][eld]);
}

LL climbQuery(int u, int g) {
  LL ans = -INF;
  while (1) {
    int _u = nxt[u];
    if (isAncestor(g, _u)) _u = g;
    ans = max(ans, rmq ::query(st[_u], st[u]));

    if (_u == g) break;
    u = anc[0][_u];
  }
  return ans;
}

LL pathQuery(int u, int v) {
  int g = lca(u, v);
  return max(climbQuery(u, g), climbQuery(v, g));
}

void init(int u, Tree &T) {
  int n = T.size();
  anc[0][u] = nxt[u] = u;
  lvl[u] = 0;
  hld(u, u, T);
  tour(u, u, T);
  initLifting(n);
}
\end{lstlisting}
\subsection{SCC}
\begin{lstlisting}
vector<int> order, comp, idx;
vector<bool> vis;
vector<vector<int>> comps;
Graph dag;

void dfs1(int u, Graph &G, string s = "") {
  vis[u] = 1;
  for (int e : G[u]) {
    int v = G(e).to(u);
    if (!vis[v]) dfs1(v, G, s);
  }
  order.push_back(u);
}
void dfs2(int u, Graph &R) {
  comp.push_back(u);
  idx[u] = comps.size();

  for (int e : R[u]) {
    int v = R(e).to(u);
    if (idx[v] == -1) dfs2(v, R);
  }
}

void init(Graph &G) {
  int n = G.n;
  vis.assign(n, 0);
  idx.assign(n, -1);

  for (int i = 0; i < n; i++) {
    if (!vis[i]) dfs1(i, G);
  }
  reverse(order.begin(), order.end());

  Graph R(n);
  for (auto &e : G.edges) R.addEdge(e.v, e.u, 0);

  for (int u : order) {
    if (idx[u] != -1) continue;
    comp.clear();
    dfs2(u, R);
    comps.push_back(comp);
  }
}

Graph &createDAG(Graph &G) {
  int sz = comps.size();
  dag = Graph(sz);

  vector<bool> taken(sz);
  vector<int> cur;

  for (int i = 0; i < sz; i++) {
    cur.clear();
    taken[i] = 1;
    for (int u : comps[i]) {
      for (int e : G[u]) {
        int v = G(e).to(u);
        int j = idx[v];
        if (taken[j]) continue;  // rejects multi-edge
        dag.addEdge(i, j, 0);
        taken[j] = 1;
        cur.push_back(j);
      }
    }
    for (int j : cur) taken[j] = 0;
  }
  return dag;
}
\end{lstlisting}
\subsection{Centroid Decompose}
\begin{lstlisting}
namespace ct {
int par[N], cnt[N], cntp[N];
LL sum[N], sump[N];
void activate(int u) {
  int v = u, _u = u;

  ans += sum[u];
  cnt[u]++;
  while (par[u] != -1) {
    u = par[u];
    LL d = ta ::dis(_u, u);
    ans += sum[u] - sump[v];
    ans += d * (cnt[u] - cntp[v]);

    sum[u] += d;
    cnt[u]++;

    sump[v] += d;
    cntp[v]++;

    v = u;
  }
}
}
namespace ctrd {
int sz[N];
bool blk[N];

int szCalc(Tree &T, int u, int p = -1) {
  sz[u] = 1;
  for (int v : T[u]) {
    if (v == p or blk[v]) continue;
    sz[u] += szCalc(T, v, u);
  }
  return sz[u];
}
int getCentroid(Tree &T, int u, int s, int p = -1) {
  for (int v : T[u]) {
    if (v == p or blk[v]) continue;
    if (sz[v] * 2 >= s) return getCentroid(T, v, s, u);
  }
  return u;
}

void decompose(Tree &T, int u, int p = -1) {
  szCalc(T, u);
  u = getCentroid(T, u, sz[u]);
  ct ::par[u] = p;

  blk[u] = 1;
  for (int v : T[u]) {
    if (!blk[v]) decompose(T, v, u);
  }
}
}
\end{lstlisting}
\subsection{Euler Tour on Edge}
\begin{lstlisting}
// for simplicity, G[idx] contains the adjacency list of a node
// while G(e) is a reference to the e-th edge.
const int N = 2e5 + 5;
int in[N], out[N], fwd[N], bck[N];
int t = 0;
void dfs(graph &G, int node, int par) {
  out[node] = t;
  for (int e : G[node]) {
    int v = G(e).to(node);
    if (v == par) continue;
    fwd[e] = t++;
    dfs(G, v, node);
    bck[e] = t++;
  }
  in[node] = t - 1;
}
void init(graph &G, int node) {
  t = 0;
  dfs(G, node, node);
}
\end{lstlisting}
\subsection{Virtual Tree}
\begin{lstlisting}
namespace lca1 {
int st[N], lvl[N];
int tbl[B][2 * N];
int t = 0;

void dfs(int u, int p, Tree &T) {
  st[u] = t;
  tbl[0][t++] = u;
  for(int v: T[u]) {
    if(v == p) continue;
    lvl[v] = lvl[u] + 1;
    dfs(v, u, T);
    tbl[0][t++] = u;
  }
}
int low(int u, int v) {
  return make_pair(lvl[u], u) < make_pair(lvl[v], v) ? u : v;
}

void makeTable(int n) {
  int m = 2 * n - 1;
  for(int b = 1; b < B; b++) {
    for(int i = 0; i < m; i++) {
      tbl[b][i] = low(tbl[b - 1][i], tbl[b - 1][i + (1 << b - 1)]);
    }
  }
}

int lca(int u, int v) {
  int l = st[u], r = st[v];
  if(l > r) swap(l, r);
  int k = __lg(r - l + 1);
  return low(tbl[k][l], tbl[k][r - (1 << k) + 1]);
}
void init(int root, Tree &T) {
  lvl[root] = 0;
  t = 0;
  dfs(root, root, T);
  makeTable(T.size());
}
}
namespace vt {
int st[N], en[N], t;
vector <int> adj[N];

void dfs(int u, int p, Tree &T) {
  st[u] = t++;
  for(int v: T[u]) if(v != p) dfs(v, u, T);
  en[u] = t++;
}
bool comp(int u, int v) {
  return st[u] < st[v];
}
bool isAncestor(int u, int p) {
  return st[p] <= st[u] and en[u] <= en[p];
}

void construct(vector <int> &nodes) {
  sort(nodes.begin(), nodes.end(), comp);
  int n = nodes.size();
  for(int i = 0; i + 1 < n; i++) {
    nodes.push_back(lca1 :: lca(nodes[i], nodes[i + 1]));
  }
  sort(nodes.begin(), nodes.end(), comp);
  nodes.erase(unique(nodes.begin(), nodes.end()), nodes.end());
  n = nodes.size();
  stack <int> s;
  s.push(nodes[0]);
  for(int i = 1; i < n; i++) {
    int u = nodes[i];
    while(not isAncestor(u, s.top())) s.pop();
    adj[s.top()].push_back(u);
    s.push(u);
  }
}
void clear(vector <int> &nodes) {
  for(int u: nodes) {
      adj[u].clear();
  }
}

void init(int root, Tree &T) {
  lca1 :: init(root, T);
  t = 0;
  dfs(root, root, T);
}
}\end{lstlisting}
\subsection{Dinic Max Flow}
\begin{lstlisting}
/// flow with demand(lower bound) only for DAG
// create new src and sink
// add_edge(new src, u, sum(in_demand[u]))
// add_edge(u, new sink, sum(out_demand[u]))
// add_edge(old sink, old src, inf)
//  if (sum of lower bound == flow) then demand satisfied
// flow in every edge i = demand[i] + e.flow

using Ti = long long;
const Ti INF = 1LL << 60;
struct edge {
  int v, u;
  Ti cap, flow = 0;
  edge(int v, int u, Ti cap) : v(v), u(u), cap(cap) {}
};
const int N = 1e5 + 50;
vector<edge> edges;
vector<int> adj[N];
int m = 0, n;
int level[N], ptr[N];
queue<int> q;
bool bfs(int s, int t) {
  for (q.push(s), level[s] = 0; !q.empty(); q.pop()) {
    for (int id : adj[q.front()]) {
      auto &ed = edges[id];
      if (ed.cap - ed.flow > 0 and level[ed.u] == -1)
        level[ed.u] = level[ed.v] + 1, q.push(ed.u);
    }
  }
  return level[t] != -1;
}
Ti dfs(int v, Ti pushed, int t) {
  if (pushed == 0) return 0;
  if (v == t) return pushed;
  for (int &cid = ptr[v]; cid < adj[v].size(); cid++) {
    int id = adj[v][cid];
    auto &ed = edges[id];
    if (level[v] + 1 != level[ed.u] || ed.cap - ed.flow < 1) continue;
    Ti tr = dfs(ed.u, min(pushed, ed.cap - ed.flow), t);
    if (tr == 0) continue;
    ed.flow += tr;
    edges[id ^ 1].flow -= tr;
    return tr;
  }
  return 0;
}
void init(int nodes) {
  m = 0, n = nodes;
  for (int i = 0; i < n; i++) level[i] = -1, ptr[i] = 0, adj[i].clear();
}
void addEdge(int v, int u, Ti cap) {
  edges.emplace_back(v, u, cap), adj[v].push_back(m++);
  edges.emplace_back(u, v, 0), adj[u].push_back(m++);
}
Ti maxFlow(int s, int t) {
  Ti f = 0;
  for (auto &ed : edges) ed.flow = 0;
  for (; bfs(s, t); memset(level, -1, n * 4)) {
    for (memset(ptr, 0, n * 4); Ti pushed = dfs(s, INF, t); f += pushed)
      ;
  }
  return f;
}
\end{lstlisting}
\subsection{Min Cost Max Flow}
\begin{lstlisting}
mt19937 rnd(chrono::steady_clock::now().time_since_epoch().count());
const LL inf = 1e9;
struct edge {
  int v, rev;
  LL cap, cost, flow;
  edge() {}
  edge(int v, int rev, LL cap, LL cost)
      : v(v), rev(rev), cap(cap), cost(cost), flow(0) {}
};
struct mcmf {
  int src, sink, n;
  vector<int> par, idx, Q;
  vector<bool> inq;
  vector<LL> dis;
  vector<vector<edge>> g;
  mcmf() {}
  mcmf(int src, int sink, int n)
      : src(src),
        sink(sink),
        n(n),
        par(n),
        idx(n),
        inq(n),
        dis(n),
        g(n),
        Q(10000005) {}  // use Q(n) if not using random
  void add_edge(int u, int v, LL cap, LL cost, bool directed = true) {
    edge _u = edge(v, g[v].size(), cap, cost);
    edge _v = edge(u, g[u].size(), 0, -cost);
    g[u].pb(_u);
    g[v].pb(_v);
    if (!directed) add_edge(v, u, cap, cost, true);
  }
  bool spfa() {
    for (int i = 0; i < n; i++) {
      dis[i] = inf, inq[i] = false;
    }
    int f = 0, l = 0;
    dis[src] = 0, par[src] = -1, Q[l++] = src, inq[src] = true;
    while (f < l) {
      int u = Q[f++];
      for (int i = 0; i < g[u].size(); i++) {
        edge &e = g[u][i];
        if (e.cap <= e.flow) continue;
        if (dis[e.v] > dis[u] + e.cost) {
          dis[e.v] = dis[u] + e.cost;
          par[e.v] = u, idx[e.v] = i;
          if (!inq[e.v]) inq[e.v] = true, Q[l++] = e.v;
          // if (!inq[e.v]) {
          //   inq[e.v] = true;
          //   if (f && rnd() & 7) Q[--f] = e.v;
          //   else Q[l++] = e.v;
          // }
        }
      }
      inq[u] = false;
    }
    return (dis[sink] != inf);
  }
  pair<LL, LL> solve() {
    LL mincost = 0, maxflow = 0;
    while (spfa()) {
      LL bottleneck = inf;
      for (int u = par[sink], v = idx[sink]; u != -1; v = idx[u], u = par[u]) {
        edge &e = g[u][v];
        bottleneck = min(bottleneck, e.cap - e.flow);
      }
      for (int u = par[sink], v = idx[sink]; u != -1; v = idx[u], u = par[u]) {
        edge &e = g[u][v];
        e.flow += bottleneck;
        g[e.v][e.rev].flow -= bottleneck;
      }
      mincost += bottleneck * dis[sink], maxflow += bottleneck;
    }
    return make_pair(mincost, maxflow);
  }
};
// want to minimize cost and don't care about flow
// add edge from sink to dummy sink (cap = inf, cost = 0)
// add edge from source to sink (cap = inf, cost = 0)
// run mcmf, cost returned is the minimum cost
\end{lstlisting}
\subsection{Block Cut Tree}
\begin{lstlisting}
vector<vector<int> > components;
vector<int> cutpoints, start, low;
vector<bool> is_cutpoint;
stack<int> st;
void find_cutpoints(int node, graph &G, int par = -1, int d = 0) {
  low[node] = start[node] = d++;
  st.push(node);
  int cnt = 0;
  for (int e : G[node])
    if (int to = G(e).to(node); to != par) {
      if (start[to] == -1) {
        find_cutpoints(to, G, node, d + 1);
        cnt++;
        if (low[to] >= start[node]) {
          is_cutpoint[node] = par != -1 or cnt > 1;
          components.push_back({node});  // starting a new block with the point
          while (st.top() != node)
            components.back().push_back(st.top()), st.pop();
        }
      }
      low[node] = min(low[node], low[to]);
    }
}
graph tree;
vector<int> id;
void init(graph &G) {
  int n = G.n;
  start.assign(n, -1), low.resize(n), is_cutpoint.resize(n), id.assign(n, -1);
  find_cutpoints(0, G);
  for (int u = 0; u < n; ++u)
    if (is_cutpoint[u]) id[u] = tree.addNode();
  for (auto &comp : components) {
    int node = tree.addNode();
    for (int u : comp)
      if (!is_cutpoint[u])
        id[u] = node;
      else
        tree.addEdge(node, id[u]);
  }
  if (id[0] == -1)  // corner - 1
    id[0] = tree.addNode();
}
\end{lstlisting}
\subsection{Bridge Tree}
\begin{lstlisting}
vector<vector<int>> comps;
vector<int> depth, low, id;
stack<int> st;
vector<Edge> bridges;
Graph tree;

void dfs(int u, Graph &G, int ed = -1, int d = 0) {
  low[u] = depth[u] = d;
  st.push(u);
  for (int e : G[u]) {
    if (e == ed) continue;
    int v = G(e).to(u);
    if (depth[v] == -1) dfs(v, G, e, d + 1);
    low[u] = min(low[u], low[v]);

    if (low[v] <= depth[u]) continue;
    bridges.emplace_back(u, v);
    comps.emplace_back();
    do {
      comps.back().push_back(st.top()), st.pop();
    } while (comps.back().back() != v);
  }
  if (ed == -1) {
    comps.emplace_back();
    while (!st.empty()) comps.back().push_back(st.top()), st.pop();
  }
}
Graph &createTree() {
  for (auto &comp : comps) {
    int idx = tree.addNode();
    for (auto &e : comp) id[e] = idx;
  }
  for (auto &[l, r] : bridges) tree.addEdge(id[l], id[r]);
  return tree;
}

void init(Graph &G) {
  int n = G.n;
  depth.assign(n, -1), id.assign(n, -1), low.resize(n);
  for (int i = 0; i < n; i++) {
    if (depth[i] == -1) dfs(i, G);
  }
}
\end{lstlisting}
\subsection{Tree Isomorphism}
\begin{lstlisting}
mp["01"] = 1;
ind = 1;
int dfs(int u, int p) {
  int cnt = 0;
  vector<int> vs;
  for (auto v : g1[u]) {
    if (v != p) {
      int got = dfs(v, u);
      vs.pb(got);
      cnt++;
    }
  }
  if (!cnt) return 1;

  sort(vs.begin(), vs.end());
  string s = "0";
  for (auto i : vs) s += to_string(i);
  vs.clear();
  s.pb('1');
  if (mp.find(s) == mp.end()) mp[s] = ++ind;
  int ret = mp[s];
  return ret;
}
\end{lstlisting}
\section{Math}
\subsection{Combi}
\begin{lstlisting}
array<int, N + 1> fact, inv, inv_fact;
void init() {
  fact[0] = inv_fact[0] = 1;
  for (int i = 1; i <= N; i++) {
    inv[i] = i == 1 ? 1 : (LL)inv[i - mod % i] * (mod / i + 1) % mod;
    fact[i] = (LL)fact[i - 1] * i % mod;
    inv_fact[i] = (LL)inv_fact[i - 1] * inv[i] % mod;
  }
}
LL C(int n, int r) {
  return (r < 0 or r > n) ? 0 : (LL)fact[n] * inv_fact[r] % mod * inv_fact[n - r] % mod;
}
\end{lstlisting}
\subsection{Linear Sieve}
\begin{lstlisting}
const int N = 1e7;
vector<int> primes;
int spf[N + 5], phi[N + 5], NOD[N + 5], cnt[N + 5], POW[N + 5];
bool prime[N + 5];
int SOD[N + 5];
void init() {
  fill(prime + 2, prime + N + 1, 1);
  SOD[1] = NOD[1] = phi[1] = spf[1] = 1;
  for (LL i = 2; i <= N; i++) {
    if (prime[i]) {
      primes.push_back(i), spf[i] = i;
      phi[i] = i - 1;
      NOD[i] = 2, cnt[i] = 1;
      SOD[i] = i + 1, POW[i] = i;
    }
    for (auto p : primes) {
      if (p * i > N or p > spf[i]) break;
      prime[p * i] = false, spf[p * i] = p;
      if (i % p == 0) {
        phi[p * i] = p * phi[i];
        NOD[p * i] = NOD[i] / (cnt[i] + 1) * (cnt[i] + 2),
                cnt[p * i] = cnt[i] + 1;
        SOD[p * i] = SOD[i] / SOD[POW[i]] * (SOD[POW[i]] + p * POW[i]),
                POW[p * i] = p * POW[i];
        break;
      } else {
        phi[p * i] = phi[p] * phi[i];
        NOD[p * i] = NOD[p] * NOD[i], cnt[p * i] = 1;
        SOD[p * i] = SOD[p] * SOD[i], POW[p * i] = p;
      }
    }
  }
}

\end{lstlisting}
\subsection{Pollard Rho}
\begin{lstlisting}
LL mul(LL a, LL b, LL mod) {
    return (__int128)a * b % mod;
    // LL ans = a * b - mod * (LL) (1.L / mod * a * b);
    // return ans + mod * (ans < 0) - mod * (ans >= (LL) mod);
}
LL bigmod(LL num, LL pow, LL mod) {
    LL ans = 1;
    for (; pow > 0; pow >>= 1, num = mul(num, num, mod))
        if (pow & 1) ans = mul(ans, num, mod);
    return ans;
}
bool is_prime(LL n) {
    if (n < 2 or n % 6 % 4 != 1) return (n | 1) == 3;
    LL a[] = {2, 325, 9375, 28178, 450775, 9780504, 1795265022};
    LL s = __builtin_ctzll(n - 1), d = n >> s;
    for (LL x : a) {
        LL p = bigmod(x % n, d, n), i = s;
        for (; p != 1 and p != n - 1 and x % n and i--; p = mul(p, p, n))
            ;
        if (p != n - 1 and i != s) return false;
    }
    return true;
}
LL get_factor(LL n) {
    auto f = [&](LL x) { return mul(x, x, n) + 1; };
    LL x = 0, y = 0, t = 0, prod = 2, i = 2, q;
    for (; t++ % 40 or gcd(prod, n) == 1; x = f(x), y = f(f(y))) {
        (x == y) ? x = i++, y = f(x) : 0;
        prod = (q = mul(prod, max(x, y) - min(x, y), n)) ? q : prod;
    }
    return gcd(prod, n);
}
map<LL, int> factorize(LL n) {
    map<LL, int> res;
    if (n < 2) return res;
    LL small_primes[] = {2,  3,  5,  7,  11, 13, 17, 19, 23, 29, 31, 37, 41,
                                              43, 47, 53, 59, 61, 67, 71, 73, 79, 83, 89, 97};
    for (LL p : small_primes)
        for (; n % p == 0; n /= p, res[p]++)
            ;

    auto _factor = [&](LL n, auto &_factor) {
        if (n == 1) return;
        if (is_prime(n))
            res[n]++;
        else {
            LL x = get_factor(n);
            _factor(x, _factor);
            _factor(n / x, _factor);
        }
    };
    _factor(n, _factor);
    return res;
}
\end{lstlisting}
\subsection{Chinese Remainder Theorem}
\begin{lstlisting}
// given a, b will find solutions for
// ax + by = 1
tuple<LL, LL, LL> EGCD(LL a, LL b) {
  if (b == 0)
    return {1, 0, a};
  else {
    auto [x, y, g] = EGCD(b, a % b);
    return {y, x - a / b * y, g};
  }
}
// given modulo equations, will apply CRT
PLL CRT(vector<PLL> &v) {
  LL V = 0, M = 1;
  for (auto &[v, m] : v) {  // value % mod
    auto [x, y, g] = EGCD(M, m);
    if ((v - V) % g != 0) return {-1, 0};
    V += x * (v - V) / g % (m / g) * M, M *= m / g;
    V = (V % M + M) % M;
  }
  return make_pair(V, M);
}
\end{lstlisting}
\subsection{Mobius Function}
\begin{lstlisting}
const int N = 1e6 + 5;
int mob[N];
void mobius() {
  memset(mob, -1, sizeof mob);
  mob[1] = 1;
  for (int i = 2; i < N; i++)
    if (mob[i]) {
      for (int j = i + i; j < N; j += i) mob[j] -= mob[i];
    }
}

\end{lstlisting}
\subsection{FFT}
\begin{lstlisting}
using CD = complex<double>;
typedef long long LL;
const double PI = acos(-1.0L);

int N;
vector<int> perm;
vector<CD> wp[2];
void precalculate(int n) {
  assert((n & (n - 1)) == 0), N = n;
  perm = vector<int>(N, 0);
  for (int k = 1; k < N; k <<= 1) {
    for (int i = 0; i < k; i++) {
      perm[i] <<= 1;
      perm[i + k] = 1 + perm[i];
    }
  }
  wp[0] = wp[1] = vector<CD>(N);
  for (int i = 0; i < N; i++) {
    wp[0][i] = CD(cos(2 * PI * i / N), sin(2 * PI * i / N));
    wp[1][i] = CD(cos(2 * PI * i / N), -sin(2 * PI * i / N));
  }
}
void fft(vector<CD> &v, bool invert = false) {
  if (v.size() != perm.size()) precalculate(v.size());
  for (int i = 0; i < N; i++)
    if (i < perm[i]) swap(v[i], v[perm[i]]);
  for (int len = 2; len <= N; len *= 2) {
    for (int i = 0, d = N / len; i < N; i += len) {
      for (int j = 0, idx = 0; j < len / 2; j++, idx += d) {
        CD x = v[i + j];
        CD y = wp[invert][idx] * v[i + j + len / 2];
        v[i + j] = x + y;
        v[i + j + len / 2] = x - y;
      }
    }
  }
  if (invert) {
    for (int i = 0; i < N; i++) v[i] /= N;
  }
}
void pairfft(vector<CD> &a, vector<CD> &b, bool invert = false) {
  int N = a.size();
  vector<CD> p(N);
  for (int i = 0; i < N; i++) p[i] = a[i] + b[i] * CD(0, 1);
  fft(p, invert);
  p.push_back(p[0]);
  for (int i = 0; i < N; i++) {
    if (invert) {
      a[i] = CD(p[i].real(), 0);
      b[i] = CD(p[i].imag(), 0);
    } else {
      a[i] = (p[i] + conj(p[N - i])) * CD(0.5, 0);
      b[i] = (p[i] - conj(p[N - i])) * CD(0, -0.5);
    }
  }
}
vector<LL> multiply(const vector<LL> &a, const vector<LL> &b) {
  int n = 1;
  while (n < a.size() + b.size()) n <<= 1;
  vector<CD> fa(a.begin(), a.end()), fb(b.begin(), b.end());
  fa.resize(n);
  fb.resize(n);
  //        fft(fa); fft(fb);
  pairfft(fa, fb);
  for (int i = 0; i < n; i++) fa[i] = fa[i] * fb[i];
  fft(fa, true);
  vector<LL> ans(n);
  for (int i = 0; i < n; i++) ans[i] = round(fa[i].real());
  return ans;
}
const int M = 1e9 + 7, B = sqrt(M) + 1;
vector<LL> anyMod(const vector<LL> &a, const vector<LL> &b) {
  int n = 1;
  while (n < a.size() + b.size()) n <<= 1;
  vector<CD> al(n), ar(n), bl(n), br(n);
  for (int i = 0; i < a.size(); i++) al[i] = a[i] % M / B, ar[i] = a[i] % M % B;
  for (int i = 0; i < b.size(); i++) bl[i] = b[i] % M / B, br[i] = b[i] % M % B;
  pairfft(al, ar);
  pairfft(bl, br);
  //        fft(al); fft(ar); fft(bl); fft(br);
  for (int i = 0; i < n; i++) {
    CD ll = (al[i] * bl[i]), lr = (al[i] * br[i]);
    CD rl = (ar[i] * bl[i]), rr = (ar[i] * br[i]);
    al[i] = ll;
    ar[i] = lr;
    bl[i] = rl;
    br[i] = rr;
  }
  pairfft(al, ar, true);
  pairfft(bl, br, true);
  //        fft(al, true); fft(ar, true); fft(bl, true); fft(br, true);
  vector<LL> ans(n);
  for (int i = 0; i < n; i++) {
    LL right = round(br[i].real()), left = round(al[i].real());
    ;
    LL mid = round(round(bl[i].real()) + round(ar[i].real()));
    ans[i] = ((left % M) * B * B + (mid % M) * B + right) % M;
  }
  return ans;
}
\end{lstlisting}
\subsection{NTT}
\begin{lstlisting}
const LL N = 1 << 18;
const LL MOD = 786433;

vector<LL> P[N];
LL rev[N], w[N | 1], a[N], b[N], inv_n, g;
LL Pow(LL b, LL p) {
  LL ret = 1;
  while (p) {
    if (p & 1) ret = (ret * b) % MOD;
    b = (b * b) % MOD;
    p >>= 1;
  }
  return ret;
}
LL primitive_root(LL p) {
  vector<LL> factor;
  LL phi = p - 1, n = phi;
  for (LL i = 2; i * i <= n; i++) {
    if (n % i) continue;
    factor.emplace_back(i);
    while (n % i == 0) n /= i;
  }
  if (n > 1) factor.emplace_back(n);
  for (LL res = 2; res <= p; res++) {
    bool ok = true;
    for (LL i = 0; i < factor.size() && ok; i++)
      ok &= Pow(res, phi / factor[i]) != 1;
    if (ok) return res;
  }
  return -1;
}
void prepare(LL n) {
  LL sz = abs(31 - __builtin_clz(n));
  LL r = Pow(g, (MOD - 1) / n);
  inv_n = Pow(n, MOD - 2);
  w[0] = w[n] = 1;
  for (LL i = 1; i < n; i++) w[i] = (w[i - 1] * r) % MOD;
  for (LL i = 1; i < n; i++)
    rev[i] = (rev[i >> 1] >> 1) | ((i & 1) << (sz - 1));
}
void NTT(LL *a, LL n, LL dir = 0) {
  for (LL i = 1; i < n - 1; i++)
    if (i < rev[i]) swap(a[i], a[rev[i]]);
  for (LL m = 2; m <= n; m <<= 1) {
    for (LL i = 0; i < n; i += m) {
      for (LL j = 0; j < (m >> 1); j++) {
        LL &u = a[i + j], &v = a[i + j + (m >> 1)];
        LL t = v * w[dir ? n - n / m * j : n / m * j] % MOD;
        v = u - t < 0 ? u - t + MOD : u - t;
        u = u + t >= MOD ? u + t - MOD : u + t;
      }
    }
  }
  if (dir)
    for (LL i = 0; i < n; i++) a[i] = (inv_n * a[i]) % MOD;
}
vector<LL> mul(vector<LL> p, vector<LL> q) {
  LL n = p.size(), m = q.size();
  LL t = n + m - 1, sz = 1;
  while (sz < t) sz <<= 1;
  prepare(sz);

  for (LL i = 0; i < n; i++) a[i] = p[i];
  for (LL i = 0; i < m; i++) b[i] = q[i];
  for (LL i = n; i < sz; i++) a[i] = 0;
  for (LL i = m; i < sz; i++) b[i] = 0;

  NTT(a, sz);
  NTT(b, sz);
  for (LL i = 0; i < sz; i++) a[i] = (a[i] * b[i]) % MOD;
  NTT(a, sz, 1);

  vector<LL> c(a, a + sz);
  while (c.size() && c.back() == 0) c.pop_back();
  return c;
}
\end{lstlisting}
\subsection{WalshHadamard}
\begin{lstlisting}
#include <bits/stdc++.h>
using namespace std;
typedef long long LL;
#define bitwiseXOR 1
// #define bitwiseAND 2
// #define bitwiseOR 3
const LL MOD = 30011;

LL BigMod(LL b, LL p) {
  LL ret = 1;
  while (p > 0) {
    if (p % 2 == 1) {
      ret = (ret * b) % MOD;
    }
    p = p / 2;
    b = (b * b) % MOD;
  }
  return ret % MOD;
}

void FWHT(vector<LL>& p, bool inverse) {
  LL n = p.size();
  assert((n & (n - 1)) == 0);

  for (LL len = 1; 2 * len <= n; len <<= 1) {
    for (LL i = 0; i < n; i += len + len) {
      for (LL j = 0; j < len; j++) {
        LL u = p[i + j];
        LL v = p[i + len + j];

#ifdef bitwiseXOR
        p[i + j] = (u + v) % MOD;
        p[i + len + j] = (u - v + MOD) % MOD;
#endif  // bitwiseXOR

#ifdef bitwiseAND
        if (!inverse) {
          p[i + j] = v % MOD;
          p[i + len + j] = (u + v) % MOD;
        } else {
          p[i + j] = (-u + v) % MOD;
          p[i + len + j] = u % MOD;
        }
#endif  // bitwiseAND

#ifdef bitwiseOR
        if (!inverse) {
          p[i + j] = u + v;
          p[i + len + j] = u;
        } else {
          p[i + j] = v;
          p[i + len + j] = u - v;
        }
#endif  // bitwiseOR
      }
    }
  }

#ifdef bitwiseXOR
  if (inverse) {
    LL val = BigMod(n, MOD - 2);  // Option 2: Exclude
    for (LL i = 0; i < n; i++) {
      // assert(p[i]%n==0); //Option 2: Include
      p[i] = (p[i] * val) % MOD;  // Option 2: p[i]/=n;
    }
  }
#endif  // bitwiseXOR
}
\end{lstlisting}
\subsection{Berlekamp Massey}
\begin{lstlisting}
struct berlekamp_massey { // for linear recursion
  typedef long long LL;
  static const int SZ = 2e5 + 5;
  static const int MOD = 1e9 + 7; /// mod must be a prime
  LL m , a[SZ] , h[SZ] , t_[SZ] , s[SZ] , t[SZ];
  // bigmod goes here
  inline vector <LL> BM( vector <LL> &x ) {
    LL lf , ld;
    vector <LL> ls , cur;
    for ( int i = 0; i < int(x.size()); ++i ) {
      LL t = 0;
      for ( int j = 0; j < int(cur.size()); ++j ) t = (t + x[i - j - 1] * cur[j]) % MOD;
      if ( (t - x[i]) % MOD == 0 ) continue;
      if ( !cur.size() ) {
        cur.resize( i + 1 );
        lf = i; ld = (t - x[i]) % MOD;
        continue;
      }
      LL k = -(x[i] - t) * bigmod( ld , MOD - 2 , MOD ) % MOD;
      vector <LL> c(i - lf - 1);
      c.push_back( k );
      for ( int j = 0; j < int(ls.size()); ++j ) c.push_back(-ls[j] * k % MOD);
      if ( c.size() < cur.size() ) c.resize( cur.size() );
      for ( int j = 0; j < int(cur.size()); ++j ) c[j] = (c[j] + cur[j]) % MOD;
      if (i - lf + (int)ls.size() >= (int)cur.size() ) ls = cur, lf = i, ld = (t - x[i]) % MOD;
      cur = c;
    }
    for ( int i = 0; i < int(cur.size()); ++i ) cur[i] = (cur[i] % MOD + MOD) % MOD;
    return cur;
  }
  inline void mull( LL *p , LL *q ) {
    for ( int i = 0; i < m + m; ++i ) t_[i] = 0;
    for ( int i = 0; i < m; ++i ) if ( p[i] )
        for ( int j = 0; j < m; ++j ) t_[i + j] = (t_[i + j] + p[i] * q[j]) % MOD;
    for ( int i = m + m - 1; i >= m; --i ) if ( t_[i] )
        for ( int j = m - 1; ~j; --j ) t_[i - j - 1] = (t_[i - j - 1] + t_[i] * h[j]) % MOD;
    for ( int i = 0; i < m; ++i ) p[i] = t_[i];
  }
  inline LL calc( LL K ) {
    for ( int i = m; ~i; --i ) s[i] = t[i] = 0;
    s[0] = 1; if ( m != 1 ) t[1] = 1; else t[0] = h[0];
    while ( K ) {
      if ( K & 1 ) mull( s , t );
      mull( t , t ); K >>= 1;
    }
    LL su = 0;
    for ( int i = 0; i < m; ++i ) su = (su + s[i] * a[i]) % MOD;
    return (su % MOD + MOD) % MOD;
  }
  /// already calculated upto k , now calculate upto n.
  inline vector <LL> process( vector <LL> &x , int n , int k ) {
    auto re = BM( x );
    x.resize( n + 1 );
    for ( int i = k + 1; i <= n; i++ ) {
      for ( int j = 0; j < re.size(); j++ ) {
        x[i] += 1LL * x[i - j - 1] % MOD * re[j] % MOD; x[i] %= MOD;
      }
    }
    return x;
  }
  inline LL work( vector <LL> &x , LL n ) {
    if ( n < int(x.size()) ) return x[n] % MOD;
    vector <LL> v = BM( x ); m = v.size(); if ( !m ) return 0;
    for ( int i = 0; i < m; ++i ) h[i] = v[i], a[i] = x[i];
    return calc( n ) % MOD;
  }
} rec;
vector <LL> v;
void solve() {
  int n;
  cin >> n;
  cout << rec.work(v, n - 1) << endl;
}

\end{lstlisting}
\subsection{Lagrange}
\begin{lstlisting}
// p is a polynomial with n points.
// p(0), p(1), p(2), ... p(n-1) are given.
// Find p(x).
LL Lagrange(vector<LL> &p, LL x) {
  LL n = p.size(), L, i, ret;
  if (x < n) return p[x];
  L = 1;
  for (i = 1; i < n; i++) {
    L = (L * (x - i)) % MOD;
    L = (L * bigmod(MOD - i, MOD - 2)) % MOD;
  }
  ret = (L * p[0]) % MOD;
  for (i = 1; i < n; i++) {
    L = (L * (x - i + 1)) % MOD;
    L = (L * bigmod(x - i, MOD - 2)) % MOD;
    L = (L * bigmod(i, MOD - 2)) % MOD;
    L = (L * (MOD + i - n)) % MOD;
    ret = (ret + L * p[i]) % MOD;
  }
  return ret;
}
\end{lstlisting}
\subsection{Shanks' Baby Step, Giant Step}
\begin{lstlisting}
// Finds a^x = b (mod p)

LL bigmod(LL b, LL p, LL m) {}

LL babyStepGiantStep(LL a, LL b, LL p) {
  LL i, j, c, sq = sqrt(p);
  map<LL, LL> babyTable;

  for (j = 0, c = 1; j <= sq; j++, c = (c * a) % p) babyTable[c] = j;

  LL giant = bigmod(a, sq * (p - 2), p);

  for (i = 0, c = 1; i <= sq; i++, c = (c * giant) % p) {
    if (babyTable.find((c * b) % p) != babyTable.end())
      return i * sq + babyTable[(c * b) % p];
  }

  return -1;
}
\end{lstlisting}
\subsection{Xor Basis}
\begin{lstlisting}
struct XorBasis {
  static const int sz = 64;
  array<ULL, sz> base = {0}, back;
  array<int, sz> pos;
  void insert(ULL x, int p) {
    ULL cur = 0;
    for (int i = sz - 1; ~i; i--)
      if (x >> i & 1) {
        if (!base[i]) {
          base[i] = x, back[i] = cur, pos[i] = p;
          break;
        } else x ^= base[i], cur |= 1ULL << i;
      }
  }
  pair<ULL, vector<int>> construct(ULL mask) {
    ULL ok = 0, x = mask;
    for (int i = sz - 1; ~i; i--)
      if (mask >> i & 1 and base[i]) mask ^= base[i], ok |= 1ULL << i;
    vector<int> ans;
    for (int i = 0; i < sz; i++)
      if (ok >> i & 1) {
        ans.push_back(pos[i]);
        ok ^= back[i];
      }
    return {x ^ mask, ans};
  }
};
\end{lstlisting}
\section{String}
\subsection{Aho Corasick}
\begin{lstlisting}
struct AC {
int N, P;
const int A = 26;
vector<vector<int>> next;
vector<int> link, out_link;
vector<vector<int>> out;
AC() : N(0), P(0) { node(); }
int node() {
  next.emplace_back(A, 0);
  link.emplace_back(0);
  out_link.emplace_back(0);
  out.emplace_back(0);
  return N++;
}
inline int get(char c) { return c - 'a'; }
int add_pattern(const string T) {
  int u = 0;
  for (auto c : T) {
    if (!next[u][get(c)]) next[u][get(c)] = node();
    u = next[u][get(c)];
  }
  out[u].push_back(P);
  return P++;
}
void compute() {
  queue<int> q;
  for (q.push(0); !q.empty();) {
    int u = q.front(); q.pop();
    for (int c = 0; c < A; ++c) {
      int v = next[u][c];
      if (!v) next[u][c] = next[link[u]][c];
      else {
        link[v] = u ? next[link[u]][c] : 0;
        out_link[v] = out[link[v]].empty() ? out_link[link[v]] : link[v];
        q.push(v);
      }
    }
  }
}
int advance(int u, char c) {
  while (u && !next[u][get(c)]) u = link[u];
  u = next[u][get(c)];
  return u;
}
void match(const string S) {
  int u = 0;
  for (auto c : S) {
    u = advance(u, c);
    for (int v = u; v; v = out_link[v]) {
      for (auto p : out[v]) cout << "match " << p << endl;
    }
  }
}
};
int main() {
  AC aho; int n; cin >> n;
  while (n--) {
    string s; cin >> s;
    aho.add_pattern(s);
  }
  aho.compute(); string text;
  cin >> text; aho.match(text);
  return 0;
}
\end{lstlisting}
\subsection{Double hash}
\begin{lstlisting}
// define +, -, * for (PLL, LL) and (PLL, PLL), % for (PLL, PLL);
PLL base(1949313259, 1997293877);
PLL mod(2091573227, 2117566807);

PLL power(PLL a, LL p) {
  PLL ans = PLL(1, 1);
  for(; p; p >>= 1, a = a * a % mod) {
      if(p & 1) ans = ans * a % mod;
  }
  return ans;
}

PLL inverse(PLL a) { return power(a, (mod.ff - 1) * (mod.ss - 1) - 1); }
PLL inv_base = inverse(base);
PLL val;
vector<PLL> P;

void hash_init(int n) {
  P.resize(n + 1);
  P[0] = PLL(1, 1);
  for (int i = 1; i <= n; i++) P[i] = (P[i - 1] * base) % mod;
}
PLL append(PLL cur, char c) { return (cur * base + c) % mod; }
/// prepends c to string with size k
PLL prepend(PLL cur, int k, char c) { return (P[k] * c + cur) % mod; }
/// replaces the i-th (0-indexed) character from right from a to b;
PLL replace(PLL cur, int i, char a, char b) {
  cur = (cur + P[i] * (b - a)) % mod;
  return (cur + mod) % mod;
}
/// Erases c from the back of the string
PLL pop_back(PLL hash, char c) {
  return (((hash - c) * inv_base) % mod + mod) % mod;
}
/// Erases c from front of the string with size len
PLL pop_front(PLL hash, int len, char c) {
  return ((hash - P[len - 1] * c) % mod + mod) % mod;
}
/// concatenates two strings where length of the right is k
PLL concat(PLL left, PLL right, int k) { return (left * P[k] + right) % mod; }
/// Calculates hash of string with size len repeated cnt times
/// This is O(log n). For O(1), pre-calculate inverses
PLL repeat(PLL hash, int len, LL cnt) {
  PLL mul = (P[len * cnt] - 1) * inverse(P[len] - 1);
  mul = (mul % mod + mod) % mod;
  PLL ret = (hash * mul) % mod;
  if (P[len].ff == 1) ret.ff = hash.ff * cnt;
  if (P[len].ss == 1) ret.ss = hash.ss * cnt;
  return ret;
}
LL get(PLL hash) { return ((hash.ff << 32) ^ hash.ss); }
struct hashlist {
  int len;
  vector<PLL> H, R;
  hashlist() {}
  hashlist(string& s) {
    len = (int)s.size();
    hash_init(len);
    H.resize(len + 1, PLL(0, 0)), R.resize(len + 2, PLL(0, 0));
    for (int i = 1; i <= len; i++) H[i] = append(H[i - 1], s[i - 1]);
    for (int i = len; i >= 1; i--) R[i] = append(R[i + 1], s[i - 1]);
  }
  /// 1-indexed
  PLL range_hash(int l, int r) {
    return ((H[r] - H[l - 1] * P[r - l + 1]) % mod + mod) % mod;
  }
  PLL reverse_hash(int l, int r) {
    return ((R[l] - R[r + 1] * P[r - l + 1]) % mod + mod) % mod;
  }
  PLL concat_range_hash(int l1, int r1, int l2, int r2) {
    return concat(range_hash(l1, r1), range_hash(l2, r2), r2 - l2 + 1);
  }
  PLL concat_reverse_hash(int l1, int r1, int l2, int r2) {
    return concat(reverse_hash(l2, r2), reverse_hash(l1, r1), r1 - l1 + 1);
  }
};
\end{lstlisting}
\subsection{Manacher's}
\begin{lstlisting}
vector<int> d1(n);
// d[i] = number of palindromes taking s[i] as center
for (int i = 0, l = 0, r = -1; i < n; i++) {
  int k = (i > r) ? 1 : min(d1[l + r - i], r - i + 1);
  while (0 <= i - k && i + k < n && s[i - k] == s[i + k]) k++;
  d1[i] = k--;
  if (i + k > r) l = i - k, r = i + k;
}
vector<int> d2(n);
// d[i] = number of palindromes taking s[i-1] and s[i] as center
for (int i = 0, l = 0, r = -1; i < n; i++) {
  int k = (i > r) ? 0 : min(d2[l + r - i + 1], r - i + 1);
  while (0 <= i - k - 1 && i + k < n && s[i - k - 1] == s[i + k]) k++;
  d2[i] = k--;
  if (i + k > r) l = i - k - 1, r = i + k;
}
\end{lstlisting}
\subsection{Suffix Array}
\begin{lstlisting}
vector<VI> c;
VI sort_cyclic_shifts(const string &s) {
  int n = s.size();
  const int alphabet = 256;
  VI p(n), cnt(alphabet, 0);

  c.clear();
  c.emplace_back();
  c[0].resize(n);

  for (int i = 0; i < n; i++) cnt[s[i]]++;
  for (int i = 1; i < alphabet; i++) cnt[i] += cnt[i - 1];
  for (int i = 0; i < n; i++) p[--cnt[s[i]]] = i;

  c[0][p[0]] = 0;
  int classes = 1;

  for (int i = 1; i < n; i++) {
    if (s[p[i]] != s[p[i - 1]]) classes++;
    c[0][p[i]] = classes - 1;
  }

  VI pn(n), cn(n);
  cnt.resize(n);
  for (int h = 0; (1 << h) < n; h++) {
    for (int i = 0; i < n; i++) {
      pn[i] = p[i] - (1 << h);
      if (pn[i] < 0) pn[i] += n;
    }
    fill(cnt.begin(), cnt.end(), 0);
    /// radix sort
    for (int i = 0; i < n; i++) cnt[c[h][pn[i]]]++;
    for (int i = 1; i < classes; i++) cnt[i] += cnt[i - 1];
    for (int i = n - 1; i >= 0; i--) p[--cnt[c[h][pn[i]]]] = pn[i];

    cn[p[0]] = 0;
    classes = 1;

    for (int i = 1; i < n; i++) {
      PII cur = {c[h][p[i]], c[h][(p[i] + (1 << h)) % n]};
      PII prev = {c[h][p[i - 1]], c[h][(p[i - 1] + (1 << h)) % n]};
      if (cur != prev) ++classes;
      cn[p[i]] = classes - 1;
    }
    c.push_back(cn);
  }
  return p;
}
VI suffix_array_construction(string s) {
  s += "!";
  VI sorted_shifts = sort_cyclic_shifts(s);
  sorted_shifts.erase(sorted_shifts.begin());
  return sorted_shifts;
}
/// LCP between the ith and jth (i != j) suffix of the STRING
int suffixLCP(int i, int j) {
  assert(i != j);
  int log_n = c.size() - 1;

  int ans = 0;
  for (int k = log_n; k >= 0; k--) {
    if (c[k][i] == c[k][j]) {
      ans += 1 << k;
      i += 1 << k;
      j += 1 << k;
    }
  }
  return ans;
}

VI lcp_construction(const string &s, const VI &sa) {
  int n = s.size();
  VI rank(n, 0);
  VI lcp(n - 1, 0);

  for (int i = 0; i < n; i++) rank[sa[i]] = i;

  for (int i = 0, k = 0; i < n; i++, k -= (k != 0)) {
    if (rank[i] == n - 1) {
      k = 0;
      continue;
    }
    int j = sa[rank[i] + 1];
    while (i + k < n && j + k < n && s[i + k] == s[j + k]) k++;
    lcp[rank[i]] = k;
  }
  return lcp;
}
\end{lstlisting}
\subsection{Z Algo}
\begin{lstlisting}
vector<int> calcz(string s) {
  int n = s.size();
  vector<int> z(n);
  int l = 0, r = 0;
  for (int i = 1; i < n; i++) {
    if (i > r) {
      l = r = i;
      while (r < n && s[r] == s[r - l]) r++;
      z[i] = r - l, r--;
    } else {
      int k = i - l;
      if (z[k] < r - i + 1) z[i] = z[k];
      else {
        l = i;
        while (r < n && s[r] == s[r - l]) r++;
        z[i] = r - l, r--;
      }
    }
  }
  return z;
}
\end{lstlisting}
\section{DP}
\subsection{1D-1D}
\begin{lstlisting}
/// Author: anachor

#include <bits/stdc++.h>
using namespace std;

/// Solves dp[i] = min(dp[j] + cost(j+1, i)) given that cost() is QF
long long solve1D(int n, long long cost(int, int)) {
  vector<long long> dp(n + 1), opt(n + 1);
  deque<pair<int, int>> dq;
  dq.push_back({0, 1});
  dp[0] = 0;

  for (int i = 1; i <= n; i++) {
    opt[i] = dq.front().first;
    dp[i] = dp[opt[i]] + cost(opt[i] + 1, i);
    if (i == n) break;

    dq[0].second++;
    if (dq.size() > 1 && dq[0].second == dq[1].second) dq.pop_front();

    int en = n;
    while (dq.size()) {
      int o = dq.back().first, st = dq.back().second;
      if (dp[o] + cost(o + 1, st) >= dp[i] + cost(i + 1, st))
        dq.pop_back();
      else {
        int lo = st, hi = en;
        while (lo < hi) {
          int mid = (lo + hi + 1) / 2;
          if (dp[o] + cost(o + 1, mid) < dp[i] + cost(i + 1, mid))
            lo = mid;
          else
            hi = mid - 1;
        }
        if (lo < n) dq.push_back({i, lo + 1});
        break;
      }
      en = st - 1;
    }
    if (dq.empty()) dq.push_back({i, i + 1});
  }
  return dp[n];
}

/// Solves https://open.kattis.com/problems/coveredwalkway
const int N = 1e6 + 7;
long long x[N];
int c;
long long cost(int l, int r) { return (x[r] - x[l]) * (x[r] - x[l]) + c; }

int main() {
  ios::sync_with_stdio(false);
  cin.tie(0);

  int n;
  cin >> n >> c;
  for (int i = 1; i <= n; i++) cin >> x[i];
  cout << solve1D(n, cost) << endl;
}
\end{lstlisting}
\subsection{Convex Hull Trick}
\begin{lstlisting}
struct line {
  ll m, c;
  line() {}
  line(ll m, ll c) : m(m), c(c) {}
};
struct convex_hull_trick {
  vector<line> lines;
  int ptr = 0;
  convex_hull_trick() {}
  bool bad(line a, line b, line c) {
    return 1.0 * (c.c - a.c) * (a.m - b.m) < 1.0 * (b.c - a.c) * (a.m - c.m);
  }
  void add(line L) {
    int sz = lines.size();
    while (sz >= 2 && bad(lines[sz - 2], lines[sz - 1], L)) {
      lines.pop_back();
      sz--;
    }
    lines.pb(L);
  }
  ll get(int idx, int x) { return (1ll * lines[idx].m * x + lines[idx].c); }
  ll query(int x) {
    if (lines.empty()) return 0;
    if (ptr >= lines.size()) ptr = lines.size() - 1;
    while (ptr < lines.size() - 1 && get(ptr, x) > get(ptr + 1, x)) ptr++;
    return get(ptr, x);
  }
};
ll sum[MAX];
ll dp[MAX];
int arr[MAX];
int main() {
  fastio;
  int t;
  cin >> t;
  while (t--) {
    int n, a, b, c;
    cin >> n >> a >> b >> c;
    for (int i = 1; i <= n; i++) cin >> sum[i];
    for (int i = 1; i <= n; i++) dp[i] = 0, sum[i] += sum[i - 1];
    convex_hull_trick cht;
    cht.add(line(0, 0));
    for (int pos = 1; pos <= n; pos++) {
      dp[pos] = cht.query(sum[pos]) - 1ll * a * sqr(sum[pos]) - c;
      cht.add(line(2ll * a * sum[pos], dp[pos] - a * sqr(sum[pos])));
    }
    ll ans = (-1ll * dp[n]);
    ans += (1ll * sum[n] * b);
    cout << ans << "\n";
  }
}
\end{lstlisting}
\subsection{Divide and Conquer dp}
\begin{lstlisting}
const int K = 805, N = 4005;
LL dp[2][N], _cost[N][N];
// 1-indexed for convenience
LL cost(int l, int r) {
  return _cost[r][r] - _cost[l - 1][r] - _cost[r][l - 1] + _cost[l - 1][l - 1] >> 1;
}
void compute(int cnt, int l, int r, int optl, int optr) {
  if (l > r) return;
  int mid = l + r >> 1;
  LL best = INT_MAX;
  int opt = -1;
  for (int i = optl; i <= min(mid, optr); i++) {
    LL cur = dp[cnt ^ 1][i - 1] + cost(i, mid);
    if (cur < best) best = cur, opt = i;
  }
  dp[cnt][mid] = best;
  compute(cnt, l, mid - 1, optl, opt);
  compute(cnt, mid + 1, r, opt, optr);
}
LL dnc_dp(int k, int n) {
  fill(dp[0] + 1, dp[0] + n + 1, INT_MAX);
  for (int cnt = 1; cnt <= k; cnt++) {
    compute(cnt & 1, 1, n, 1, n);
  }
  return dp[k & 1][n];
}
\end{lstlisting}
\subsection{Dynamic CHT}
\begin{lstlisting}
typedef long long LL;

const LL IS_QUERY = -(1LL << 62);

struct line {
  LL m, b;
  mutable function <const line*()> succ;

  bool operator < (const line &rhs) const {
    if (rhs.b != IS_QUERY) return m < rhs.m;
    const line *s = succ();
    if (!s) return 0;
    LL x = rhs.m;
    return b - s -> b < (s -> m - m) * x;
  }
};

struct HullDynamic : public multiset <line> { 
  bool bad (iterator y) {
    auto z = next(y);
    if (y == begin()) {
      if (z == end()) return 0;
      return y -> m == z -> m && y -> b <= z -> b;
    }
    auto x = prev(y);
    if (z == end()) return y -> m == x -> m && y -> b <= x -> b;
    return 1.0 * (x -> b - y -> b) * (z -> m - y -> m) >= 1.0 * (y -> b - z -> b) * (y -> m - x -> m);
  }

  void insert_line (LL m, LL b) {
    auto y = insert({m, b});
    y -> succ = [=] {return next(y) == end() ? 0 : &*next(y);};
    if (bad(y)) {erase(y); return;}
    while (next(y) != end() && bad(next(y))) erase(next(y));
    while (y != begin() && bad(prev(y))) erase(prev(y));
  }

  LL eval (LL x) {
    auto l = *lower_bound((line) {x, IS_QUERY});
    return l.m * x + l.b;
  }
};
\end{lstlisting}
\subsection{FFT Online}
\begin{lstlisting}
void fftOnline(vector <LL> &a, vector <LL> b) {
  int n = a.size();
  auto call = [&](int l, int r, auto &call){
    if(l >= r) return;
    int mid = l + r >> 1;
    call(l, mid, call);

    vector <LL> _a(a.begin() + l, a.begin() + mid + 1);
    vector <LL> _b(b.begin(), b.begin() + (r - l + 1));
    auto c = fft :: anyMod(_a, _b);

    for(int i = mid + 1; i <= r; i++) {
      a[i] += c[i - l];
      a[i] -= (a[i] >= mod) * mod;
    }
    call(mid + 1, r, call);
  };
  call(0, n - 1, call);
}

\end{lstlisting}
\subsection{Knuth optimization}
\begin{lstlisting}
const int N = 1005;
LL dp[N][N], a[N];
int opt[N][N];
LL cost(int i, int j) { return a[j + 1] - a[i]; }
LL knuth_optimization(int n) {
  for (int i = 0; i < n; i++) {
    dp[i][i] = 0;
    opt[i][i] = i;
  }
  for (int i = n - 2; i >= 0; i--) {
    for (int j = i + 1; j < n; j++) {
      LL mn = LLONG_MAX;
      LL c = cost(i, j);
      for (int k = opt[i][j - 1]; k <= min(j - 1, opt[i + 1][j]); k++) {
        if (mn > dp[i][k] + dp[k + 1][j] + c) {
          mn = dp[i][k] + dp[k + 1][j] + c;
          opt[i][j] = k;
        }
      }
      dp[i][j] = mn;
    }
  }
  return dp[0][n - 1];
}
\end{lstlisting}
\subsection{Li Chao Tree}
\begin{lstlisting}
struct line {
  LL m, c;
  line(LL m = 0, LL c = 0) : m(m), c(c) {}
};
LL calc(line L, LL x) { return 1LL * L.m * x + L.c; }
struct node {
  LL m, c;
  line L;
  node *lft, *rt;
  node(LL m = 0, LL c = 0, node *lft = NULL, node *rt = NULL)
      : L(line(m, c)), lft(lft), rt(rt) {}
};
struct LiChao {
  node *root;
  LiChao() { root = new node(); }
  void update(node *now, int L, int R, line newline) {
    int mid = L + (R - L) / 2;
    line lo = now->L, hi = newline;
    if (calc(lo, L) > calc(hi, L)) swap(lo, hi);
    if (calc(lo, R) <= calc(hi, R)) {
      now->L = hi;
      return;
    }
    if (calc(lo, mid) < calc(hi, mid)) {
      now->L = hi;
      if (now->rt == NULL) now->rt = new node();
      update(now->rt, mid + 1, R, lo);
    } else {
      now->L = lo;
      if (now->lft == NULL) now->lft = new node();
      update(now->lft, L, mid, hi);
    }
  }
  LL query(node *now, int L, int R, LL x) {
    if (now == NULL) return -inf;
    int mid = L + (R - L) / 2;
    if (x <= mid)
      return max(calc(now->L, x), query(now->lft, L, mid, x));
    else
      return max(calc(now->L, x), query(now->rt, mid + 1, R, x));
  }
};
\end{lstlisting}
\section{Geometry}
\subsection{Point}
\begin{lstlisting}
typedef double Tf;
typedef double Ti;  /// use long long for exactness
const Tf PI = acos(-1), EPS = 1e-9;
int dcmp(Tf x) { return abs(x) < EPS ? 0 : (x < 0 ? -1 : 1); }

struct Point {
    Ti x, y;
    Point(Ti x = 0, Ti y = 0) : x(x), y(y) {}

    Point operator+(const Point& u) const { return Point(x + u.x, y + u.y); }
    Point operator-(const Point& u) const { return Point(x - u.x, y - u.y); }
    Point operator*(const LL u) const { return Point(x * u, y * u); }
    Point operator*(const Tf u) const { return Point(x * u, y * u); }
    Point operator/(const Tf u) const { return Point(x / u, y / u); }

    bool operator==(const Point& u) const {
        return dcmp(x - u.x) == 0 && dcmp(y - u.y) == 0;
    }
    bool operator!=(const Point& u) const { return !(*this == u); }
    bool operator<(const Point& u) const {
        return dcmp(x - u.x) < 0 || (dcmp(x - u.x) == 0 && dcmp(y - u.y) < 0);
    }
};
Ti dot(Point a, Point b) { return a.x * b.x + a.y * b.y; }
Ti cross(Point a, Point b) { return a.x * b.y - a.y * b.x; }
Tf length(Point a) { return sqrt(dot(a, a)); }
Ti sqLength(Point a) { return dot(a, a); }
Tf distance(Point a, Point b) { return length(a - b); }
Tf angle(Point u) { return atan2(u.y, u.x); }

// returns angle between oa, ob in (-PI, PI]
Tf angleBetween(Point a, Point b) {
    Tf ans = angle(b) - angle(a);
    return ans <= -PI ? ans + 2 * PI : (ans > PI ? ans - 2 * PI : ans);
}
// Rotate a ccw by rad radians, Tf Ti same
Point rotate(Point a, Tf rad) {
    return Point(a.x * cos(rad) - a.y * sin(rad),
                              a.x * sin(rad) + a.y * cos(rad));
}
// rotate a ccw by angle th with cos(th) = co && sin(th) = si, tf ti same
Point rotatePrecise(Point a, Tf co, Tf si) {
    return Point(a.x * co - a.y * si, a.y * co + a.x * si);
}
Point rotate90(Point a) { return Point(-a.y, a.x); }
// scales vector a by s such that length of a becomes s, Tf Ti same
Point scale(Point a, Tf s) { return a / length(a) * s; }
// returns an unit vector perpendicular to vector a, Tf Ti same
Point normal(Point a) {
    Tf l = length(a);
    return Point(-a.y / l, a.x / l);
}
// returns 1 if c is left of ab, 0 if on ab && -1 if right of ab
int orient(Point a, Point b, Point c) { return dcmp(cross(b - a, c - a)); }
/// Use as sort(v.begin(), v.end(), polarComp(O, dir))
/// Polar comparator around O starting at direction dir
struct polarComp {
    Point O, dir;
    polarComp(Point O = Point(0, 0), Point dir = Point(1, 0)) : O(O), dir(dir) {}
    bool half(Point p) {
        return dcmp(cross(dir, p)) < 0 ||
                      (dcmp(cross(dir, p)) == 0 && dcmp(dot(dir, p)) > 0);
    }
    bool operator()(Point p, Point q) {
        return make_tuple(half(p), 0) < make_tuple(half(q), cross(p, q));
    }
};
struct Segment {
    Point a, b;
    Segment(Point aa, Point bb) : a(aa), b(bb) {}
};
typedef Segment Line;
struct Circle {
    Point o;
    Tf r;
    Circle(Point o = Point(0, 0), Tf r = 0) : o(o), r(r) {}
    // returns true if point p is in || on the circle
    bool contains(Point p) { return dcmp(sqLength(p - o) - r * r) <= 0; }
    // returns a point on the circle rad radians away from +X CCW
    Point point(Tf rad) {
        static_assert(is_same<Tf, Ti>::value);
        return Point(o.x + cos(rad) * r, o.y + sin(rad) * r);
    }
    // area of a circular sector with central angle rad
    Tf area(Tf rad = PI + PI) { return rad * r * r / 2; }
    // area of the circular sector cut by a chord with central angle alpha
    Tf sector(Tf alpha) { return r * r * 0.5 * (alpha - sin(alpha)); }
};
\end{lstlisting}
\subsection{Linear}
\begin{lstlisting}
// **** LINE LINE INTERSECTION START ****
// returns true if point p is on segment s
bool onSegment(Point p, Segment s) {
  return dcmp(cross(s.a - p, s.b - p)) == 0 && dcmp(dot(s.a - p, s.b - p)) <= 0;
}
// returns true if segment p && q touch or intersect
bool segmentsIntersect(Segment p, Segment q) {
  if (onSegment(p.a, q) || onSegment(p.b, q)) return true;
  if (onSegment(q.a, p) || onSegment(q.b, p)) return true;

  Ti c1 = cross(p.b - p.a, q.a - p.a);
  Ti c2 = cross(p.b - p.a, q.b - p.a);
  Ti c3 = cross(q.b - q.a, p.a - q.a);
  Ti c4 = cross(q.b - q.a, p.b - q.a);
  return dcmp(c1) * dcmp(c2) < 0 && dcmp(c3) * dcmp(c4) < 0;
}
bool linesParallel(Line p, Line q) {
  return dcmp(cross(p.b - p.a, q.b - q.a)) == 0;
}
// lines are represented as a ray from a point: (point, vector)
// returns false if two lines (p, v) && (q, w) are parallel or collinear
// true otherwise, intersection point is stored at o via reference, Tf Ti Same
bool lineLineIntersection(Point p, Point v, Point q, Point w, Point& o) {
  if (dcmp(cross(v, w)) == 0) return false;
  Point u = p - q;
  o = p + v * (cross(w, u) / cross(v, w));
  return true;
}
// returns false if two lines p && q are parallel or collinear
// true otherwise, intersection point is stored at o via reference
bool lineLineIntersection(Line p, Line q, Point& o) {
  return lineLineIntersection(p.a, p.b - p.a, q.a, q.b - q.a, o);
}
// returns the distance from point a to line l
// **** LINE LINE INTERSECTION FINISH ****
Tf distancePointLine(Point p, Line l) {
  return abs(cross(l.b - l.a, p - l.a) / length(l.b - l.a));
}
// returns the shortest distance from point a to segment s
Tf distancePointSegment(Point p, Segment s) {
  if (s.a == s.b) return length(p - s.a);
  Point v1 = s.b - s.a, v2 = p - s.a, v3 = p - s.b;
  if (dcmp(dot(v1, v2)) < 0)
    return length(v2);
  else if (dcmp(dot(v1, v3)) > 0)
    return length(v3);
  else
    return abs(cross(v1, v2) / length(v1));
}
// returns the shortest distance from segment p to segment q
Tf distanceSegmentSegment(Segment p, Segment q) {
  if (segmentsIntersect(p, q)) return 0;
  Tf ans = distancePointSegment(p.a, q);
  ans = min(ans, distancePointSegment(p.b, q));
  ans = min(ans, distancePointSegment(q.a, p));
  ans = min(ans, distancePointSegment(q.b, p));
  return ans;
}
// returns the projection of point p on line l, Tf Ti Same
Point projectPointLine(Point p, Line l) {
  Point v = l.b - l.a;
  return l.a + v * ((Tf)dot(v, p - l.a) / dot(v, v));
}
\end{lstlisting}
\subsection{Circular}
\begin{lstlisting}
// Extremely inaccurate for finding near touches
// compute intersection of line l with circle c
// The intersections are given in order of the ray (l.a, l.b), Tf Ti same
vector<Point> circleLineIntersection(Circle c, Line l) {
    vector<Point> ret;
    Point b = l.b - l.a, a = l.a - c.o;
    Tf A = dot(b, b), B = dot(a, b);
    Tf C = dot(a, a) - c.r * c.r, D = B * B - A * C;
    if (D < -EPS) return ret;
    ret.push_back(l.a + b * (-B - sqrt(D + EPS)) / A);
    if (D > EPS) ret.push_back(l.a + b * (-B + sqrt(D)) / A);
    return ret;
}
// signed area of intersection of circle(c.o, c.r) &&
// triangle(c.o, s.a, s.b) [cross(a-o, b-o)/2]
Tf circleTriangleIntersectionArea(Circle c, Segment s) {
    using Linear::distancePointSegment;
    Tf OA = length(c.o - s.a);
    Tf OB = length(c.o - s.b);
    // sector
    if (dcmp(distancePointSegment(c.o, s) - c.r) >= 0)
        return angleBetween(s.a - c.o, s.b - c.o) * (c.r * c.r) / 2.0;
    // triangle
    if (dcmp(OA - c.r) <= 0 && dcmp(OB - c.r) <= 0)
        return cross(c.o - s.b, s.a - s.b) / 2.0;
    // three part: (A, a) (a, b) (b, B)
    vector<Point> Sect = circleLineIntersection(c, s);
    return circleTriangleIntersectionArea(c, Segment(s.a, Sect[0])) +
                  circleTriangleIntersectionArea(c, Segment(Sect[0], Sect[1])) +
                  circleTriangleIntersectionArea(c, Segment(Sect[1], s.b));
}
// area of intersecion of circle(c.o, c.r) && simple polyson(p[])
Tf circlePolyIntersectionArea(Circle c, Polygon p) {
    Tf res = 0;
    int n = p.size();
    for (int i = 0; i < n; ++i)
        res += circleTriangleIntersectionArea(c, Segment(p[i], p[(i + 1) % n]));
    return abs(res);
}
// locates circle c2 relative to c1
// interior             (d < R - r)         ----> -2
// interior tangents (d = R - r)         ----> -1
// concentric        (d = 0)
// secants             (R - r < d < R + r) ---->  0
// exterior tangents (d = R + r)         ---->  1
// exterior             (d > R + r)         ---->  2
int circleCirclePosition(Circle c1, Circle c2) {
    Tf d = length(c1.o - c2.o);
    int in = dcmp(d - abs(c1.r - c2.r)), ex = dcmp(d - (c1.r + c2.r));
    return in < 0 ? -2 : in == 0 ? -1 : ex == 0 ? 1 : ex > 0 ? 2 : 0;
}
// compute the intersection points between two circles c1 && c2, Tf Ti same
vector<Point> circleCircleIntersection(Circle c1, Circle c2) {
    vector<Point> ret;
    Tf d = length(c1.o - c2.o);
    if (dcmp(d) == 0) return ret;
    if (dcmp(c1.r + c2.r - d) < 0) return ret;
    if (dcmp(abs(c1.r - c2.r) - d) > 0) return ret;

    Point v = c2.o - c1.o;
    Tf co = (c1.r * c1.r + sqLength(v) - c2.r * c2.r) / (2 * c1.r * length(v));
    Tf si = sqrt(abs(1.0 - co * co));
    Point p1 = scale(rotatePrecise(v, co, -si), c1.r) + c1.o;
    Point p2 = scale(rotatePrecise(v, co, si), c1.r) + c1.o;

    ret.push_back(p1);
    if (p1 != p2) ret.push_back(p2);
    return ret;
}
// intersection area between two circles c1, c2
Tf circleCircleIntersectionArea(Circle c1, Circle c2) {
    Point AB = c2.o - c1.o;
    Tf d = length(AB);
    if (d >= c1.r + c2.r) return 0;
    if (d + c1.r <= c2.r) return PI * c1.r * c1.r;
    if (d + c2.r <= c1.r) return PI * c2.r * c2.r;

    Tf alpha1 = acos((c1.r * c1.r + d * d - c2.r * c2.r) / (2.0 * c1.r * d));
    Tf alpha2 = acos((c2.r * c2.r + d * d - c1.r * c1.r) / (2.0 * c2.r * d));
    return c1.sector(2 * alpha1) + c2.sector(2 * alpha2);
}
// returns tangents from a point p to circle c, Tf Ti same
vector<Point> pointCircleTangents(Point p, Circle c) {
    vector<Point> ret;
    Point u = c.o - p;
    Tf d = length(u);
    if (d < c.r)
        ;
    else if (dcmp(d - c.r) == 0) {
        ret = {rotate(u, PI / 2)};
    } else {
        Tf ang = asin(c.r / d);
        ret = {rotate(u, -ang), rotate(u, ang)};
    }
    return ret;
}
// returns the points on tangents that touches the circle, Tf Ti Same
vector<Point> pointCircleTangencyPoints(Point p, Circle c) {
    Point u = p - c.o;
    Tf d = length(u);
    if (d < c.r)
        return {};
    else if (dcmp(d - c.r) == 0)
        return {c.o + u};
    else {
        Tf ang = acos(c.r / d);
        u = u / length(u) * c.r;
        return {c.o + rotate(u, -ang), c.o + rotate(u, ang)};
    }
}
// for two circles c1 && c2, returns two list of points a && b
// such that a[i] is on c1 && b[i] is c2 && for every i
// Line(a[i], b[i]) is a tangent to both circles
// CAUTION: a[i] = b[i] in case they touch | -1 for c1 = c2
int circleCircleTangencyPoints(Circle c1, Circle c2, vector<Point> &a,
                                                              vector<Point> &b) {
    a.clear(), b.clear();
    int cnt = 0;
    if (dcmp(c1.r - c2.r) < 0) {
        swap(c1, c2);
        swap(a, b);
    }
    Tf d2 = sqLength(c1.o - c2.o);
    Tf rdif = c1.r - c2.r, rsum = c1.r + c2.r;
    if (dcmp(d2 - rdif * rdif) < 0) return 0;
    if (dcmp(d2) == 0 && dcmp(c1.r - c2.r) == 0) return -1;

    Tf base = angle(c2.o - c1.o);
    if (dcmp(d2 - rdif * rdif) == 0) {
        a.push_back(c1.point(base));
        b.push_back(c2.point(base));
        cnt++;
        return cnt;
    }

    Tf ang = acos((c1.r - c2.r) / sqrt(d2));
    a.push_back(c1.point(base + ang));
    b.push_back(c2.point(base + ang));
    cnt++;
    a.push_back(c1.point(base - ang));
    b.push_back(c2.point(base - ang));
    cnt++;

    if (dcmp(d2 - rsum * rsum) == 0) {
        a.push_back(c1.point(base));
        b.push_back(c2.point(PI + base));
        cnt++;
    } else if (dcmp(d2 - rsum * rsum) > 0) {
        Tf ang = acos((c1.r + c2.r) / sqrt(d2));
        a.push_back(c1.point(base + ang));
        b.push_back(c2.point(PI + base + ang));
        cnt++;
        a.push_back(c1.point(base - ang));
        b.push_back(c2.point(PI + base - ang));
        cnt++;
    }
    return cnt;
}
\end{lstlisting}
\subsection{Convex}
\begin{lstlisting}
/// minkowski sum of two polygons in O(n)
Polygon minkowskiSum(Polygon A, Polygon B) {
    int n = A.size(), m = B.size();
    rotate(A.begin(), min_element(A.begin(), A.end()), A.end());
    rotate(B.begin(), min_element(B.begin(), B.end()), B.end());

    A.push_back(A[0]);
    B.push_back(B[0]);
    for (int i = 0; i < n; i++) A[i] = A[i + 1] - A[i];
    for (int i = 0; i < m; i++) B[i] = B[i + 1] - B[i];

    Polygon C(n + m + 1);
    C[0] = A.back() + B.back();
    merge(A.begin(), A.end() - 1, B.begin(), B.end() - 1, C.begin() + 1,
                polarComp(Point(0, 0), Point(0, -1)));
    for (int i = 1; i < C.size(); i++) C[i] = C[i] + C[i - 1];
    C.pop_back();
    return C;
}
// finds the rectangle with minimum area enclosing a convex polygon and
// the rectangle with minimum perimeter enclosing a convex polygon
// Tf Ti Same
pair<Tf, Tf> rotatingCalipersBoundingBox(const Polygon &p) {
    using Linear::distancePointLine;
    int n = p.size();
    int l = 1, r = 1, j = 1;
    Tf area = 1e100;
    Tf perimeter = 1e100;
    for (int i = 0; i < n; i++) {
        Point v = (p[(i + 1) % n] - p[i]) / length(p[(i + 1) % n] - p[i]);
        while (dcmp(dot(v, p[r % n] - p[i]) - dot(v, p[(r + 1) % n] - p[i])) < 0)
            r++;
        while (j < r || dcmp(cross(v, p[j % n] - p[i]) -
                                                  cross(v, p[(j + 1) % n] - p[i])) < 0)
            j++;
        while (l < j ||
                      dcmp(dot(v, p[l % n] - p[i]) - dot(v, p[(l + 1) % n] - p[i])) > 0)
            l++;
        Tf w = dot(v, p[r % n] - p[i]) - dot(v, p[l % n] - p[i]);
        Tf h = distancePointLine(p[j % n], Line(p[i], p[(i + 1) % n]));
        area = min(area, w * h);
        perimeter = min(perimeter, 2 * w + 2 * h);
    }
    return make_pair(area, perimeter);
}
// returns the left side of polygon u after cutting it by ray a->b
Polygon cutPolygon(Polygon u, Point a, Point b) {
    using Linear::lineLineIntersection;
    using Linear::onSegment;

    Polygon ret;
    int n = u.size();
    for (int i = 0; i < n; i++) {
        Point c = u[i], d = u[(i + 1) % n];
        if (dcmp(cross(b - a, c - a)) >= 0) ret.push_back(c);
        if (dcmp(cross(b - a, d - c)) != 0) {
            Point t;
            lineLineIntersection(a, b - a, c, d - c, t);
            if (onSegment(t, Segment(c, d))) ret.push_back(t);
        }
    }
    return ret;
}
// returns true if point p is in or on triangle abc
bool pointInTriangle(Point a, Point b, Point c, Point p) {
    return dcmp(cross(b - a, p - a)) >= 0 && dcmp(cross(c - b, p - b)) >= 0 &&
                  dcmp(cross(a - c, p - c)) >= 0;
}
// pt must be in ccw order with no three collinear points
// returns inside = -1, on = 0, outside = 1
int pointInConvexPolygon(const Polygon &pt, Point p) {
    int n = pt.size();
    assert(n >= 3);

    int lo = 1, hi = n - 1;
    while (hi - lo > 1) {
        int mid = (lo + hi) / 2;
        if (dcmp(cross(pt[mid] - pt[0], p - pt[0])) > 0)
            lo = mid;
        else
            hi = mid;
    }

    bool in = pointInTriangle(pt[0], pt[lo], pt[hi], p);
    if (!in) return 1;

    if (dcmp(cross(pt[lo] - pt[lo - 1], p - pt[lo - 1])) == 0) return 0;
    if (dcmp(cross(pt[hi] - pt[lo], p - pt[lo])) == 0) return 0;
    if (dcmp(cross(pt[hi] - pt[(hi + 1) % n], p - pt[(hi + 1) % n])) == 0)
        return 0;
    return -1;
}
// Extreme Point for a direction is the farthest point in that direction
// u is the direction for extremeness
int extremePoint(const Polygon &poly, Point u) {
    int n = (int)poly.size();
    int a = 0, b = n;
    while (b - a > 1) {
        int c = (a + b) / 2;
        if (dcmp(dot(poly[c] - poly[(c + 1) % n], u)) >= 0 &&
                dcmp(dot(poly[c] - poly[(c - 1 + n) % n], u)) >= 0) {
            return c;
        }

        bool a_up = dcmp(dot(poly[(a + 1) % n] - poly[a], u)) >= 0;
        bool c_up = dcmp(dot(poly[(c + 1) % n] - poly[c], u)) >= 0;
        bool a_above_c = dcmp(dot(poly[a] - poly[c], u)) > 0;

        if (a_up && !c_up)
            b = c;
        else if (!a_up && c_up)
            a = c;
        else if (a_up && c_up) {
            if (a_above_c)
                b = c;
            else
                a = c;
        } else {
            if (!a_above_c)
                b = c;
            else
                a = c;
        }
    }

    if (dcmp(dot(poly[a] - poly[(a + 1) % n], u)) > 0 &&
            dcmp(dot(poly[a] - poly[(a - 1 + n) % n], u)) > 0)
        return a;
    return b % n;
}
// For a convex polygon p and a line l, returns a list of segments
// of p that touch or intersect line l.
// the i'th segment is considered (p[i], p[(i + 1) modulo |p|])
// #1 If a segment is collinear with the line, only that is returned
// #2 Else if l goes through i'th point, the i'th segment is added
// Complexity: O(lg |p|)
vector<int> lineConvexPolyIntersection(const Polygon &p, Line l) {
    assert((int)p.size() >= 3);
    assert(l.a != l.b);

    int n = p.size();
    vector<int> ret;

    Point v = l.b - l.a;
    int lf = extremePoint(p, rotate90(v));
    int rt = extremePoint(p, rotate90(v) * Ti(-1));
    int olf = orient(l.a, l.b, p[lf]);
    int ort = orient(l.a, l.b, p[rt]);

    if (!olf || !ort) {
        int idx = (!olf ? lf : rt);
        if (orient(l.a, l.b, p[(idx - 1 + n) % n]) == 0)
            ret.push_back((idx - 1 + n) % n);
        else
            ret.push_back(idx);
        return ret;
    }
    if (olf == ort) return ret;

    for (int i = 0; i < 2; ++i) {
        int lo = i ? rt : lf;
        int hi = i ? lf : rt;
        int olo = i ? ort : olf;

        while (true) {
            int gap = (hi - lo + n) % n;
            if (gap < 2) break;

            int mid = (lo + gap / 2) % n;
            int omid = orient(l.a, l.b, p[mid]);
            if (!omid) {
                lo = mid;
                break;
            }
            if (omid == olo)
                lo = mid;
            else
                hi = mid;
        }
        ret.push_back(lo);
    }
    return ret;
}
// Calculate [ACW, CW] tangent pair from an external point
constexpr int CW = -1, ACW = 1;
bool isGood(Point u, Point v, Point Q, int dir) {
    return orient(Q, u, v) != -dir;
}
Point better(Point u, Point v, Point Q, int dir) {
    return orient(Q, u, v) == dir ? u : v;
}
Point pointPolyTangent(const Polygon &pt, Point Q, int dir, int lo, int hi) {
    while (hi - lo > 1) {
        int mid = (lo + hi) / 2;
        bool pvs = isGood(pt[mid], pt[mid - 1], Q, dir);
        bool nxt = isGood(pt[mid], pt[mid + 1], Q, dir);

        if (pvs && nxt) return pt[mid];
        if (!(pvs || nxt)) {
            Point p1 = pointPolyTangent(pt, Q, dir, mid + 1, hi);
            Point p2 = pointPolyTangent(pt, Q, dir, lo, mid - 1);
            return better(p1, p2, Q, dir);
        }

        if (!pvs) {
            if (orient(Q, pt[mid], pt[lo]) == dir)
                hi = mid - 1;
            else if (better(pt[lo], pt[hi], Q, dir) == pt[lo])
                hi = mid - 1;
            else
                lo = mid + 1;
        }
        if (!nxt) {
            if (orient(Q, pt[mid], pt[lo]) == dir)
                lo = mid + 1;
            else if (better(pt[lo], pt[hi], Q, dir) == pt[lo])
                hi = mid - 1;
            else
                lo = mid + 1;
        }
    }

    Point ret = pt[lo];
    for (int i = lo + 1; i <= hi; i++) ret = better(ret, pt[i], Q, dir);
    return ret;
}
// [ACW, CW] Tangent
pair<Point, Point> pointPolyTangents(const Polygon &pt, Point Q) {
    int n = pt.size();
    Point acw_tan = pointPolyTangent(pt, Q, ACW, 0, n - 1);
    Point cw_tan = pointPolyTangent(pt, Q, CW, 0, n - 1);
    return make_pair(acw_tan, cw_tan);
}
\end{lstlisting}
\subsection{Polygon}
\begin{lstlisting}
typedef vector<Point> Polygon;
// removes redundant colinear points
// polygon can't be all colinear points
Polygon RemoveCollinear(const Polygon &poly) {
    Polygon ret;
    int n = poly.size();
    for (int i = 0; i < n; i++) {
        Point a = poly[i];
        Point b = poly[(i + 1) % n];
        Point c = poly[(i + 2) % n];
        if (dcmp(cross(b - a, c - b)) != 0 && (ret.empty() || b != ret.back()))
            ret.push_back(b);
    }
    return ret;
}
// returns the signed area of polygon p of n vertices
Tf signedPolygonArea(const Polygon &p) {
    Tf ret = 0;
    for (int i = 0; i < (int)p.size() - 1; i++)
        ret += cross(p[i] - p[0], p[i + 1] - p[0]);
    return ret / 2;
}
// given a polygon p of n vertices, generates the convex hull in in CCW
// Tested on https://acm.timus.ru/problem.aspx?space=1&num=1185
// Caution: when all points are colinear AND removeRedundant == false
// output will be contain duplicate points (from upper hull) at back
Polygon convexHull(Polygon p, bool removeRedundant) {
    int check = removeRedundant ? 0 : -1;
    sort(p.begin(), p.end());
    p.erase(unique(p.begin(), p.end()), p.end());

    int n = p.size();
    Polygon ch(n + n);
    int m = 0;  // preparing lower hull
    for (int i = 0; i < n; i++) {
        while (m > 1 &&
                      dcmp(cross(ch[m - 1] - ch[m - 2], p[i] - ch[m - 1])) <= check)
            m--;
        ch[m++] = p[i];
    }
    int k = m;  // preparing upper hull
    for (int i = n - 2; i >= 0; i--) {
        while (m > k &&
                      dcmp(cross(ch[m - 1] - ch[m - 2], p[i] - ch[m - 2])) <= check)
            m--;
        ch[m++] = p[i];
    }
    if (n > 1) m--;
    ch.resize(m);
    return ch;
}
// returns inside = -1, on = 0, outside = 1
int pointInPolygon(const Polygon &p, Point o) {
    using Linear::onSegment;
    int wn = 0, n = p.size();
    for (int i = 0; i < n; i++) {
        int j = (i + 1) % n;
        if (onSegment(o, Segment(p[i], p[j])) || o == p[i]) return 0;
        int k = dcmp(cross(p[j] - p[i], o - p[i]));
        int d1 = dcmp(p[i].y - o.y);
        int d2 = dcmp(p[j].y - o.y);
        if (k > 0 && d1 <= 0 && d2 > 0) wn++;
        if (k < 0 && d2 <= 0 && d1 > 0) wn--;
    }
    return wn ? -1 : 1;
}
// Given a simple polygon p, and a line l, returns (x, y)
// x = longest segment of l in p, y = total length of l in p.
pair<Tf, Tf> linePolygonIntersection(Line l, const Polygon &p) {
    using Linear::lineLineIntersection;
    int n = p.size();
    vector<pair<Tf, int>> ev;
    for (int i = 0; i < n; ++i) {
        Point a = p[i], b = p[(i + 1) % n], z = p[(i - 1 + n) % n];
        int ora = orient(l.a, l.b, a), orb = orient(l.a, l.b, b),
                orz = orient(l.a, l.b, z);
        if (!ora) {
            Tf d = dot(a - l.a, l.b - l.a);
            if (orz && orb) {
                if (orz != orb) ev.emplace_back(d, 0);
                // else  // Point Touch
            } else if (orz)
                ev.emplace_back(d, orz);
            else if (orb)
                ev.emplace_back(d, orb);
        } else if (ora == -orb) {
            Point ins;
            lineLineIntersection(l, Line(a, b), ins);
            ev.emplace_back(dot(ins - l.a, l.b - l.a), 0);
        }
    }
    sort(ev.begin(), ev.end());

    Tf ans = 0, len = 0, last = 0, tot = 0;
    bool active = false;
    int sign = 0;
    for (auto &qq : ev) {
        int tp = qq.second;
        Tf d = qq.first;  /// current Segment is (last, d)
        if (sign) {       /// On Border
            len += d - last;
            tot += d - last;
            ans = max(ans, len);
            if (tp != sign) active = !active;
            sign = 0;
        } else {
            if (active) {  /// Strictly Inside
                len += d - last;
                tot += d - last;
                ans = max(ans, len);
            }
            if (tp == 0)
                active = !active;
            else
                sign = tp;
        }
        last = d;
        if (!active) len = 0;
    }
    ans /= length(l.b - l.a);
    tot /= length(l.b - l.a);
    return {ans, tot};
}
\end{lstlisting}
\subsection{Half Plane}
\begin{lstlisting}
using Linear::lineLineIntersection;
struct DirLine {
    Point p, v;
    Tf ang;
    DirLine() {}
    /// Directed line containing point P in the direction v
    DirLine(Point p, Point v) : p(p), v(v) { ang = atan2(v.y, v.x); }
    bool operator<(const DirLine& u) const { return ang < u.ang; }
};
// returns true if point p is on the ccw-left side of ray l
bool onLeft(DirLine l, Point p) { return dcmp(cross(l.v, p - l.p)) >= 0; }

// Given a set of directed lines returns a polygon such that
// the polygon is the intersection by halfplanes created by the
// left side of the directed lines. MAY CONTAIN DUPLICATE POINTS
int halfPlaneIntersection(vector<DirLine>& li, Polygon& poly) {
    int n = li.size();
    sort(li.begin(), li.end());

    int first, last;
    Point* p = new Point[n];
    DirLine* q = new DirLine[n];
    q[first = last = 0] = li[0];

    for (int i = 1; i < n; i++) {
        while (first < last && !onLeft(li[i], p[last - 1])) last--;
        while (first < last && !onLeft(li[i], p[first])) first++;
        q[++last] = li[i];

        if (dcmp(cross(q[last].v, q[last - 1].v)) == 0) {
            last--;
            if (onLeft(q[last], li[i].p)) q[last] = li[i];
        }

        if (first < last)
            lineLineIntersection(q[last - 1].p, q[last - 1].v, q[last].p, q[last].v,
                                                      p[last - 1]);
    }

    while (first < last && !onLeft(q[first], p[last - 1])) last--;
    if (last - first <= 1) {
        delete[] p;
        delete[] q;
        poly.clear();
        return 0;
    }
    lineLineIntersection(q[last].p, q[last].v, q[first].p, q[first].v, p[last]);

    int m = 0;
    poly.resize(last - first + 1);
    for (int i = first; i <= last; i++) poly[m++] = p[i];
    delete[] p;
    delete[] q;
    return m;
}
\end{lstlisting}
\end{multicols*}
\begin{multicols*}{3}
\newpage
\section{Equations and Formulas}
\subsection{Catalan Numbers}
$\displaystyle C_n=\frac{1}{n+1}{2n \choose n}$
$\displaystyle C_0=1,C_1=1\text{ and }C_n=\sum \limits_{k=0}^{n-1}C_k C_{n-1-k}$ \\
The number of ways to completely parenthesize $n$+$\displaystyle 1$ factors. \\
The number of triangulations of a convex polygon with $n$+$\displaystyle 2$ sides (i.e. the number of partitions of polygon into disjoint triangles by using the diagonals). \\
The number of ways to connect the $\displaystyle 2n$ points on a circle to form $n$ disjoint i.e. non-intersecting chords. \\
The number of rooted full binary trees with $n$+$\displaystyle 1$ leaves (vertices are not numbered). A rooted binary tree is full if every vertex has either two children or no children. \\
Number of permutations of $\displaystyle {1, …, n}$ that avoid the pattern $\displaystyle 123$ (or any of the other patterns of length $3$); that is, the number of permutations with no three-term increasing sub-sequence. For $n = 3$, these permutations are $\displaystyle 132,\ 213,\ 231,\ 312$ and $321.$

\subsection{Stirling Numbers First Kind}
The Stirling numbers of the first kind count permutations according to their number of cycles (counting fixed points as cycles of length one). \\
$S(n,k)$ counts the number of permutations of $n$ elements with $\displaystyle \displaystyle k$ disjoint cycles. \\
$S(n,k)=(n-1) \cdot S(n-1,k)+S(n-1,k-1),$ \(where,\; S(0,0)=1,S(n,0)=S(0,n)=0\)
$\displaystyle \displaystyle\sum_{k=0}^{n}S(n,k) = n!$ \\
The unsigned Stirling numbers may also be defined algebraically, as the coefficient of the rising factorial:
\[\displaystyle x^{\bar{n}} = x(x+1)...(x+n-1) = \sum_{k=0}^{n}{ S(n, k) x^k}\]
Lets $[n, k]$ be the stirling number of the first kind, then

\[\displaystyle \bigl[\!\begin{smallmatrix} n \\ n\ -\ k \end{smallmatrix}\!\bigr] = \sum_{0 \leq i_1 < i_2 < i_k < n}{i_1i_2....i_k.}\]

\subsection{Stirling Numbers Second Kind}
Stirling number of the second kind is the number of ways to partition a set of n objects into k non-empty subsets. \\
$S(n,k)=k \cdot S(n-1,k)+S(n-1,k-1)$, \(where \; S(0,0)=1,S(n,0)=S(0,n)=0\)
$S(n,2)=2^{n-1}-1$ 
$S(n,k) \cdot k!$ = number of ways to color $n$ nodes using colors from $\displaystyle 1$ to $\displaystyle \displaystyle k$ such that each color is used at least once. \\
An $r$-associated Stirling number of the second kind is the number of ways to partition a set of $n$ objects into $\displaystyle \displaystyle k$ subsets, with each subset containing at least $r$ elements. It is denoted by $S_r( n , k )$ and obeys the recurrence relation. $\displaystyle \displaystyle S_r(n+1,k) = k S_r(n,k) + \binom{n}{r-1} S_r(n-r+1,k-1)$ \\ 
Denote the n objects to partition by the integers $\displaystyle 1, 2, …., n$. Define the reduced Stirling numbers of the second kind, denoted $S^d(n, k)$, to be the number of ways to partition the integers $\displaystyle 1, 2, …., n$ into k nonempty subsets such that all elements in each subset have pairwise distance at least d. That is, for any integers i and j in a given subset, it is required that $|i - j| \geq d$. It has been shown that these numbers satisfy, \(S^d(n, k) = S(n - d + 1, k - d + 1), n \geq k \geq d\)
\subsection{Other Combinatorial Identities}
$\displaystyle \displaystyle {n \choose k}=\frac{n}{k}{n-1 \choose k-1}$ \\
$\displaystyle \sum \limits_{i= 0}^k{n+i \choose i}= \sum \limits_{i= 0}^k{n+i \choose n} = {n+k+1 \choose k}$ \\
$\displaystyle \ n,r \in N, n > r, \sum \limits_{i=r}^n{i \choose r}={n+1 \choose r+1}$ \\
If $\displaystyle P(n)=\sum_{k=0}^{n}{n \choose k} \cdot Q(k)$, then,
\[Q(n)=\sum_{k=0}^{n}(-1)^{n-k}{n \choose k} \cdot P(k)\] \\
If $\displaystyle P(n)=\sum_{k=0}^{n}(-1)^{k}{n \choose k} \cdot Q(k)$ , then,
\[Q(n)=\sum_{k=0}^{n}(-1)^{k}{n \choose k} \cdot P(k)\]

\subsection{Different Math Formulas}
\textbf{Picks Theorem : } $ A = i + b / 2 - 1 $ \\ 
\textbf{Deragements : } $ d(i) = (i - 1) \times \left( d(i - 1) + d(i - 2) \right) $ \\ 
\begin{multline*}
\displaystyle \frac{n}{ab}-\Big\{\frac{b{\prime} n}{a}\Big\}-\Big\{\frac{a{\prime} n}{b}\Big\} + 1
\end{multline*}
\subsection{GCD and LCM}
if $m$ is any integer, then $\displaystyle \gcd(a + m {\cdot} b, b) = \gcd(a, b)$ \\
The gcd is a multiplicative function in the following sense: if $\displaystyle a_1$ and $\displaystyle a_2$ are relatively prime, then $\displaystyle \gcd(a_1 \cdot a_2, b) = \gcd(a_1, b) \cdot \gcd(a_2,b )$. \\
$\displaystyle \gcd(a, \operatorname{lcm}(b, c)) = \operatorname{lcm}(\gcd(a, b), \gcd(a, c))$. \\
$\displaystyle \operatorname{lcm}(a, \gcd(b, c)) = \gcd(\operatorname{lcm}(a, b), \operatorname{lcm}(a, c))$. \\
For non-negative integers $\displaystyle a$ and $b$, where $\displaystyle a$ and $b$ are not both zero, $\displaystyle \gcd({n^a} - 1, {n^b} - 1) = n^{\gcd(a,b)} - 1$ \\
$\displaystyle \gcd(a, b) = \displaystyle \sum_{k|a \, \text{and} \, k|b} {\phi(k)}$ \\
$\displaystyle \displaystyle \sum_{i=1}^n [\gcd(i, n) = k] = { \phi{\left(\frac{n}{k}\right)}}$ \\
$\displaystyle \displaystyle \sum_{k=1}^n \gcd(k, n) = \displaystyle \sum_{d|n} d \cdot {\phi{\left(\frac{n}{d}\right)}}$ \\
$\displaystyle \displaystyle \sum_{k=1}^n x^{\gcd(k,n)} = \displaystyle \sum_{d|n} x^d \cdot {\phi{\left(\frac{n}{d}\right)}}$ \\
$\displaystyle \displaystyle \sum_{k=1}^n \frac{1}{\gcd(k, n)} = \displaystyle \sum_{d|n} \frac{1}{d} \cdot {\phi{\left(\frac{n}{d}\right)}} = \frac{1}{n} \displaystyle \sum_{d|n} d \cdot \phi(d)$ \\
$\displaystyle \displaystyle \sum_{k=1}^n \frac{k}{\gcd(k, n)} = \frac{n}{2} \cdot \displaystyle \sum_{d|n} \frac{1}{d} \cdot {\phi{\left(\frac{n}{d}\right)}} = \frac{n}{2} \cdot \frac{1}{n} \cdot \displaystyle \sum_{d|n} d \cdot \phi(d)$ \\
$\displaystyle \displaystyle \sum_{k=1}^n \frac{n}{\gcd(k, n)} = 2 * \displaystyle \sum_{k=1}^n \frac{k}{\gcd(k, n)} - 1$, for $n > 1$ \\
$\displaystyle \displaystyle \sum_{i=1}^n \sum_{j=1}^n [\gcd(i, j) = 1] = \displaystyle \sum_{d=1}^n \mu(d) \lfloor {\frac{n}{d} \rfloor}^2$ \\
$\displaystyle \displaystyle \sum_{i=1}^n \displaystyle\sum_{j=1}^n \gcd(i, j) = \displaystyle \sum_{d=1}^n \phi(d) \lfloor {\frac{n}{d} \rfloor}^2$ \\
$\displaystyle \sum_{i=1}^n \sum_{j=1}^n i \cdot j[\gcd(i, j) = 1] = \sum_{i=1}^n \phi(i)i^2$ \\
$\displaystyle F(n) = \displaystyle \sum_{i=1}^n \displaystyle \sum_{j=1}^n \operatorname{lcm}(i, j) = \displaystyle \sum_{l=1}^n {\left(\frac{\left( 1 + \lfloor{\frac{n}{l} \rfloor} \right) \left( \lfloor{\frac{n}{l} \rfloor} \right)} {2} \right)}^2 \displaystyle \sum_{d|l} \mu(d)ld$ \\



\subsection{Geometry}
\textbf{Cone:} \( V = \frac{1}{3} \pi r^2 h \), \( A = \pi r (r + \sqrt{h^2 + r^2}) \) \\
\textbf{Pyramid:} \( V = \frac{1}{3} \times \text{base} \times \text{height} \), \( A = \text{base area} + \frac{1}{2} \times \text{perimeter} \times \text{slant height} \) \\
\textbf{Triangular Prism:} \( V = \frac{1}{2} \times \text{base} \times \text{height} \times \text{depth} \), \( A = \text{base} \times \text{height} + 3 \times \left(\frac{1}{2} \times \text{side} \times \text{perimeter} \right) \) \\
\textbf{Torus:} \( V = 2 \pi^2 R r^2 \), \( A = 4 \pi^2 R r \) \\
\textbf{Ellipsoid:} \( V = \frac{4}{3} \pi a b c \), \( A = 4 \pi \left( \frac{(ab)^{1.6} + (bc)^{1.6} + (ca)^{1.6}}{3} \right)^{1/1.6} \)



\end{multicols*}

\end{document}
