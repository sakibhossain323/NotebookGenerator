\begin{multicols*}{3}
\newpage
\section{Equations and Formulas}
\subsection{Catalan Numbers}
$\displaystyle C_n=\frac{1}{n+1}{2n \choose n}$
$\displaystyle C_0=1,C_1=1\text{ and }C_n=\sum \limits_{k=0}^{n-1}C_k C_{n-1-k}$ \\
The number of ways to completely parenthesize $n$+$\displaystyle 1$ factors. \\
The number of triangulations of a convex polygon with $n$+$\displaystyle 2$ sides (i.e. the number of partitions of polygon into disjoint triangles by using the diagonals). \\
The number of ways to connect the $\displaystyle 2n$ points on a circle to form $n$ disjoint i.e. non-intersecting chords. \\
The number of rooted full binary trees with $n$+$\displaystyle 1$ leaves (vertices are not numbered). A rooted binary tree is full if every vertex has either two children or no children. \\
Number of permutations of $\displaystyle {1, …, n}$ that avoid the pattern $\displaystyle 123$ (or any of the other patterns of length $3$); that is, the number of permutations with no three-term increasing sub-sequence. For $n = 3$, these permutations are $\displaystyle 132,\ 213,\ 231,\ 312$ and $321.$

\subsection{Stirling Numbers First Kind}
The Stirling numbers of the first kind count permutations according to their number of cycles (counting fixed points as cycles of length one). \\
$S(n,k)$ counts the number of permutations of $n$ elements with $\displaystyle \displaystyle k$ disjoint cycles. \\
$S(n,k)=(n-1) \cdot S(n-1,k)+S(n-1,k-1),$ \(where,\; S(0,0)=1,S(n,0)=S(0,n)=0\)
$\displaystyle \displaystyle\sum_{k=0}^{n}S(n,k) = n!$ \\
The unsigned Stirling numbers may also be defined algebraically, as the coefficient of the rising factorial:
\[\displaystyle x^{\bar{n}} = x(x+1)...(x+n-1) = \sum_{k=0}^{n}{ S(n, k) x^k}\]
Lets $[n, k]$ be the stirling number of the first kind, then

\[\displaystyle \bigl[\!\begin{smallmatrix} n \\ n\ -\ k \end{smallmatrix}\!\bigr] = \sum_{0 \leq i_1 < i_2 < i_k < n}{i_1i_2....i_k.}\]

\subsection{Stirling Numbers Second Kind}
Stirling number of the second kind is the number of ways to partition a set of n objects into k non-empty subsets. \\
$S(n,k)=k \cdot S(n-1,k)+S(n-1,k-1)$, \(where \; S(0,0)=1,S(n,0)=S(0,n)=0\)
$S(n,2)=2^{n-1}-1$ 
$S(n,k) \cdot k!$ = number of ways to color $n$ nodes using colors from $\displaystyle 1$ to $\displaystyle \displaystyle k$ such that each color is used at least once. \\
An $r$-associated Stirling number of the second kind is the number of ways to partition a set of $n$ objects into $\displaystyle \displaystyle k$ subsets, with each subset containing at least $r$ elements. It is denoted by $S_r( n , k )$ and obeys the recurrence relation. $\displaystyle \displaystyle S_r(n+1,k) = k S_r(n,k) + \binom{n}{r-1} S_r(n-r+1,k-1)$ \\ 
Denote the n objects to partition by the integers $\displaystyle 1, 2, …., n$. Define the reduced Stirling numbers of the second kind, denoted $S^d(n, k)$, to be the number of ways to partition the integers $\displaystyle 1, 2, …., n$ into k nonempty subsets such that all elements in each subset have pairwise distance at least d. That is, for any integers i and j in a given subset, it is required that $|i - j| \geq d$. It has been shown that these numbers satisfy, \(S^d(n, k) = S(n - d + 1, k - d + 1), n \geq k \geq d\)
\subsection{Other Combinatorial Identities}
$\displaystyle \displaystyle {n \choose k}=\frac{n}{k}{n-1 \choose k-1}$ \\
$\displaystyle \sum \limits_{i= 0}^k{n+i \choose i}= \sum \limits_{i= 0}^k{n+i \choose n} = {n+k+1 \choose k}$ \\
$\displaystyle \ n,r \in N, n > r, \sum \limits_{i=r}^n{i \choose r}={n+1 \choose r+1}$ \\
If $\displaystyle P(n)=\sum_{k=0}^{n}{n \choose k} \cdot Q(k)$, then,
\[Q(n)=\sum_{k=0}^{n}(-1)^{n-k}{n \choose k} \cdot P(k)\] \\
If $\displaystyle P(n)=\sum_{k=0}^{n}(-1)^{k}{n \choose k} \cdot Q(k)$ , then,
\[Q(n)=\sum_{k=0}^{n}(-1)^{k}{n \choose k} \cdot P(k)\]

\subsection{Different Math Formulas}
\textbf{Picks Theorem : } $ A = i + b / 2 - 1 $ \\ 
\textbf{Deragements : } $ d(i) = (i - 1) \times \left( d(i - 1) + d(i - 2) \right) $ \\ 
\begin{multline*}
\displaystyle \frac{n}{ab}-\Big\{\frac{b{\prime} n}{a}\Big\}-\Big\{\frac{a{\prime} n}{b}\Big\} + 1
\end{multline*}
\subsection{GCD and LCM}
if $m$ is any integer, then $\displaystyle \gcd(a + m {\cdot} b, b) = \gcd(a, b)$ \\
The gcd is a multiplicative function in the following sense: if $\displaystyle a_1$ and $\displaystyle a_2$ are relatively prime, then $\displaystyle \gcd(a_1 \cdot a_2, b) = \gcd(a_1, b) \cdot \gcd(a_2,b )$. \\
$\displaystyle \gcd(a, \operatorname{lcm}(b, c)) = \operatorname{lcm}(\gcd(a, b), \gcd(a, c))$. \\
$\displaystyle \operatorname{lcm}(a, \gcd(b, c)) = \gcd(\operatorname{lcm}(a, b), \operatorname{lcm}(a, c))$. \\
For non-negative integers $\displaystyle a$ and $b$, where $\displaystyle a$ and $b$ are not both zero, $\displaystyle \gcd({n^a} - 1, {n^b} - 1) = n^{\gcd(a,b)} - 1$ \\
$\displaystyle \gcd(a, b) = \displaystyle \sum_{k|a \, \text{and} \, k|b} {\phi(k)}$ \\
$\displaystyle \displaystyle \sum_{i=1}^n [\gcd(i, n) = k] = { \phi{\left(\frac{n}{k}\right)}}$ \\
$\displaystyle \displaystyle \sum_{k=1}^n \gcd(k, n) = \displaystyle \sum_{d|n} d \cdot {\phi{\left(\frac{n}{d}\right)}}$ \\
$\displaystyle \displaystyle \sum_{k=1}^n x^{\gcd(k,n)} = \displaystyle \sum_{d|n} x^d \cdot {\phi{\left(\frac{n}{d}\right)}}$ \\
$\displaystyle \displaystyle \sum_{k=1}^n \frac{1}{\gcd(k, n)} = \displaystyle \sum_{d|n} \frac{1}{d} \cdot {\phi{\left(\frac{n}{d}\right)}} = \frac{1}{n} \displaystyle \sum_{d|n} d \cdot \phi(d)$ \\
$\displaystyle \displaystyle \sum_{k=1}^n \frac{k}{\gcd(k, n)} = \frac{n}{2} \cdot \displaystyle \sum_{d|n} \frac{1}{d} \cdot {\phi{\left(\frac{n}{d}\right)}} = \frac{n}{2} \cdot \frac{1}{n} \cdot \displaystyle \sum_{d|n} d \cdot \phi(d)$ \\
$\displaystyle \displaystyle \sum_{k=1}^n \frac{n}{\gcd(k, n)} = 2 * \displaystyle \sum_{k=1}^n \frac{k}{\gcd(k, n)} - 1$, for $n > 1$ \\
$\displaystyle \displaystyle \sum_{i=1}^n \sum_{j=1}^n [\gcd(i, j) = 1] = \displaystyle \sum_{d=1}^n \mu(d) \lfloor {\frac{n}{d} \rfloor}^2$ \\
$\displaystyle \displaystyle \sum_{i=1}^n \displaystyle\sum_{j=1}^n \gcd(i, j) = \displaystyle \sum_{d=1}^n \phi(d) \lfloor {\frac{n}{d} \rfloor}^2$ \\
$\displaystyle \sum_{i=1}^n \sum_{j=1}^n i \cdot j[\gcd(i, j) = 1] = \sum_{i=1}^n \phi(i)i^2$ \\
$\displaystyle F(n) = \displaystyle \sum_{i=1}^n \displaystyle \sum_{j=1}^n \operatorname{lcm}(i, j) = \displaystyle \sum_{l=1}^n {\left(\frac{\left( 1 + \lfloor{\frac{n}{l} \rfloor} \right) \left( \lfloor{\frac{n}{l} \rfloor} \right)} {2} \right)}^2 \displaystyle \sum_{d|l} \mu(d)ld$ \\



\subsection{Geometry}
\textbf{Cone:} \( V = \frac{1}{3} \pi r^2 h \), \( A = \pi r (r + \sqrt{h^2 + r^2}) \) \\
\textbf{Pyramid:} \( V = \frac{1}{3} \times \text{base} \times \text{height} \), \( A = \text{base area} + \frac{1}{2} \times \text{perimeter} \times \text{slant height} \) \\
\textbf{Triangular Prism:} \( V = \frac{1}{2} \times \text{base} \times \text{height} \times \text{depth} \), \( A = \text{base} \times \text{height} + 3 \times \left(\frac{1}{2} \times \text{side} \times \text{perimeter} \right) \) \\
\textbf{Torus:} \( V = 2 \pi^2 R r^2 \), \( A = 4 \pi^2 R r \) \\
\textbf{Ellipsoid:} \( V = \frac{4}{3} \pi a b c \), \( A = 4 \pi \left( \frac{(ab)^{1.6} + (bc)^{1.6} + (ca)^{1.6}}{3} \right)^{1/1.6} \)



\end{multicols*}
